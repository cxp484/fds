% !TEX root = FDS_Validation_Guide.tex

\chapter{Burning Rate and Fire Spread}

This chapter contains a series of validation exercises where the aim is to {\em predict} the burning and spread rate of the fire. Most of the simulations included in the previous chapters involved a {\em specified} burning or heat release rate. Here, the objective is to apply measured thermophysical properties of the material and predict its burning rate, either with a specified heat flux or as a free burn.

\section{FAA Polymers}\label{sec_FAA_Polymers}

The U.S.~Federal Aviation Administration (FAA) has studied various plastics that are commonly used aboard commercial aircraft.

This section presents measured properties of various polymers and the numerical predictions of their mass loss and/or burning rates under constant heat heating. Two types of experiments are considered. First, the NIST Gasification Apparatus is used to measure the mass loss rate of non-burning samples in a nitrogen atmosphere. Second, the standard Cone Calorimeter~\cite{conecal} is used to measure the heat release rate of materials in a normal atmosphere. When just the mass loss rate of a non-burning sample has been measured, FDS is run in ``solid phase only'' mode; that is, a 1-D heat conduction calculation is performed in a single grid cell. The result is the predicted mass loss rate as a function of time. To simulate a cone calorimeter experiment, FDS simulates the burning of a 10~cm by 10~cm sample with a specified heat flux to represent the effect of the cone heater. The cone itself is not included in the simulation. As the sample burns, FDS predicts the additional radiative and convective heating of the sample as a result of the fire.

In general, the burning/gasification rate of a charring polymer is more difficult to predict than a non-charring one because there are more parameters that need to be measured and
more complicated behavior, like intumescence, need to be considered.

\subsection{Glossary of Terms}
\label{glossary}

\begin{description}
\item[Assumption:]  Characteristics were assumed from known properties in similar materials.
\item[Cone Calorimeter] (ASTM E 1354 \cite{conecal}) The Cone Calorimeter exposes a small sample to a constant external radiant heat flux simulating exposure of the sample to a large scale fire. The device records mass loss data along with heat release data through oxygen consumption calorimetry. From this a variety of heat release related properties can be found including heat of combustion.
\item[Constant Volume:] The material is assumed to maintain a constant volume during the solid phase reactions.
\item[Direct:]  Direct measurement of densities is performed by measuring the dimensions and mass of the sample.
\item[DSC:] (ASTM E 2070 \cite{diffscancal}) A Differential Scanning Calorimeter precisely raises the temperature of a small sample of material at a constant rate. This coupled with knowledge of heat absorbed by the sample allows for the calculation of the specific heat function of a material as well as heats of reaction and phase change.
\item[Estimated:] Characteristics were approximated based on known properties in similar materials.
\item[FTIR:] Fourier Transform Infrared Spectroscopy uses a spectrometer to simultaneously characterize the absorption of all frequencies of infrared light. In testing a sample is exposed to infrared light and a detector records light that has passed through the sample. A Fourier transform of detector measurement is then translated into absorption information.
\item[Gasification Apparatus:] Similar to the Cone Calorimeter however flaming is prevented. This is done typically through the introduction of inert purge gases.
\item[Inherited:] The properties of the product or component are assumed to be the same as the original material.
\item[Inverse Analysis:] Property was established by fitting a model to measured temperatures from the Cone Calorimeter or Gasification Apparatus.
\item[IS:] (ASTM E 1175 \cite{intgsphere}) An Integrating Sphere, or an Ulbricht Sphere, is a hollow cavity whose interior has a high diffuse reflectivity. A sample placed inside the sphere is exposed to incident radiation and reflectivity measured. Emissivity can be determined from this information. The standard above is for measurement of Solar reflectivity, and was not necessarily precisely followed.
\item[Laser Flash:] (ASTM E1461 \cite{laserflash}) In the Laser Flash Method one surface of a sample is rapidly heated using a single pulse from a laser. Heat sensors on the opposite side of the sample record the arrival of the resulting temperature disturbance. From this thermal diffusivity/thermal conductivity can be calculated.
\item[Literature:] Results were found within previously published literature.
\item[MCC:] (ASTM D 7309 \cite{microcc}) The Microscale Combustion Calorimeter (MCC)  rapidly pyrolyzes a milligram size sample in an inert atmosphere. The pyrolyzate is then exposed to an abundance of oxygen.  Heat release history is obtained from oxygen consumption. Similar to TGA with  heat release recorded rather than mass loss rate.
\item[Pulsed Current:] Can refer to different types of tests. Generally, a sample is positioned between two electrodes in a sealed chamber with an inert atmosphere. The sample is heated through pulses of current. Measurements of the sample and the chamber can give information regarding specific heat, emissivity, or other material properties.
\item[TGA:] (ASTM E 1131 \cite{thermalga}) In Thermal Gravimetric Analysis (TGA)  a small sample is heated at uniform rate, generally in a Nitrogen (N$_2$) atmosphere. The percentage weight loss of the sample is recorded relative to the sample's temperature. Rate constants can then be fitted to the data. Similar to MCC with mass loss recorded instead of heat release.
\item[TLS:] (ASTM D 5930 \cite{transline}) The Transient Line Source method records temperature of a single point at a fixed distance in a sample over time using a probe. Given knowledge of the heat exposure of the sample the thermal conductivity can be found from the slope of the recorded data.
\end{description}



\newpage

\subsection{Non-Charring Polymers, HDPE, HIPS, and PMMA}

A non-charring polymer is considered one of the easiest solids to model because it typically involves only a single, first order reaction that converts solid plastic to fuel vapor. No residue is formed and the plastic is completely pyrolyzed. Table~\ref{FAA_Properties} lists nine parameters for each polymer studied. These values have been input directly into FDS, and the predicted mass loss rates are compared with measured values from the NIST Gasification Apparatus, a device that pyrolyzes the solid in a nitrogen environment to prevent combustion of fuel gases. The results are shown in Fig.~\ref{FAA_Polymers}. The exposing heat flux was 52~kW/m$^2$. A 1~cm layer of insulation was placed under the sample. Its properties are given in Ref.~\cite{Stoliarov:CF2009}.

\begin{table}[h!]
\caption[FAA non-charring polymer properties]{Input parameters for FAA Polymers non-charring samples. Courtesy S.~Stoliarov, M.~McKinnon and J.~Li, University of Maryland.
See Sec.~\ref{glossary} for an explanation of terms.}
\begin{center}
\begin{tabular}{|l|c|c|c|c|c|l|l|}
\hline
Property                    & Units         & HDPE                  & HIPS                  & PMMA                  & Unc. (\%) & Method                &  Ref.                         \\ \hline \hline
Density                     & kg/m$^3$      & 860                   & 950                   & 1100                  & 5         & Direct                &  \cite{Stoliarov:CF2009}      \\ \hline
Conductivity                & W/m/K         & 0.29                  & 0.22                  & 0.20                  & 15        & TLC                   &  \cite{Stoliarov:CF2009}      \\ \hline
Specific Heat               & kJ/kg/K       & 3.5                   & 2.0                   & 2.2                   & 15        & DSC                   &  \cite{Stoliarov:PDS2008}     \\ \hline
Emissivity                  &               & 0.92                  & 0.86                  & 0.85                  & 20        & IS                    &  \cite{Hallman:PES1974}       \\ \hline
Absorption Coef.            & m$^{-1}$      & 1300                  & 2700                  & 2700                  & 50        & FTIR                  &  \cite{Tsilingiris:ECM2003}   \\ \hline
Pre-Exp.~Factor             & s$^{-1}$      & $4.8 \times 10^{22}$  & $1.2 \times 10^{16}$  & $8.5 \times 10^{12}$  & 50        & TGA                   &  \cite{Stoliarov:CF2009}      \\ \hline
Activation Energy           & J/mol       & $3.49 \times 10^{5}$  & $2.47 \times 10^{5}$  & $1.88 \times 10^{5}$  & 3         & TGA                   &  \cite{Stoliarov:CF2009}      \\ \hline
Heat of Reaction            & kJ/kg         & 920                   & 1000                  & 870                   & 15        & DSC                   &  \cite{Stoliarov:PDS2008}     \\ \hline
\end{tabular}
\end{center}
\label{FAA_Properties}
\end{table}



\begin{figure}[h!]
\begin{tabular*}{\textwidth}{l@{\extracolsep{\fill}}r}
\includegraphics[height=2.15in]{SCRIPT_FIGURES/FAA_Polymers/FAA_Polymers_HDPE} &
\includegraphics[height=2.15in]{SCRIPT_FIGURES/FAA_Polymers/FAA_Polymers_HIPS} \\
\includegraphics[height=2.15in]{SCRIPT_FIGURES/FAA_Polymers/FAA_Polymers_PMMA}&
\end{tabular*}
\caption[Results of FAA Polymers, non-charring, comparison]
{Comparison of predicted and measured mass loss rates for three non-charring polymers exposed to a heat flux of 52~kW/m$^2$ in a
nitrogen environment.}
\label{FAA_Polymers}
\end{figure}

\clearpage

\subsection{Complex Non-Charring Polymers: PP, PA66, POM, and PET}

The polymers described in this section exhibit slightly more complex behavior than those in the previous section because they exhibit foaming and bubbling as they degrade. Table~\ref{FAA_Properties2} lists the properties of each polymer.  In the model, the polymers melt to form a liquid with identical properties as the solid. The melting is characterized by a relatively fast reaction that occurs near the melting temperature with a heat of reaction equivalent to a heat of melting. The predicted mass loss rates are compared with measured values from the NIST Gasification Apparatus, a device that pyrolyzes the solid in a nitrogen environment to prevent combustion of fuel gases. The results are shown in Fig.~\ref{FAA_Polymers2}. The exposing heat flux is 50~kW/m$^2$. A thin sheet of aluminum foil and a 2.5~cm layer of Foamglas insulation underlies the sample. Its properties are given in Ref.~\cite{Stoliarov:FM2012}.


\begin{table}[h!]
\caption[FAA complex non-charring polymer properties]{Input parameters for FAA Polymers complex non-charring samples~\cite{Stoliarov:FM2012}. Courtesy S.~Stoliarov, G.~Linteris and R.E.~Lyon. See Sec.~\ref{glossary} for an explanation of terms.}
\begin{center}
\begin{tabular}{|l|c|c|c|c|c|c|l|}
\hline
Property                    & Units         & PP                    & PA66                  & POM                   & PET                   & Unc.      & Method    \\
                            &               &                       &                       &                       &                       & (\%)      &           \\ \hline \hline
Density                     & kg/m$^3$      & 910                   & 1150                  & 1425                  & 1380                  & 5         & Direct    \\ \hline
Conductivity                & W/m/K         & 0.24                  & 0.34                  & 0.28                  & 0.29                  & 15        & TLC       \\ \hline
Specific Heat               & kJ/kg/K       & 2.68                  & 2.54                  & 1.88                  & 2.01                  & 15        & DSC       \\ \hline
Emissivity                  &               & 0.96                  & 0.95                  & 0.95                  & 0.903                 & 20        & IS        \\ \hline
Absorption Coef.            & m$^{-1}$      & 966                   & 3920                  & 3550                  & 2937                  & 50        & FTIR      \\ \hline
Pre-Exp.~Factor             & s$^{-1}$      & $1.6 \times 10^{23}$  & $5.7 \times 10^{17}$  & $3.7 \times 10^{10}$  & $4.50 \times 10^{18}$ & 50        & TGA       \\ \hline
Activation Energy           & J/mol       & $3.52 \times 10^{5}$  & $2.74 \times 10^{5}$  & $1.57 \times 10^{5}$  & $2.81 \times 10^{5}$  & 3         & TGA       \\ \hline
Heat of Reaction            & kJ/kg         & 1310                  & 1390                  & 1570                  & 1800                  & 15        & DSC       \\ \hline
Heat of Melting             & kJ/kg         & 80                    & 55                    & 141                   & 37                    & 15        & DSC       \\ \hline
Melting Temperature         & K             & 158                   & 262                   & 165                   & 253                   & 15        & DSC       \\ \hline

\end{tabular}
\end{center}
\label{FAA_Properties2}
\end{table}



\begin{figure}[h!]
\begin{tabular*}{\textwidth}{l@{\extracolsep{\fill}}r}
\includegraphics[height=2.15in]{SCRIPT_FIGURES/FAA_Polymers/FAA_Polymers_PP} &
\includegraphics[height=2.15in]{SCRIPT_FIGURES/FAA_Polymers/FAA_Polymers_PA66} \\
\includegraphics[height=2.15in]{SCRIPT_FIGURES/FAA_Polymers/FAA_Polymers_POM}&
\includegraphics[height=2.15in]{SCRIPT_FIGURES/FAA_Polymers/FAA_Polymers_PET} \\
\end{tabular*}
\caption[Results of FAA Polymers, complex, non-charring, comparison]
{Comparison of predicted and measured mass loss rates for four complex non-charring polymers exposed to a heat flux of 50~kW/m$^2$ in a
nitrogen environment.}
\label{FAA_Polymers2}
\end{figure}

\clearpage


\subsection{Polycarbonate (PC)}

Table~\ref{Properties_PC} lists the measured properties of polycarbonate. These values have been input directly into FDS, and the predicted heat release rates are compared with measured values from the Cone Calorimeter. The results for samples of various thicknesses and imposed heat fluxes are shown in Fig.~\ref{HRR_PC}. A 1~cm layer of Kaowool insulation was placed under the sample. Its properties are given in Ref.~\cite{Stoliarov:CF2010}. It is assumed that the polymer undergoes a single step reaction that forms fuel gas and char.
\be
   \hbox{PC} \to 0.21 \, \hbox{Char} + 0.79 \, \hbox{Gas}
\ee

\begin{table}[h!]
\caption[Properties of polycarbonate (PC)]{Properties of polycarbonate (PC). Courtesy S.~Stoliarov, University of Maryland. See Sec.~\ref{glossary} for an explanation of terms.}
\begin{center}
\begin{tabular}{|l|c|c|l|l|}
\hline
Property                    & Units         & Value                             & Method                &  Reference                                \\ \hline \hline
Polymer Density             & kg/m$^3$      & 1180 $\pm$ 60                     & Direct                &  \cite{Stoliarov:CF2010}                  \\ \hline
Polymer Conductivity        & W/m/K         & 0.22 $\pm$ 0.03                   & Literature            &  \cite{Stoliarov:CF2010}                  \\ \hline
Polymer Specific Heat       & kJ/kg/K       & 1.9 $\pm$ 0.3                     & DSC                   &  \cite{Stoliarov:PDS2008}                 \\ \hline
Polymer Emissivity          &               & 0.90 $\pm$ 0.05                   & IS                    &  \cite{Hallman:PES1974}                   \\ \hline
Polymer Absorption Coef.    & m$^{-1}$      & 1770 $\pm$ 590                    & FTIR                  &  \cite{Tsilingiris:ECM2003}               \\ \hline
Char Density                & kg/m$^3$      & 248                               & Cone Calorimeter      &  \cite{Stoliarov:CF2010}                  \\ \hline
Char Conductivity           & W/m/K         & 0.37                              & Cone Calorimeter      &  \cite{Stoliarov:CF2010}                  \\ \hline
Char Specific Heat          & kJ/kg/K       & 1.72 $\pm$ 0.17                   & Pulsed Current        &  \cite{Stoliarov:CF2010,Matsumoto:1996}   \\ \hline
Char Emissivity             &               & 0.85 $\pm$ 0.05                   & Pulsed Current        &  \cite{Stoliarov:CF2010,Matsumoto:1996}   \\ \hline
Char Absorption Coef.       & m$^{-1}$      & Opaque                            & Assumption            &  \cite{Stoliarov:CF2010}                  \\ \hline
Pre-Exp.~Factor             & s$^{-1}$      & $(1.9 \pm 1.1) \times 10^{18}$    & TGA                   &  \cite{Stoliarov:CF2010}                  \\ \hline
Activation Energy           & J/mol       & $(2.95 \pm 0.06) \times 10^{5}$   & TGA                   &  \cite{Stoliarov:CF2010}                  \\ \hline
Heat of Reaction            & kJ/kg         & 830 $\pm$ 140                     & DSC                   &  \cite{Stoliarov:PDS2008}                 \\ \hline
Heat of Combustion          & kJ/kg         & 25600 $\pm$ 130                   & MCC                   &  \cite{Stoliarov:CF2010}                  \\ \hline
Combustion Efficiency       &               & 0.84 $\pm$ 0.03                   & Cone Calorimeter      &  \cite{Stoliarov:CF2010}                  \\ \hline
\end{tabular}
\end{center}
\label{Properties_PC}
\end{table}

\begin{figure}[p]
\begin{tabular*}{\textwidth}{l@{\extracolsep{\fill}}r}
\includegraphics[height=2.15in]{SCRIPT_FIGURES/FAA_Polymers/FAA_Polymers_PC_6_75} &
\includegraphics[height=2.15in]{SCRIPT_FIGURES/FAA_Polymers/FAA_Polymers_PC_6_92} \\
\includegraphics[height=2.15in]{SCRIPT_FIGURES/FAA_Polymers/FAA_Polymers_PC_6_50} &
\includegraphics[height=2.15in]{SCRIPT_FIGURES/FAA_Polymers/FAA_Polymers_PC_3_75} \\
\includegraphics[height=2.15in]{SCRIPT_FIGURES/FAA_Polymers/FAA_Polymers_PC_9_75} &
\end{tabular*}
\caption[Heat release rate of polycarbonate (PC)]
{Comparison of predicted and measured heat release rates for polycarbonate (PC).}
\label{HRR_PC}
\end{figure}

\clearpage


\subsection{Poly(vinyl chloride) (PVC)}

Table~\ref{Properties_PVC} lists the measured properties of poly(vinyl chloride). These values have been input directly into FDS, and the predicted heat release rates are compared with measured values from the Cone Calorimeter. The results for samples of various thicknesses and imposed heat fluxes are shown in Fig.~\ref{HRR_PVC}. A 1~cm layer of Kaowool insulation was placed under the sample. Its properties are given in Ref.~\cite{Stoliarov:CF2010}.

It is assumed that the polymer decomposes via a two-step reaction:
\begin{eqnarray}
   \hbox{Polymer} &\to& 0.44 \, \hbox{Char 1} + 0.56 \, \hbox{Gas 1}  \\
   \hbox{Char 1}  &\to& 0.47 \, \hbox{Char 2} + 0.53 \, \hbox{Gas 2}
\end{eqnarray}


\begin{table}[h!]
\caption[Properties of poly(vinyl chloride) (PVC)]{Properties of poly(vinyl chloride) (PVC). Courtesy S.~Stoliarov, University of Maryland. See Sec.~\ref{glossary} for an explanation of terms.}
\begin{center}
\begin{tabular}{|l|c|c|l|l|}
\hline
Property                    & Units         & Value                             & Method                    &  Reference                                \\ \hline \hline
Polymer Density             & kg/m$^3$      & 1430 $\pm$ 70                     & Direct                    &  \cite{Stoliarov:CF2010}                  \\ \hline
Polymer Conductivity        & W/m/K         & 0.17 $\pm$ 0.01                   & Literature                &  \cite{Stoliarov:CF2010}                  \\ \hline
Polymer Specific Heat       & kJ/kg/K       & 1.55 $\pm$ 0.25                   & DSC                       &  \cite{Stoliarov:PDS2008}                 \\ \hline
Polymer Emissivity          &               & 0.90 $\pm$ 0.05                   & IS                        &  \cite{Hallman:PES1974}                   \\ \hline
Polymer Absorption Coef.    & m$^{-1}$      & 2145 $\pm$ 715                    & FTIR                      &  \cite{Tsilingiris:ECM2003}               \\ \hline
Char 1 Density              & kg/m$^3$      & 629                               & Constant Volume           &  \cite{Stoliarov:CF2010}                  \\ \hline
Char 1 Conductivity         & W/m/K         & 0.17                              & Inherited                 &  \cite{Stoliarov:CF2010}                  \\ \hline
Char 1 Specific Heat        & kJ/kg/K       & 1.55 $\pm$ 0.25                   & Inherited                 &  \cite{Stoliarov:CF2010}                  \\ \hline
Char 1 Emissivity           &               & 0.90 $\pm$ 0.05                   & Inherited                 &  \cite{Stoliarov:CF2010}                  \\ \hline
Char 1 Absorption Coef.     & m$^{-1}$      & 2453                              & Inverse Analysis          &  \cite{Stoliarov:CF2010}                  \\ \hline
Char 2 Density              & kg/m$^3$      & 296                               & Constant Volume           &  \cite{Stoliarov:CF2010}                  \\ \hline
Char 2 Conductivity         & W/m/K         & 0.26                              & Inverse Analysis          &  \cite{Stoliarov:CF2010}                  \\ \hline
Char 2 Specific Heat        & kJ/kg/K       & 1.72 $\pm$ 0.17                   & Pulsed Current            &  \cite{Stoliarov:CF2010,Matsumoto:1996}   \\ \hline
Char 2 Emissivity           &               & 0.85 $\pm$ 0.05                   & Pulsed Current            &  \cite{Stoliarov:CF2010,Matsumoto:1996}   \\ \hline
Char 2 Absorption Coef.     & m$^{-1}$      & Opaque                            & Assumption                &  \cite{Stoliarov:CF2010}                  \\ \hline
Reac 1 Pre-Exp.~Factor      & s$^{-1}$      & $(1.4 \pm 0.8) \times 10^{33}$    & TGA                       &  \cite{Stoliarov:CF2010}                  \\ \hline
Reac 1 Activation Energy    & J/mol       & $(3.67 \pm 0.07) \times 10^{5}$   & TGA                       &  \cite{Stoliarov:CF2010}                  \\ \hline
Reac 1 Char Yield           &               & $0.44 \pm 0.01$                   & TGA                       &  \cite{Stoliarov:CF2010}                  \\ \hline
Reac 1 Heat of Reaction     & kJ/kg         & 170 $\pm$ 17                      & DSC                       &  \cite{Stoliarov:PDS2008}                 \\ \hline
Gas 1 Heat of Combustion    & kJ/kg         & 2700 $\pm$ 300                    & MCC                       &  \cite{Stoliarov:CF2010}                  \\ \hline
Gas 1 Combustion Efficiency &               & 0.75 $\pm$ 0.03                   & Cone Calorimeter          &  \cite{Stoliarov:CF2010}                  \\ \hline
Reac 2 Pre-Exp.~Factor      & s$^{-1}$      & $(3.5 \pm 2.1) \times 10^{12}$    & TGA                       &  \cite{Stoliarov:CF2010}                  \\ \hline
Reac 2 Activation Energy    & J/mol       & $(2.07 \pm 0.04) \times 10^{5}$   & TGA                       &  \cite{Stoliarov:CF2010}                  \\ \hline
Reac 2 Char Yield           &               & $0.47 \pm 0.01$                   & TGA                       &  \cite{Stoliarov:CF2010}                  \\ \hline
Reac 2 Heat of Reaction     & kJ/kg         & 1200 $\pm$ 900                    & DSC                       &  \cite{Stoliarov:PDS2008}                 \\ \hline
Gas 2 Heat of Combustion    & kJ/kg         & 36500 $\pm$ 1800                  & MCC                       &  \cite{Stoliarov:CF2010}                  \\ \hline
Gas 2 Combustion Efficiency &               & 0.75 $\pm$ 0.03                   & Cone Calorimeter          &  \cite{Stoliarov:CF2010}                  \\ \hline
\end{tabular}
\end{center}
\label{Properties_PVC}
\end{table}

\begin{figure}[p]
\begin{tabular*}{\textwidth}{l@{\extracolsep{\fill}}r}
\includegraphics[height=2.15in]{SCRIPT_FIGURES/FAA_Polymers/FAA_Polymers_PVC_6_75} &
\includegraphics[height=2.15in]{SCRIPT_FIGURES/FAA_Polymers/FAA_Polymers_PVC_6_92} \\
\includegraphics[height=2.15in]{SCRIPT_FIGURES/FAA_Polymers/FAA_Polymers_PVC_6_50} &
\includegraphics[height=2.15in]{SCRIPT_FIGURES/FAA_Polymers/FAA_Polymers_PVC_3_75} \\
\includegraphics[height=2.15in]{SCRIPT_FIGURES/FAA_Polymers/FAA_Polymers_PVC_9_75} &
\end{tabular*}
\caption[Heat release rate of poly(vinyl chloride) (PVC)]
{Comparison of predicted and measured heat release rates for poly(vinyl chloride) (PVC).}
\label{HRR_PVC}
\end{figure}

\clearpage



\subsection{Poly(aryl ether ether ketone)) (PEEK)}

Table~\ref{Properties_PEEK} lists the measured properties of poly(aryl ether ether ketone)\footnote{Trade name VICTREX PEEK 450G. The sample has been thoroughly dried.}. Its property values have been input directly into FDS, and the predicted heat release rates are compared with measured values from the Cone Calorimeter. It is assumed that the polymer decomposes via a four-step reaction:
\begin{eqnarray}
   \hbox{Polymer} &\to& 0.62 \, \hbox{Char 1} + 0.38 \, \hbox{Gas 1}  \\
   \hbox{Char 1}  &\to& 0.88 \, \hbox{Char 2} + 0.12 \, \hbox{Gas 2}  \\
   \hbox{Char 2}  &\to& 0.88 \, \hbox{Char 3} + 0.12 \, \hbox{Gas 2}  \\
   \hbox{Char 3}  &\to& \hbox{Gas 2}
\end{eqnarray}
It is also assumed that the gaseous fuel molecule is C$_{19}$H$_{12}$O$_3$. A 1~cm layer of Kaowool insulation was placed under the sample. Its properties are given in Ref.~\cite{Stoliarov:CF2010}.

The results for 3.9~mm samples at imposed heat fluxes of 50~kW/m$^2$, 70~kW/m$^2$, and 90~kW/m$^2$ are shown in Fig.~\ref{HRR_PEEK}. Note that the plots on the left are the results of simulations of the solid phase only, where the heat feedback from the fire is assumed to be 15~kW/m$^2$ and it is applied at the time of ignition. The plots on the right are from 3-D simulations of the solid sample and the fire. In these cases, the radiative feedback is not specified but rather calculated.

\begin{table}[p]
\caption[Properties of poly(aryl ether ether ketone) (PEEK)]{Properties of poly(aryl ether ether ketone) (PEEK). Courtesy E.~Oztekin, U.S.~FAA and S.~Stoliarov,
University of Maryland. See Sec.~\ref{glossary} for an explanation of terms.}
\begin{center}
\begin{tabular}{|l|c|c|l|l|}
\hline
Property                    & Units         & Value                             & Method                    &  Reference                              \\ \hline \hline
Polymer Density             & kg/m$^3$      & 1300                              & Direct                    &  \cite{Oztekin:CF2012}                  \\ \hline
Polymer Conductivity        & W/m/K         & 0.28                              & Inverse Analysis          &  \cite{Oztekin:CF2012}                  \\ \hline
Polymer Specific Heat       & kJ/kg/K       & 2.05                              & Inverse Analysis          &  \cite{Oztekin:CF2012}                  \\ \hline
Polymer Emissivity          &               & 0.90                              & Inverse Analysis          &  \cite{Oztekin:CF2012}                  \\ \hline
Polymer Absorption Coef.    & m$^{-1}$      & 1690                              & Inverse Analysis          &  \cite{Oztekin:CF2012}                  \\ \hline
Char 1 Density              & kg/m$^3$      & 810                               & Constant Volume           &  \cite{Oztekin:CF2012}                  \\ \hline
Char 1 Conductivity         & W/m/K         & 0.37                              & Inverse Analysis          &  \cite{Oztekin:CF2012}                  \\ \hline
Char 1 Specific Heat        & kJ/kg/K       & 0.24                              & Assumed                   &  \cite{Oztekin:CF2012}                  \\ \hline
Char 1 Emissivity           &               & 1                                 & Assumed                   &  \cite{Oztekin:CF2012}                  \\ \hline
Char 1 Absorption Coef.     & m$^{-1}$      & 81000                             & Assumed opaque            &  \cite{Oztekin:CF2012}                  \\ \hline
Char 2 Density              & kg/m$^3$      & 710                               & Constant Volume           &  \cite{Oztekin:CF2012}                  \\ \hline
Char 2 Conductivity         & W/m/K         & 0.37                              & Inverse Analysis          &  \cite{Oztekin:CF2012}                  \\ \hline
Char 2 Specific Heat        & kJ/kg/K       & 0.27                              & Assumed                   &  \cite{Oztekin:CF2012}                  \\ \hline
Char 2 Emissivity           &               & 1                                 & Assumed                   &  \cite{Oztekin:CF2012}                  \\ \hline
Char 2 Absorption Coef.     & m$^{-1}$      & 71000                             & Assumed opaque            &  \cite{Oztekin:CF2012}                  \\ \hline
Reac 1 Pre-Exp.~Factor      & s$^{-1}$      & $1.0 \times 10^{32}$              & TGA                       &  \cite{Oztekin:CF2012}                  \\ \hline
Reac 1 Activation Energy    & J/mol       & $5.57 \times 10^5$                & TGA                       &  \cite{Oztekin:CF2012}                  \\ \hline
Reac 1 Char Yield           &               & 0.62                              & TGA                       &  \cite{Oztekin:CF2012}                  \\ \hline
Reac 1 Heat of Reaction     & kJ/kg         & 350                               & Inverse Analysis          &  \cite{Oztekin:CF2012}                  \\ \hline
Gas 1 Heat of Combustion    & kJ/kg         & 16000                             & Cone calorimetry          &  \cite{Oztekin:CF2012}                  \\ \hline
Gas 1 Combustion Efficiency &               & 1                                 & Assumed                   &  \cite{Oztekin:CF2012}                  \\ \hline
Reac 2 Pre-Exp.~Factor      & s$^{-1}$      & $1.0 \times 10^3$                 & TGA                       &  \cite{Oztekin:CF2012}                  \\ \hline
Reac 2 Activation Energy    & J/mol       & $8.9 \times 10^4$                 & TGA                       &  \cite{Oztekin:CF2012}                  \\ \hline
Reac 2 Char Yield           &               & 0.88                              & TGA                       &  \cite{Oztekin:CF2012}                  \\ \hline
Reac 2 Heat of Reaction     & kJ/kg         & 0                                 & Assumed                   &  \cite{Oztekin:CF2012}                  \\ \hline
Gas 2 Heat of Combustion    & kJ/kg         & 27000                             & Cone Calorimetry          &  \cite{Oztekin:CF2012}                  \\ \hline
Gas 2 Combustion Efficiency &               & 1                                 & Assumed                   &  \cite{Oztekin:CF2012}                  \\ \hline
Reac 3 Pre-Exp.~Factor      & s$^{-1}$      & $1.0 \times 10^5$                 & TGA                       &  \cite{Oztekin:CF2012}                  \\ \hline
Reac 3 Activation Energy    & J/mol       & $1.47 \times 10^5$                & TGA                       &  \cite{Oztekin:CF2012}                  \\ \hline
Reac 3 Char Yield           &               & 0.88                              & TGA                       &  \cite{Oztekin:CF2012}                  \\ \hline
Reac 3 Heat of Reaction     & kJ/kg         & 0                                 & Assumed                   &  \cite{Oztekin:CF2012}                  \\ \hline
Reac 4 Pre-Exp.~Factor      & s$^{-1}$      & $1.0 \times 10^3$                 & TGA                       &  \cite{Oztekin:CF2012}                  \\ \hline
Reac 4 Activation Energy    & J/mol       & $1.29 \times 10^5$                & TGA                       &  \cite{Oztekin:CF2012}                  \\ \hline
Reac 4 Char Yield           &               & 0                                 & TGA                       &  \cite{Oztekin:CF2012}                  \\ \hline
Reac 4 Heat of Reaction     & kJ/kg         & 0                                 & Assumed                   &  \cite{Oztekin:CF2012}                  \\ \hline
\end{tabular}
\end{center}
\label{Properties_PEEK}
\end{table}

\begin{figure}[p]
\begin{tabular*}{\textwidth}{l@{\extracolsep{\fill}}r}
\includegraphics[height=2.15in]{SCRIPT_FIGURES/FAA_Polymers/FAA_Polymers_PEEK_50_solid_only} &
\includegraphics[height=2.15in]{SCRIPT_FIGURES/FAA_Polymers/FAA_Polymers_PEEK_50} \\
\includegraphics[height=2.15in]{SCRIPT_FIGURES/FAA_Polymers/FAA_Polymers_PEEK_70_solid_only} &
\includegraphics[height=2.15in]{SCRIPT_FIGURES/FAA_Polymers/FAA_Polymers_PEEK_70} \\
\includegraphics[height=2.15in]{SCRIPT_FIGURES/FAA_Polymers/FAA_Polymers_PEEK_90_solid_only} &
\includegraphics[height=2.15in]{SCRIPT_FIGURES/FAA_Polymers/FAA_Polymers_PEEK_90}
\end{tabular*}
\caption[Heat release rate of poly(aryl ether ether ketone) (PEEK)]
{Comparison of predicted and measured heat release rates for poly(aryl ether ether ketone) (PEEK). The plots on the left include only a simulation of the solid phase with an added heat flux of 15~kW/m$^2$ to account for the radiative feedback from the flame. The plots on the right are 3-D simulations of the solid sample and the fire.}
\label{HRR_PEEK}
\end{figure}

\clearpage


\subsection{Poly(butylene terephtalate) (PBT)}

Samples of poly(butylene terephtalate) (PBT)\footnote{Tradename Arnite T06-200, DSM Engineering Plastics} have been burned without oxygen in the Gasification Apparatus and with oxygen in the Cone Calorimeter. The properties of PBT are listed in Table~\ref{Properties_PBT}. It is assumed that the polymer undergoes a single step reaction that forms fuel gas and no char.

The results of the simulations are shown in Fig.~\ref{HRR_PBT}. Note that the effect of the flame radiation heat feedback to the sample surface is accounted for by increasing the imposed heat fluxes of 35~kW/m$^2$ by 39~\%, 50~kW/m$^2$ by 22~\%, and 70~kW/m$^2$ by 6~\%~\cite{Kempel:1}.


\begin{table}[h!]
\caption[Properties of poly(butylene terephtalate) (PBT)]{Properties of poly(butylene terephtalate) (PBT). Courtesy S.~Stoliarov, University of Maryland, and
Florian Kempel.
See Sec.~\ref{glossary} for an explanation of terms. Note that the Specific Heat and Conductivity result from averaging the reported temperature dependent
properties over the room to decomposition temperature range (300~K -- 650~K).
The heat capacity value is increased by 0.13~kJ/kg/K to account for the heat of melting (-46~kJ/kg), which takes place at 493~K.}
\begin{center}
\begin{tabular}{|l|c|c|l|l|}
\hline
Property                & Units     & Value                             & Method                                & Reference                     \\ \hline \hline
Density                 & kg/m$^3$  & 1300 $\pm$ 70                     & Direct                                & \cite{Kempel:1}               \\ \hline
Specific Heat           & kJ/kg/K   & 2.23 $\pm$ 0.34                   & DSC                                   & \cite{Kempel:1}               \\ \hline
Conductivity            & W/m/K     & 0.29 $\pm$ 0.05                   & TLS                                   & \cite{Kempel:1}               \\ \hline
Emissivity              &           & 0.88 $\pm$ 0.05                   & FTIR                                  & \cite{Linteris:2}             \\ \hline
Absorption Coefficient  & m$^{-1}$  & 2561 $\pm$ 140                    & FTIR                                  & \cite{Linteris:2}             \\ \hline
Pre-Exp. Factor         & s$^{-1}$  & $(2.49 \pm 0.62) \times 10^{14}$  & TGA                                   & \cite{Kempel:1}               \\ \hline
Activation Energy       & J/mol   & $(2.12 \pm 0.53) \times 10^{5}$   & TGA                                   & \cite{Kempel:1}               \\ \hline
Heat of Reaction        & kJ/kg     & 507                               & DSC, Literature                       & \cite{Kempel:1,Lyon:Ency2005} \\ \hline
Heat of Combustion      & kJ/kg     & 19500                             & Cone Calorimeter                      & \cite{Kempel:1}               \\ \hline
Combustion Efficiency   &           & 1                                 & Assumption                            & \cite{Kempel:1}               \\ \hline
\end{tabular}
\end{center}
\label{Properties_PBT}
\end{table}


\begin{figure}[h!]
\begin{tabular*}{\textwidth}{l@{\extracolsep{\fill}}r}
 &
\includegraphics[height=2.15in]{SCRIPT_FIGURES/FAA_Polymers/FAA_Polymers_PBT_35_solid_only} \\
\includegraphics[height=2.15in]{SCRIPT_FIGURES/FAA_Polymers/FAA_Polymers_PBT} &
\includegraphics[height=2.15in]{SCRIPT_FIGURES/FAA_Polymers/FAA_Polymers_PBT_50_solid_only} \\
 &
\includegraphics[height=2.15in]{SCRIPT_FIGURES/FAA_Polymers/FAA_Polymers_PBT_70_solid_only}
\end{tabular*}
\caption[Mass loss rate of poly(butylene terephtalate) (PBT)]
{Comparison of predicted and measured mass loss rates for poly(butylene terephtalate) (PBT)
in both the Gasification Apparatus and Cone Calorimeter.}
\label{HRR_PBT}
\end{figure}


\clearpage


\subsection{PBT with Glass Fibers (PBT-GF)}

Samples of poly(butylene terephtalate) (PBT), blended with 30~\% by mass glass fibers\footnote{Tradename Arnite TV4-261, DSM Engineering Plastics}, have been burned without oxygen in the Gasification Apparatus and with oxygen in the Cone Calorimeter. The properties of PBT-GF are listed in Table~\ref{Properties_PBT-GF}. It is assumed that the polymer undergoes a single step reaction that forms fuel gas and char.
\be
   \hbox{PBT-GF} \to 0.32 \, \hbox{Char} + 0.68 \, \hbox{Gas}
\ee
The results of the simulations are shown in Fig.~\ref{HRR_PBTGF}. Note that the effect of the flame radiation heat feedback to the sample surface is accounted for by increasing the imposed heat fluxes of 35~kW/m$^2$ by 33~\%, 50~kW/m$^2$ by 16~\%, and 70~kW/m$^2$ by 5~\%~\cite{Kempel:1}.

\begin{table}[h!]
\caption[Properties of poly(butylene terephtalate) with glass fibers (PBT-GF)]{Properties of poly(butylene terephtalate) with glass fibers (PBT-GF).
Courtesy S.~Stoliarov, University of Maryland. See Sec.~\ref{glossary} for an explanation of terms. Note that the Polymer Specific Heat and Polymer Conductivity
are the result of averaging the reported temperature dependent properties over the room to decomposition temperature range (300~K -- 650~K).
The heat capacity value is increased by 0.09~kJ/kg/K to account for the heat of melting (-32~kJ/kg), which takes place at 493~K.}
\begin{center}
\begin{tabular}{|l|c|c|l|l|}
\hline
Property                &      Units    &      Value                        & Method                                    & Reference                     \\ \hline \hline
Polymer Density         &     kg/m$^3$  & 1520 $\pm$ 80                     & Direct                                    & \cite{Kempel:1}               \\ \hline
Polymer Specific Heat   &    kJ/kg/K    & 1.68 $\pm$ 0.26                   & DSC                                       & \cite{Kempel:1}               \\ \hline
Polymer Conductivity    &      W/m/K    & 0.36 $\pm$ 0.06                   & TLS                                       & \cite{Kempel:1}               \\ \hline
Polymer Emissivity      &               & 0.87 $\pm$ 0.05                   & FTIR                                      & \cite{Linteris:2}             \\ \hline
Polymer Absorption Coef.&  m$^{-1}$     & 2860 $\pm$ 150                    & FTIR                                      & \cite{Linteris:2}             \\ \hline
Char Density            &     kg/m$^3$  &        482                        & Constant Volume                           & \cite{Kempel:1}               \\ \hline
Char Specific Heat      &    kJ/kg/K    &        0.85                       & Literature                                & \cite{SCHOTT}                 \\ \hline
Char Conductivity       &      W/m/K    & 0.07 $\pm$ 0.02                   & Laser Flash                               & \cite{Kempel:1}               \\ \hline
Char Emissivity         &               &       0.85                        & Literature                                & \cite{Braeuer:1}              \\ \hline
Char Absorption Coef.   &      m$^{-1}$ &       10000                       & Estimated                                 & \cite{Kempel:1}               \\ \hline
Pre-Exp. Factor         &      s$^{-1}$ & $(2.49 \pm 0.63) \times 10^{14}$  & TGA                                       & \cite{Kempel:1}               \\ \hline
Activation Energy       &    J/mol    & $(2.12 \pm 0.53) \times 10^{5}$   & TGA                                       & \cite{Kempel:1}               \\ \hline
Heat of Reaction        &      kJ/kg    &        355                        & DSC, Literature                           & \cite{Kempel:1,Lyon:Ency2005} \\ \hline
Heat of Combustion      &      kJ/kg    & 19500                             & Cone Calorimeter                          & \cite{Kempel:1}               \\ \hline
Char Yield              &               & 0.32 $\pm$ 0.05                   & Gasification Device                       & \cite{Kempel:1}               \\ \hline
Combustion Efficiency   &               &          1                        & Assumption                                & \cite{Kempel:1}               \\ \hline
\end{tabular}
\end{center}
\label{Properties_PBT-GF}
\end{table}

\begin{figure}[h!]
\begin{tabular*}{\textwidth}{l@{\extracolsep{\fill}}r}
 &
\includegraphics[height=2.15in]{SCRIPT_FIGURES/FAA_Polymers/FAA_Polymers_PBTGF_35_solid_only} \\
\includegraphics[height=2.15in]{SCRIPT_FIGURES/FAA_Polymers/FAA_Polymers_PBTGF} &
\includegraphics[height=2.15in]{SCRIPT_FIGURES/FAA_Polymers/FAA_Polymers_PBTGF_50_solid_only} \\
 &
\includegraphics[height=2.15in]{SCRIPT_FIGURES/FAA_Polymers/FAA_Polymers_PBTGF_70_solid_only}
\end{tabular*}
\caption[Mass loss rate of poly(butylene terephtalate) with glass fibers (PBT-GF)]
{Comparison of predicted and measured mass loss rates for poly(butylene terephtalate) with glass fibers (PBT-GF) in both the Gasification Apparatus and Cone Calorimeter.}
\label{HRR_PBTGF}
\end{figure}

\clearpage

\section{NIST Polymers}
\label{NIST_Polymers_Properties}

Black PMMA was chosen as the first material to be studied by the MaCFP Condenced Phase Subgroup. Its properties are listed in Table~\ref{NIST_PMMA_Properties}. It is assumed that the solid decomposes in two steps~\cite{Fiola:FSJ2021}.
\begin{eqnarray}
   \hbox{PMMA}_{\rm melt}   &\to& 0.98 \, \hbox{PMMA}_{\rm int} + 0.02 \, \hbox{MMA}_{\rm gas}  \label{PMMA1of2} \\
   \hbox{PMMA}_{\rm int}    &\to& 0.002 \, \hbox{Char} +         0.998 \, \hbox{MMA}_{\rm gas}   \label{PMMA2of2}
\end{eqnarray}

\begin{table}[!h]
\caption[NIST PMMA properties]{Input parameters for PMMA~\cite{Fiola:FSJ2021}.}
\begin{center}
\begin{tabular}{|l|c|c|c|}
\hline
Property                    & Units         & Value                                                                                                                                        & Unc.           \\ \hline \hline
Density                     & kg/m$^3$      & 1210                                                                                                                                         & 30             \\ \hline
Conductivity                & W/m/K         & $\begin{array}{ll} 0.16  & T<395 \; \hbox{K} \\  0.34 - 4.2 \times 10^{-4} \, T  & T>395 \; \hbox{K} \end{array}$                            &                \\ \hline
Specific Heat               & kJ/kg/K       & $\begin{array}{ll} -1.39+ 8.33 \times 10^{-3} \, T & T<395 \; \hbox{K} \\ 0.851 + 3.07 \times 10^{-3} \, T & T>395 \; \hbox{K} \end{array}$  &                \\ \hline
Emissivity                  &               & 0.96                                                                                                                                         &                \\ \hline
Absorption Coef.            & m$^{-1}$      & 2870                                                                                                                                         & 280            \\ \hline
Pre-Exp.~Factor             & s$^{-1}$      & $\begin{array}{ll} 4.95 \times 10^{16} & \hbox{ Step 1} \\ 1.35 \times 10^{11} & \hbox{ Step 2} \end{array}$                                 &                \\ \hline
Activation Energy           & J/mol         & $\begin{array}{ll} 1.64 \times 10^{5}  & \hbox{ Step 1} \\ 1.64 \times 10^{5}  & \hbox{ Step 2} \end{array}$                                 &                \\ \hline
Heat of Reaction            & kJ/kg         & $\begin{array}{ll} 5                   & \hbox{ Step 1} \\ 817                 & \hbox{ Step 2} \end{array}$                                 &                \\ \hline
\end{tabular}
\end{center}
\label{NIST_PMMA_Properties}
\end{table}

Figure~\ref{NIST_PMMA} displays the results of measurements performed at NIST and FDS simulations of pyrolyzing black PMMA in the TGA (10~K/min), MCC (60~K/min) and DSC (10~K/min). Figure~\ref{NIST_PMMA_Cone} displays results of the Gasification Apparatus (50~kW/m$^2$) and Cone Calorimeter (25~kW/m$^2$).

\begin{figure}[!h]
\begin{tabular*}{\textwidth}{l@{\extracolsep{\fill}}r}
\includegraphics[height=2.15in]{SCRIPT_FIGURES/NIST_Polymers/PMMA_TGA_10K} &
\includegraphics[height=2.15in]{SCRIPT_FIGURES/NIST_Polymers/PMMA_MCC_60K} \\
\multicolumn{2}{c}{\includegraphics[height=2.15in]{SCRIPT_FIGURES/NIST_Polymers/PMMA_DSC_10K}}
\end{tabular*}
\caption[NIST Polymers, TGA, MCC and DSC analysis of black PMMA]
{(Top left) Mass of a 5~mg sample of black PMMA undergoing Thermo-Gravimetric Analysis in nitrogen with a heating rate of 10~K/min. (Top right) Heat release rate of a small sample of PMMA in the Micro-Combustion Calorimeter with a heating rate of 60~K/min. (Bottom) Heat flow into a small sample of PMMA in the Differential Scanning Calorimeter at 10~K/min.}
\label{NIST_PMMA}
\end{figure}

\begin{figure}[!h]
\begin{tabular*}{\textwidth}{l@{\extracolsep{\fill}}r}
\includegraphics[height=2.15in]{SCRIPT_FIGURES/NIST_Polymers/PMMA_Gasification_50kW} &
\includegraphics[height=2.15in]{SCRIPT_FIGURES/NIST_Polymers/PMMA_Cone_25kW}
\end{tabular*}
\caption[NIST Polymers, PMMA in the Cone and Gasification Apparatus]
{(Left) Mass loss rate of a 6~mm thick sample of PMMA exposed to 50~kW/m$^2$ in the Gasification Apparatus. (Right) Heat release rate of a 6~mm thick sample of PMMA exposed to 25~kW/m$^2$ in the Cone Calorimeter.}
\label{NIST_PMMA_Cone}
\end{figure}


\clearpage

\section{UMD Polymers}

This section contains a description of seven polymers analyzed by J.~Li for his doctoral thesis at the University of Maryland~\cite{Li:Thesis}. In addition to the thesis itself, details of the measurement techniques can be found in Refs.~\cite{Li:IJHMT,Li:CF,Linteris:2,Li:PDS_2014,Li:PDS_2015}.

In the experiments, samples of seven different polymers were exposed to several different heat flux levels in the controlled atmosphere pyrolysis apparatus (CAPA) developed at the University of Maryland. This apparatus is similar to a cone calorimeter, but with a nitrogen environment. Thus, it is similar in function to the Gasification Apparatus. In each experiment, a roughly 6~mm sample was placed upon a wire mesh with no insulated backing. The top side of the sample was exposed to a specified heat flux, while the bottom remained exposed to ambient conditions. The mass loss rate of the sample was measured, and in the sections to follow the measured values are compared to FDS predictions. The seven polymers are organized into groups with one, two, or three degradation steps.

\subsection{One-Step Degradation: ABS, HIPS, and PMMA}

These three polymers are assumed to pyrolyze according to the following single step process:
\begin{equation}
   \hbox{Polymer}  \to \nu_{\rm r} \, \hbox{Char} + (1-\nu_{\rm r}) \, \hbox{Gas}
\end{equation}
The properties of the virgin polymer, the char, and the reaction kinetics are listed in Table~\ref{Properties_ABS_HIPS_PMMA}.

\begin{table}[h!]
\caption[Properties of ABS, HIPS, and PMMA]{Properties of ABS, HIPS, and PMMA. Note that the temperature dependence of the thermal conductivity is assumed to be linear, unlike some of those reported in Ref.~\cite{Li:Thesis}.}
\centering
\begin{tabular*}{\textwidth}{|l|@{\extracolsep\fill}c@{\extracolsep\fill}|c|c|c|}
\hline
Property                    & Units         & ABS                     & HIPS                    & PMMA                     \\ \hline \hline
Polymer Density             & kg/m$^3$      & 1050                    & 1060                    & 1160                     \\ \hline
                            &               &                         &                         & $0.45-0.00038 \, T, \; T<378$~K      \\
\raisebox{1.5ex}[0pt]{Polymer Cond.}        & \raisebox{1.5ex}[0pt]{W/m/K}         & \raisebox{1.5ex}[0pt]{$0.30-0.00028 \, T$}  & \raisebox{1.5ex}[0pt]{$0.10+0.0001  \, T$} & $0.27-0.00024 \, T, \; T\ge 378$~K  \\ \hline
Polymer Spec.~Heat          & kJ/kg/K       & $1.58+0.0013  \, T$     & $0.59+0.0034  \, T$     & $0.60+0.0036  \, T$      \\ \hline
Polymer Emissivity          &               & 0.95                    & 0.95                    & 0.95                     \\ \hline
Polymer Abs.~Coef.          & m$^{-1}$      & 1800                    & 2250                    & 2240                     \\ \hline
Char Density                & kg/m$^3$      & 80                      & Same as Polymer         & Same as Polymer          \\ \hline
Char Conductivity           & W/m/K         & $0.13-0.00054 \, T$     & Same as Polymer         & Same as Polymer          \\ \hline
Char Specific Heat          & kJ/kg/K       & $0.82+0.00011 \, T$     & Same as Polymer         & Same as Polymer          \\ \hline
Char Emissivity             &               & 0.86                    & Same as Polymer         & Same as Polymer          \\ \hline
Char Abs.~Coef.             & m$^{-1}$      & 2500                    & Opaque                  & Same as Polymer          \\ \hline
Pre-Exp.~Factor             & s$^{-1}$      & $1.00\times 10^{14}$    & $1.70\times 10^{20}$    & $8.60\times 10^{12}$     \\ \hline
Activation Energy           & J/mol       & $2.19\times 10^5$       & $3.01\times 10^5$       & $1.88\times 10^5$        \\ \hline
Heat of Reaction            & kJ/kg         & 460                     & 689                     & 846                      \\ \hline
Heat of Combustion          & kJ/kg         & 28750                   & 29900                   & 24450                    \\ \hline
Residue Fraction            &               & 0.023                   & 0.043                   & 0.015                    \\ \hline
\end{tabular*}
\label{Properties_ABS_HIPS_PMMA}
\end{table}


\begin{figure}[p]
\begin{tabular*}{\textwidth}{l@{\extracolsep{\fill}}r}
\includegraphics[height=2.1in]{SCRIPT_FIGURES/UMD_Polymers/ABS_30} &
\includegraphics[height=2.1in]{SCRIPT_FIGURES/UMD_Polymers/ABS_50} \\
\multicolumn{2}{c}{\includegraphics[height=2.1in]{SCRIPT_FIGURES/UMD_Polymers/ABS_70}} \\
\includegraphics[height=2.1in]{SCRIPT_FIGURES/UMD_Polymers/HIPS_30} &
\includegraphics[height=2.1in]{SCRIPT_FIGURES/UMD_Polymers/HIPS_50} \\
\multicolumn{2}{c}{\includegraphics[height=2.1in]{SCRIPT_FIGURES/UMD_Polymers/HIPS_70}}
\end{tabular*}
\caption[Mass loss rate of ABS and HIPS]
{Comparison of predicted and measured mass loss rates for ABS and HIPS.}
\label{ABS_HIPS}
\end{figure}

\begin{figure}[p]
\begin{tabular*}{\textwidth}{l@{\extracolsep{\fill}}r}
\includegraphics[height=2.15in]{SCRIPT_FIGURES/UMD_Polymers/PMMA_20} &
\includegraphics[height=2.15in]{SCRIPT_FIGURES/UMD_Polymers/PMMA_40} \\
\multicolumn{2}{c}{\includegraphics[height=2.15in]{SCRIPT_FIGURES/UMD_Polymers/PMMA_60}}
\end{tabular*}
\caption[Mass loss rate of PMMA]
{Comparison of predicted and measured mass loss rates for PMMA.}
\label{PMMA}
\end{figure}

\clearpage

\subsection{Two-Step Degradation: Kydex}

This polymer is assumed to pyrolyze according to the following two step process:
\begin{eqnarray}
   \hbox{Polymer}       &\to& 0.45 \, \hbox{Intermediate} + 0.55 \, \hbox{Gas}  \label{reac1of2} \\
   \hbox{Intermediate}  &\to& 0.31 \, \hbox{Char} +         0.69 \, \hbox{Gas}  \label{reac2of2}
\end{eqnarray}
The properties of the polymer and the reaction kinetics are listed in Table~\ref{Properties_Kydex} and the mass loss rate comparisons are shown on the following page. Note that nominal exposing heat flux values of 30~kW/m$^2$, 50~kW/m$^2$, and 70~kW/m$^2$ were changed slightly in the simulations to account for the fact that the intumescing material surface moved closer to the heater during the course of the experiment~\cite{Li:PDS_2015}.

\begin{table}[!h]
\caption[Properties of Kydex]{Properties of Kydex. Note that the temperature dependence of the thermal conductivity is assumed to be linear, unlike some of those reported in Ref.~\cite{Li:Thesis}.}
\centering
\begin{tabular}{|l|c|c|}
\hline
Property                                 & Units         & Kydex                    \\ \hline \hline
Polymer Density                          & kg/m$^3$      & 1350                     \\ \hline
Polymer Cond.                            & W/m/K         & $0.28-0.00029 \, T$      \\ \hline
Polymer Spec.~Heat                       & kJ/kg/K       & $-0.62+0.00593  \, T$    \\ \hline
Polymer, Int. Emissivity                 &               & 0.95                     \\ \hline
Polymer Abs.~Coef.                       & m$^{-1}$      & 2135                     \\ \hline
Int.~Density                             & kg/m$^3$      & Same as Char             \\ \hline
Int.~Cond.                               & W/m/K         & $0.55+0.00003 \, T$      \\ \hline
Int.~Spec.~Heat                          & kJ/kg/K       & $0.27+0.00301  \, T$     \\ \hline
Int.~Abs.~Coef.                          & m$^{-1}$      & 3000                     \\ \hline
Char Density                             & kg/m$^3$      & 100                      \\ \hline
Char Conductivity                        & W/m/K         & $0.21+0.00034 \, T$      \\ \hline
Char Specific Heat                       & kJ/kg/K       & $1.15+0.00010 \, T$      \\ \hline
Char Emissivity                          &               & 0.86                     \\ \hline
Char Abs.~Coef.                          & m$^{-1}$      & 10000                    \\ \hline
Reac.~\ref{reac1of2} Pre-Exp.~Factor     & s$^{-1}$      & $6.03\times 10^{10}$     \\ \hline
Reac.~\ref{reac1of2} Act.~Energy         & J/mol       & $1.41\times 10^5$        \\ \hline
Reac.~\ref{reac1of2} Heat of Reac.       & kJ/kg         & 180                      \\ \hline
Reac.~\ref{reac1of2} Residue Frac.       &               & 0.45                     \\ \hline
Reac.~\ref{reac2of2} Pre-Exp.~Factor     & s$^{-1}$      & $1.36\times 10^{10}$     \\ \hline
Reac.~\ref{reac2of2} Act.~Energy         & J/mol       & $1.74\times 10^5$        \\ \hline
Reac.~\ref{reac2of2} Heat of Reac.       & kJ/kg         & 125                      \\ \hline
Reac.~\ref{reac2of2} Residue Frac.       &               & 0.31                     \\ \hline
Gas Heat of Combustion                   & kJ/kg         & 12650                    \\ \hline
\end{tabular}
\label{Properties_Kydex}
\end{table}

\begin{figure}[p]
\begin{tabular*}{\textwidth}{l@{\extracolsep{\fill}}r}
\includegraphics[height=2.15in]{SCRIPT_FIGURES/UMD_Polymers/Kydex_30} &
\includegraphics[height=2.15in]{SCRIPT_FIGURES/UMD_Polymers/Kydex_50} \\
\multicolumn{2}{c}{\includegraphics[height=2.15in]{SCRIPT_FIGURES/UMD_Polymers/Kydex_70}}
\end{tabular*}
\caption[Mass loss rate of Kydex]
{Comparison of predicted and measured mass loss rates for Kydex.}
\label{Kydex}
\end{figure}

\clearpage

\subsection{Three-Step Degradation: PEI, PET, and POM}

These three polymers are assumed to pyrolyze following the three-step process:
\begin{eqnarray}
   \hbox{Polymer}       &\to& \hbox{Melt}  \label{reac1of3} \\
   \hbox{Melt}          &\to& \nu_{\rm r,2} \, \hbox{Intermediate} + (1-\nu_{\rm r,2}) \, \hbox{Gas}  \label{reac2of3} \\
   \hbox{Intermediate}  &\to& \nu_{\rm r,3} \, \hbox{Char} +         (1-\nu_{\rm r,3}) \, \hbox{Gas}  \label{reac3of3}
\end{eqnarray}
The property data is listed in Table~\ref{Properties_PEI_PET_POM} and the mass loss rate comparisons are shown on the subsequent pages.

\begin{table}[!h]
\caption[Properties of PEI, PET, and POM]{Properties of PEI, PET, and POM. Note that the temperature dependence of the thermal conductivity is assumed to be linear, unlike some of those reported in Ref.~\cite{Li:Thesis}.}
\centering
\begin{tabular}{|l|c|c|c|c|}
\hline
Property                                 & Units         & PEI                     & PET                     & POM                      \\ \hline \hline
Polymer, Melt Density                    & kg/m$^3$      & 1285                    & 1385                    & 1424                     \\ \hline
Polymer Cond.                            & W/m/K         & $0.40-0.00040 \, T$     & $0.34-0.00046  \, T$    & $0.25+0.00002 \, T$      \\ \hline
Polymer Spec.~Heat                       & kJ/kg/K       & $-0.04+0.00410  \, T$   & $-0.27+0.00464  \, T$   & $-1.86+0.0099  \, T$     \\ \hline
Polymer, Melt, Int. Emiss.               &               & 0.95                    & 0.95                    & 0.95                     \\ \hline
Polymer Abs.~Coef.                       & m$^{-1}$      & 1745                    & 1940                    & 3050                     \\ \hline
Melt Cond.                               & W/m/K         & $0.32-0.00033 \, T$     & $0.33-0.00002 \, T$     & $0.21+0.00001 \, T$      \\ \hline
Melt Spec.~Heat                          & kJ/kg/K       & $1.88+0.00057  \, T$    & $2.05-0.00021 \, T$     & $1.65+0.00120  \, T$     \\ \hline
Melt Abs.~Coef.                          & m$^{-1}$      & 128500                  & Same as Polymer         & Same as Polymer          \\ \hline
Int.~Density                             & kg/m$^3$      & Same as Char            & 730                     & Same as Polymer          \\ \hline
Int.~Cond.                               & W/m/K         & $0.45+0.00019 \, T$     & $0.45+0.00020  \, T$    & $0.19-0.00006 \, T$      \\ \hline
Int.~Spec.~Heat                          & kJ/kg/K       & $1.59+0.00031  \, T$    & $1.44-0.00005  \, T$    & Same as Melt             \\ \hline
Int.~Abs.~Coef.                          & m$^{-1}$      & 8000                    & 1025                    & Same as Polymer          \\ \hline
Char Density                             & kg/m$^3$      & 80                      & 80                      & Same as Int.             \\ \hline
Char Conductivity                        & W/m/K         & $0.45+0.00013 \, T$     & $0.34+0.00046  \, T$    & Same as Polymer          \\ \hline
Char Specific Heat                       & kJ/kg/K       & $1.30+0.00004 \, T$     & $0.82+0.00011 \, T$     & Same as Int.             \\ \hline
Char Emissivity                          &               & 0.86                    & 0.86                    & Same as Polymer          \\ \hline
Char Abs.~Coef.                          & m$^{-1}$      & Same as Int.            & 8000                    & Same as Polymer          \\ \hline
Reac.~\ref{reac1of3} Pre-Exp.~Factor     & s$^{-1}$      & 1                       & $1.50\times 10^{36}$    & $2.69\times 10^{42}$     \\ \hline
Reac.~\ref{reac1of3} Act.~Energy         & J/mol       & 0                       & $3.80\times 10^5$       & $3.82\times 10^5$        \\ \hline
Reac.~\ref{reac1of3} Heat of Reac.       & kJ/kg         & 1                       & 30                      & 192                      \\ \hline
Reac.~\ref{reac1of3} Residue Frac.       &               & 1                       & 1                       & 1                        \\ \hline
Reac.~\ref{reac2of3} Pre-Exp.~Factor     & s$^{-1}$      & $7.66\times 10^{27}$    & $1.60\times 10^{15}$    & $3.84\times 10^{14}$     \\ \hline
Reac.~\ref{reac2of3} Act.~Energy         & J/mol       & $4.65\times 10^5$       & $2.35\times 10^5$       & $2.00\times 10^5$        \\ \hline
Reac.~\ref{reac2of3} Heat of Reac.       & kJ/kg         & -80                     & 220                     & 1192                     \\ \hline
Reac.~\ref{reac2of3} Residue Frac.       &               & 0.65                    & 0.18                    & 0.4                      \\ \hline
Reac.~\ref{reac3of3} Pre-Exp.~Factor     & s$^{-1}$      & $6.50\times 10^{2}$     & $3.53\times 10^{4}$     & $4.76\times 10^{44}$     \\ \hline
Reac.~\ref{reac3of3} Act.~Energy         & J/mol       & $0.88\times 10^5$       & $0.96\times 10^5$       & $5.90\times 10^5$        \\ \hline
Reac.~\ref{reac3of3} Heat of Reac.       & kJ/kg         & -5                      & 250                     & 1352                     \\ \hline
Reac.~\ref{reac3of3} Residue Frac.       &               & 0.77                    & 0.72                    & 0.018                    \\ \hline
Gas Heat of Combustion                   & kJ/kg         & 18050                   & 15950                   & 14350                    \\ \hline
\end{tabular}
\label{Properties_PEI_PET_POM}
\end{table}


\begin{figure}[p]
\begin{tabular*}{\textwidth}{l@{\extracolsep{\fill}}r}
\includegraphics[height=2.15in]{SCRIPT_FIGURES/UMD_Polymers/PEI_50} &
\includegraphics[height=2.15in]{SCRIPT_FIGURES/UMD_Polymers/PEI_70} \\
\multicolumn{2}{c}{\includegraphics[height=2.15in]{SCRIPT_FIGURES/UMD_Polymers/PEI_90}} \\
\includegraphics[height=2.15in]{SCRIPT_FIGURES/UMD_Polymers/PET_50} &
\includegraphics[height=2.15in]{SCRIPT_FIGURES/UMD_Polymers/PET_70}
\end{tabular*}
\caption[Mass loss rate of PEI and PET]
{Comparison of predicted and measured mass loss rates for PEI and PET.}
\label{PEI_PET}
\end{figure}

\begin{figure}[p]
\begin{tabular*}{\textwidth}{l@{\extracolsep{\fill}}r}
\includegraphics[height=2.15in]{SCRIPT_FIGURES/UMD_Polymers/POM_30} &
\includegraphics[height=2.15in]{SCRIPT_FIGURES/UMD_Polymers/POM_50} \\
\multicolumn{2}{c}{\includegraphics[height=2.15in]{SCRIPT_FIGURES/UMD_Polymers/POM_70}}
\end{tabular*}
\caption[Mass loss rate of POM]
{Comparison of predicted and measured mass loss rates for POM.}
\label{POM}
\end{figure}

\clearpage

\section{Corrugated Cardboard}
\label{Cardboard}

Table~\ref{Properties_Cardboard} lists the measured properties of a double-wall corrugated cardboard with the conventional U.S.~designation 69-23B-69-23C-69. Corrugated cardboard is characterized by alternating layers of homogeneous, planar liner boards and corrugated sections made up of periodic flutes. The numbers in the specification indicate the areal density in lb/(1000~ft$^2$) and the letters indicate the flute designation (B indicates a range of 45 to 52 flutes per foot and C indicates a range of 39 to 43 flutes per foot). It is assumed that each layer consists of the same lingo-cellulosic, charring material with the density defined as the mass of the solid material divided by the volume of the layer. This representation requires slightly different definitions for the properties of each unique layer -- liner board (LB), C-flute layer (CFL), and B-flute layer (BFL).

The reaction mechanism for the cardboard material includes one reaction to describe the release of residual moisture and three sequential reactions to describe the thermal degradation of the virgin material to a final residual char. Each of the initial solid components (LB, CFL, and BFL) undergoes the same four-step mechanism.
\begin{eqnarray}
   \hbox{Moisture}            &\to& \hbox{Water Vapor}  \\
   \hbox{Virgin Cardboard}    &\to& 0.90 \, \hbox{Intermediary Solid} + 0.10 \, \hbox{Fuel Gas 2}  \\
   \hbox{Intermediary Solid}  &\to& 0.37 \, \hbox{Char 1}             + 0.63 \, \hbox{Fuel Gas 3}  \\
   \hbox{Char 1}              &\to& 0.59 \, \hbox{Char 2}             + 0.41 \, \hbox{Fuel Gas 4}
\end{eqnarray}

\begin{longtable}{@{\extracolsep{\fill}}|l|c|c|l|l|}
\caption[Properties of corrugated cardboard]{Properties of corrugated cardboard. Courtesy M. McKinnon, University of Maryland. See Sec.~\ref{glossary} for an explanation of terms.}
\label{Properties_Cardboard} \\
\hline
\endfirsthead
\caption[]{Continued} \\
\hline
\endhead
Property                          & Units         & Value                                   & Method                                    & Reference                             \\ \hline \hline
Moisture Density                  & kg/m$^3$      & 1000                                    & Direct                                    & \cite{McKinnon:CF2013}                \\ \hline
Moisture Conductivity             & W/m/K         & 0.1                                       & Inherited                                & \cite{McKinnon:CF2013}                \\ \hline
Moisture Specific Heat             & kJ/kg/K         & 4.19                                   & Literature                               & \cite{Coblentz:1}                     \\ \hline
Moisture Emissivity                 &               & 0.7                                       & Inherited                                & \cite{McKinnon:CF2013}                \\ \hline
LB Density                         & kg/m$^3$      &   520                                       & Direct                                   & \cite{McKinnon:CF2013}                \\ \hline
LB Conductivity                     & W/m/K         & 0.1                                       & Inverse Analysis                          & \cite{McKinnon:CF2013}                \\ \hline
LB Specific Heat                 & kJ/kg/K         & 1.8                                       & DSC                                       & \cite{McKinnon:CF2013}                \\ \hline
LB Emissivity                     &               & 0.7                                       & Inverse Analysis                          & \cite{McKinnon:CF2013}                \\ \hline
LB Intermediary Density             & kg/m$^3$     & 468                                       & Constant Volume                           & \cite{McKinnon:CF2013}                \\ \hline
LB Intermediary Conductivity     & W/m/K         & $0.05 + 7.5\times 10^{-11}\times T^3$   & Inverse Analysis                          & \cite{McKinnon:CF2013}                \\ \hline
LB Intermediary Specific Heat     & kJ/kg/K         & 1.55                                   & DSC                                       & \cite{McKinnon:CF2013}                \\ \hline
LB Intermediary Emissivity         &               & 0.775                                   & Inverse Analysis                          & \cite{McKinnon:CF2013}                \\ \hline
LB Char 1 Density                 & kg/m$^3$     & 173                                       & Constant Volume                           & \cite{McKinnon:CF2013}                \\ \hline
LB Char 1 Conductivity             & W/m/K         & $1.5\times 10^{-10}\times T^3$          & Inverse Analysis                          & \cite{McKinnon:CF2013}                \\ \hline
LB Char 1 Specific Heat             & kJ/kg/K         & 1.3                                       & DSC                                       & \cite{McKinnon:CF2013}                \\ \hline
LB Char 1 Emissivity             &               & 0.85                                   & Literature                               & \cite{Matsumoto:IJT1995}              \\ \hline
LB Char 2 Density                 & kg/m$^3$     & 102                                       & Constant Volume                           & \cite{McKinnon:CF2013}                \\ \hline
LB Char 2 Conductivity             & W/m/K         & $1.5\times 10^{-10}\times T^3$          & Inverse Analysis                          & \cite{McKinnon:CF2013}                \\ \hline
LB Char 2 Specific Heat             & kJ/kg/K         & 1.3                                       & DSC                                       & \cite{McKinnon:CF2013}                \\ \hline
LB Char 2 Emissivity             &               & 0.85                                   & Literature                               & \cite{Matsumoto:IJT1995}              \\ \hline
CFL Density                         & kg/m$^3$     & 49                                       & Constant Volume                           & \cite{McKinnon:CF2013}                \\ \hline
CFL Conductivity                 & W/m/K         & 0.1                                     & Inverse Analysis                          & \cite{McKinnon:CF2013}                \\ \hline
CFL Specific Heat                 & kJ/kg/K         & 1.8                                       & DSC                                       & \cite{McKinnon:CF2013}                \\ \hline
CFL Emissivity                     &               & 0.7                                    & Inverse Analysis                          & \cite{McKinnon:CF2013}                \\ \hline
CFL Intermediary Density         & kg/m$^3$     & 44                                       & Constant Volume                           & \cite{McKinnon:CF2013}                \\ \hline
CFL Intermediary Conductivity     & W/m/K         & $0.05 + 7.5\times 10^{-10}\times T^3$   & Inverse Analysis                          & \cite{McKinnon:CF2013}                \\ \hline
CFL Intermediary Specific Heat     & kJ/kg/K         & 1.55                                   & DSC                                       & \cite{McKinnon:CF2013}                \\ \hline
CFL Intermediary Emissivity         &               & 0.775                                    & Inverse Analysis                          & \cite{McKinnon:CF2013}                \\ \hline
CFL Char 1 Density                 & kg/m$^3$     & 16                                       & Constant Volume                           & \cite{McKinnon:CF2013}                \\ \hline
CFL Char 1 Conductivity             & W/m/K         & $1.5\times 10^{-9}\times T^3$           & Inverse Analysis                          & \cite{McKinnon:CF2013}                \\ \hline
CFL Char 1 Specific Heat         & kJ/kg/K         & 1.3                                       & DSC                                       & \cite{McKinnon:CF2013}                \\ \hline
CFL Char 1 Emissivity             &               & 0.85                                   & Literature                               & \cite{Matsumoto:IJT1995}              \\ \hline
CFL Char 2 Density                 & kg/m$^3$     & 9.4                                       & Constant Volume                           & \cite{McKinnon:CF2013}                \\ \hline
CFL Char 2 Conductivity             & W/m/K         & $1.5\times 10^{-9}\times T^3$           & Inverse Analysis                          & \cite{McKinnon:CF2013}                \\ \hline
CFL Char 2 Specific Heat         & kJ/kg/K         & 1.3                                       & DSC                                       & \cite{McKinnon:CF2013}                \\ \hline
CFL Char 2 Emissivity             &               & 0.85                                   & Literature                               & \cite{Matsumoto:IJT1995}              \\ \hline
BFL Density                         & kg/m$^3$     & 74                                       & Constant Volume                           & \cite{McKinnon:CF2013}                \\ \hline
BFL Conductivity                 & W/m/K         & 0.1                                     & Inverse Analysis                          & \cite{McKinnon:CF2013}                \\ \hline
BFL Specific Heat                 & kJ/kg/K         & 1.8                                       & DSC                                       & \cite{McKinnon:CF2013}                \\ \hline
BFL Emissivity                     &               & 0.7                                    & Inverse Analysis                          & \cite{McKinnon:CF2013}                \\ \hline
BFL Intermediary Density         & kg/m$^3$     & 67                                       & Constant Volume                           & \cite{McKinnon:CF2013}                \\ \hline
BFL Intermediary Conductivity     & W/m/K         & $0.05 + 7.5\times 10^{-10}\times T^3$   & Inverse Analysis                          & \cite{McKinnon:CF2013}                \\ \hline
BFL Intermediary Specific Heat     & kJ/kg/K         & 1.55                                   & DSC                                       & \cite{McKinnon:CF2013}                \\ \hline
BFL Intermediary Emissivity       &               & 0.775                                    & Inverse Analysis                          & \cite{McKinnon:CF2013}                \\ \hline
BFL Char 1 Density                 & kg/m$^3$     & 25                                       & Constant Volume                           & \cite{McKinnon:CF2013}                \\ \hline
BFL Char 1 Conductivity             & W/m/K         & $1.5\times 10^{-9}\times T^3$           & Inverse Analysis                          & \cite{McKinnon:CF2013}                \\ \hline
BFL Char 1 Specific Heat         & kJ/kg/K         & 1.3                                       & DSC                                       & \cite{McKinnon:CF2013}                \\ \hline
BFL Char 1 Emissivity             &               & 0.85                                   & Literature                               & \cite{Matsumoto:IJT1995}              \\ \hline
BFL Char 2 Density                 & kg/m$^3$     & 15                                       & Constant Volume                           & \cite{McKinnon:CF2013}                \\ \hline
BFL Char 2 Conductivity             & W/m/K         & $1.5\times 10^{-9}\times T^3$           & Inverse Analysis                          & \cite{McKinnon:CF2013}                \\ \hline
BFL Char 2 Specific Heat         & kJ/kg/K         & 1.3                                       & DSC                                       & \cite{McKinnon:CF2013}                \\ \hline
BFL Char 2 Emissivity             &               & 0.85                                   & Literature                               & \cite{Matsumoto:IJT1995}              \\ \hline
Reaction 1 Pre-Exp.~Factor        & s$^{-1}$     & 6.14                                   & TGA                                       & \cite{McKinnon:CF2013}                \\ \hline
Reaction 1 Activation Energy     & J/mol         & 23500                                   & TGA                                       & \cite{McKinnon:CF2013}                \\ \hline
Reaction 1 Heat of Reaction       &   kJ/kg         & 2445                                   & Literature                                & \cite{Coblentz:1}                     \\ \hline
Reaction 1 Char Yield             &               & 0                                       & TGA                                       & \cite{McKinnon:CF2013}                \\ \hline
Reaction 2 Pre-Exp.~Factor        & s$^{-1}$      & $7.95\times 10^9$                       & TGA                                       & \cite{McKinnon:CF2013}                \\ \hline
Reaction 2 Activation Energy     & J/mol         & $1.30\times 10^5$                       & TGA                                       & \cite{McKinnon:CF2013}                \\ \hline
Reaction 2 Char Yield             &               & 0.9                                       & TGA                                       & \cite{McKinnon:CF2013}                \\ \hline
Reaction 2 Heat of Reaction       & kJ/kg         & 0                                       & DSC                                       & \cite{McKinnon:CF2013}                \\ \hline
Fuel Gas 2 Heat of Combustion     & kJ/kg         & 18500                                   & MCC                                       & \cite{McKinnon:CF2013}                \\ \hline
Reaction 3 Pre-Exp.~Factor        & s$^{-1}$     & $2\times 10^{11}$                      & TGA                                       & \cite{McKinnon:CF2013}                \\ \hline
Reaction 3 Activation Energy     & J/mol         & $1.60\times 10^5$                       & TGA                                       & \cite{McKinnon:CF2013}                \\ \hline
Reaction 3 Char Yield             &               & 0.37                                   & TGA                                       & \cite{McKinnon:CF2013}                \\ \hline
Reaction 3 Heat of Reaction         & kJ/kg         & 126                                    & DSC                                       & \cite{McKinnon:CF2013}                \\ \hline
Fuel Gas 3 Heat of Combustion     & kJ/kg         & 13600                                   & MCC                                       & \cite{McKinnon:CF2013}                \\ \hline
Reaction 4 Pre-Exp.~Factor        & s$^{-1}$     & 0.0261                                   & TGA                                       & \cite{McKinnon:CF2013}                \\ \hline
Reaction 4 Activation Energy     & J/mol         & 17000                                   & TGA                                       & \cite{McKinnon:CF2013}                \\ \hline
Reaction 4 Char Yield             &               & 0.59                                   & TGA                                       & \cite{McKinnon:CF2013}                \\ \hline
Reaction 4 Heat of Reaction         & kJ/kg         & 0                                       & DSC                                       & \cite{McKinnon:CF2013}                \\ \hline
Fuel Gas 4 Heat of Combustion     & kJ/kg         & 14000                                  & MCC                                       & \cite{McKinnon:CF2013}                \\ \hline
\end{longtable}

\noindent Table~\ref{Dimensions_Cardboard} lists the composition and thickness of each of the layers. The sample is insulated with 28~mm of Kaowool PM board, manufactured by ThermalCeramics (www.thermalceramics.com).
\begin{table}[h!]
\caption[Cardboard composition and dimensions]{Cardboard composition and dimensions.}
\begin{center}
\begin{tabular}{|c|l|c|l|l|}
\hline
Layer     & Composition   & Thickness (mm) \\ \hline \hline
1         & Liner Board     & 0.64       \\ \hline
2         & C Flute Layer & 3.2        \\ \hline
3         & Liner Board     & 0.64       \\ \hline
4         & B Flute Layer & 2.1        \\ \hline
5         & Liner Board     & 0.64       \\ \hline
6         & Kaowool         & 28         \\ \hline
\end{tabular}
\end{center}
\label{Dimensions_Cardboard}
\end{table}
The gasification experiments were conducted in a modified cone calorimeter referred to as the controlled atmosphere pyrolysis apparatus (CAPA)~\cite{Semmes:IAFSS11}, in which the sample is surrounded by nitrogen to prevent ignition. Measured and predicted mass loss rates at imposed heat fluxes of 20~kW/m$^2$, 40~kW/m$^2$, and 60~kW/m$^2$ are shown in Fig.~\ref{MLR_Cardboard}.

\begin{figure}[h!]
\begin{tabular*}{\textwidth}{l@{\extracolsep{\fill}}r}
\includegraphics[height=2.15in]{SCRIPT_FIGURES/FAA_Polymers/Cardboard_DW_20} &
\includegraphics[height=2.15in]{SCRIPT_FIGURES/FAA_Polymers/Cardboard_DW_40} \\
 & \includegraphics[height=2.15in]{SCRIPT_FIGURES/FAA_Polymers/Cardboard_DW_60}
\end{tabular*}
\caption[Mass loss rate of corrugated cardboard]{Mass loss rate of corrugated cardboard.}
\label{MLR_Cardboard}
\end{figure}


\clearpage

\section{Wood Cribs and Similar Wood/Paper Combustibles}

A common combustible used in fire research and testing is a wood crib; that is, a uniform array of wooden dowels of nearly square cross section. However, a stack of wooden pallets can be viewed as a wood crib as well. Whatever its form, these cribs tend to burn at a consistent and predictable rate, making them ideal for fire testing.

\newpage

\subsection{Aalto Woods}
\label{Aalto_Woods_HRRPUA}

Cone calorimeter measurements for 2~cm thick samples of spruce and pine under flaming, inert, and smoldering conditions are shown in Figures~\ref{Aalto_Woods_HRR} and~\ref{Aalto_Woods_MLR}. Water evaporates and wood pyrolyzes through their independent reactions, and in presence of oxygen, char residue from wood pyrolysis oxidizes:
\begin{eqnarray}
   \hbox{Water} &\to& \hbox{Water vapor} \\
   \hbox{Wood} &\to& \nu_{\rm char} \, \hbox{Char} + (1-\nu_{\rm char}) \, \hbox{Pyrolyzate} \\
   \hbox{Char} + 2.67 \, \hbox{O}_2 &\to& 3.67 \, \hbox{CO}_2
\end{eqnarray}
The parameters for the reactions are listed in Table~\ref{Aalto_Woods_Properties}.

\begin{table}[h!]
\caption[Properties of spruce and pine]{Properties of spruce and pine, as estimated by PROPTI~\cite{Rinta-Paavola:2023}.}
\begin{center}
\begin{tabular}{|l|c|c|c|}
\hline
Property                     & Units         & Spruce                & Pine                                        \\ \hline \hline
\multicolumn{4}{|c|}{Moisture}                                                                                     \\ \hline
Density                      & kg/m$^3$      & \multicolumn{2}{|c|}{1000}                                          \\ \hline
Conductivity                 & W/m/K         & \multicolumn{2}{|c|}{0.6}                                           \\ \hline
Specific Heat                & kJ/kg/K       & \multicolumn{2}{|c|}{4.2}                                           \\ \hline
Emissivity, $\epsilon$       &               & \multicolumn{2}{|c|}{0.9}                                           \\ \hline
Pre-Exp.~Factor, $A$         & s$^{-1}$      & \multicolumn{2}{|c|}{$8.37 \times 10^{12}$}                         \\ \hline
Activation Energy, $E$       & J/mol         & \multicolumn{2}{|c|}{$1.21 \times 10^{5}$}                          \\ \hline
Heat of Reaction             & kJ/kg         & \multicolumn{2}{|c|}{2500}                                          \\ \hline
\multicolumn{4}{|c|}{Wood}                                                                                         \\ \hline
Density                      & kg/m$^3$      & 408                             & 493                               \\ \hline
Conductivity                 & W/m/K         & $3.16\times 10^{-4}\, T-0.0305$ & $3.57\times 10^{-4}\, T-0.00462$  \\ \hline
Specific Heat                & kJ/kg/K       & \multicolumn{2}{|c|}{$0.0044 \,T - 0.414$}                          \\ \hline
Emissivity, $\epsilon$       &               & 0.9                             & 0.9                               \\ \hline
Pre-Exp.~Factor, $A$         & s$^{-1}$      & $2.12 \times 10^{11}$           & $1.8 \times 10^{11}$              \\ \hline
Activation Energy, $E$       & J/mol         & $1.91 \times 10^{5}$            & $1.85 \times 10^{5}$              \\ \hline
Reaction Order, $n_{\rm s}$  &               & 1.89                            & 1.89                              \\ \hline
Char Yield, $\nu_{\rm char}$ &               & 0.15                            & 0.16                              \\ \hline
Heat of Reaction             & kJ/kg         & 112                             & 140                               \\ \hline
Heat of Combustion           & kJ/kg         & 13750                           & 13850                             \\ \hline
\multicolumn{4}{|c|}{Char}                                                                                         \\ \hline
Density                      & kg/m$^3$      & 52.5                            & 104                               \\ \hline
Conductivity                 & W/m/K         & \multicolumn{2}{|c|}{$8.2\times 10^{-5}\, T+0.091$}                 \\ \hline
Specific Heat                & kJ/kg/K       & \multicolumn{2}{|c|}{$1.43+0.000355\,T-7.32\times 10^4/T^2$}        \\ \hline
Emissivity, $\epsilon$       &               & 0.84                            & 0.84                              \\ \hline
Pre-Exp.~Factor, $A$         & s$^{-1}$      & 3.75                            & 1.79                              \\ \hline
Activation Energy, $E$       & J/mol         & 27685                           & 27685                             \\ \hline
$n_{\rm O_2}$                &               & 0.68                            & 0.68                              \\ \hline
Heat of Reaction             & kJ/kg         & -32000                          & -32000                            \\ \hline
\end{tabular}
\end{center}
\label{Aalto_Woods_Properties}
\end{table}


\begin{figure}[!h]
\begin{tabular*}{\textwidth}{l@{\extracolsep{\fill}}r}
\includegraphics[height=2.15in]{SCRIPT_FIGURES/Aalto_Woods/pine_flaming_25} &
\includegraphics[height=2.15in]{SCRIPT_FIGURES/Aalto_Woods/spruce_flaming_25} \\
\includegraphics[height=2.15in]{SCRIPT_FIGURES/Aalto_Woods/pine_flaming_35} &
\includegraphics[height=2.15in]{SCRIPT_FIGURES/Aalto_Woods/spruce_flaming_35} \\
\includegraphics[height=2.15in]{SCRIPT_FIGURES/Aalto_Woods/pine_flaming_50} &
\includegraphics[height=2.15in]{SCRIPT_FIGURES/Aalto_Woods/spruce_flaming_50} \\
\end{tabular*}
\caption[Aalto Woods heat release rates]{Aalto Woods heat release rates.}
\label{Aalto_Woods_HRR}
\end{figure}

\FloatBarrier

\begin{figure}[p]
\begin{tabular*}{\textwidth}{l@{\extracolsep{\fill}}r}
\includegraphics[height=2.15in]{SCRIPT_FIGURES/Aalto_Woods/pine_N2_35} &
\includegraphics[height=2.15in]{SCRIPT_FIGURES/Aalto_Woods/spruce_N2_35} \\
\includegraphics[height=2.15in]{SCRIPT_FIGURES/Aalto_Woods/pine_N2_50} &
\includegraphics[height=2.15in]{SCRIPT_FIGURES/Aalto_Woods/spruce_N2_50} \\
\includegraphics[height=2.15in]{SCRIPT_FIGURES/Aalto_Woods/pine_smolder_20} &
\includegraphics[height=2.15in]{SCRIPT_FIGURES/Aalto_Woods/spruce_smolder_25} \\
\includegraphics[height=2.15in]{SCRIPT_FIGURES/Aalto_Woods/pine_smolder_30} &
\includegraphics[height=2.15in]{SCRIPT_FIGURES/Aalto_Woods/spruce_smolder_35}
\end{tabular*}
\caption[Aalto Woods mass loss rates]{Aalto Woods mass loss rates.}
\label{Aalto_Woods_MLR}
\end{figure}

\FloatBarrier

The model for spruce estimated using the cone calorimeter measurements is employed in predicting heat release rates in an ISO 9705 room corner test~\cite{Sundstrom:1998}, and in a modified room corner test~\cite{Hietaniemi:2001} (in Finnish, English description of the test available in~\cite{Hietaniemi:1}). The selected test for simulation from~\cite{Sundstrom:1998} is M12, where the walls and ceiling of the test compartment is lined with 10~mm thick spruce timber. In the modified room corner test~\cite{Hietaniemi:2001}, the back wall opposite to the door opening, 240~cm length of an adjacent side wall, and a triangle shaped section of the ceiling connecting the two, are lined with 22~mm thick spruce timber. The gas burner is located in the timber-lined corner. All the other surfaces in the modified test, and the floor in M12~\cite{Sundstrom:1998} test are of concrete, as are the backing surfaces for timber.

For simulations of both room corner tests~\cite{Sundstrom:1998,Hietaniemi:2001}, concrete properties are assumed as given for lightweight concrete in~\cite{Hietaniemi:2001}: thickness of 20~cm, thermal conductivity of 0.1~\si{W/(m.K)} and thermal diffusivity of 2 $\times$ 10$^{-7}$ m$^2$/s. Assuming a specific heat of 0.88~\si{kJ/(kg.K)} for concrete~\cite{Drysdale:2011}, gives a density of 568~kg/m$^3$ from thermal diffusivity.

Figure~\ref{Aalto_Woods_RCT} shows heat release rates in room corner test M12 and in the modified room corner test. The burner output of 100~kW is reduced from both the experimental and simulation data, thus the figure shows only the heat release contribution from burning timber.

\begin{figure}[!h]
\begin{tabular*}{\textwidth}{l@{\extracolsep{\fill}}r}
\includegraphics[height=2.15in]{SCRIPT_FIGURES/Aalto_Woods/Roomcorner_M12} &
\includegraphics[height=2.15in]{SCRIPT_FIGURES/Aalto_Woods/Roomcorner_modified} \\
\end{tabular*}
\caption[Heat release in room corner tests, Aalto Woods model]{Heat release in room corner tests, Aalto Woods model}
\label{Aalto_Woods_RCT}
\end{figure}


\clearpage

\subsection{BST/FRS Wood Crib Experiments}
\label{BST_FRS_wood_cribs_temperature}

Figures~\ref{BST_FRS_wood_cribs_Tests_1to2} through \ref{BST_FRS_wood_cribs_Tests_5to6} display comparisons of the measured and predicted gas temperatures along three ``crib lines'' within a long compartment filled with 33 wooden cribs. The experiments are described in Sec.~\ref{BST_FRS_wood_cribs_description}. The crib lines are located at row 2 (back), 6 (middle) and 10 (front). Each measurement is an average of three thermocouples at the same distance.

\begin{figure}[!h]
\begin{tabular*}{\textwidth}{l@{\extracolsep{\fill}}r}
\includegraphics[height=2.15in]{SCRIPT_FIGURES/BST_FRS_wood_cribs/BST_FRS_1_Line2} &
\includegraphics[height=2.15in]{SCRIPT_FIGURES/BST_FRS_wood_cribs/BST_FRS_2_Line2} \\
\includegraphics[height=2.15in]{SCRIPT_FIGURES/BST_FRS_wood_cribs/BST_FRS_1_Line6} &
\includegraphics[height=2.15in]{SCRIPT_FIGURES/BST_FRS_wood_cribs/BST_FRS_2_Line6} \\
\includegraphics[height=2.15in]{SCRIPT_FIGURES/BST_FRS_wood_cribs/BST_FRS_1_Line10} &
\includegraphics[height=2.15in]{SCRIPT_FIGURES/BST_FRS_wood_cribs/BST_FRS_2_Line10}
\end{tabular*}
\caption[BST/FRS Wood Cribs temperatures, Tests 1 and 2]{BST/FRS Wood Cribs temperatures, Tests 1 (left) and 2 (right).}
\label{BST_FRS_wood_cribs_Tests_1to2}
\end{figure}

\begin{figure}[p]
\begin{tabular*}{\textwidth}{l@{\extracolsep{\fill}}r}
\includegraphics[height=2.15in]{SCRIPT_FIGURES/BST_FRS_wood_cribs/BST_FRS_3_Line2} &
\includegraphics[height=2.15in]{SCRIPT_FIGURES/BST_FRS_wood_cribs/BST_FRS_4_Line2} \\
\includegraphics[height=2.15in]{SCRIPT_FIGURES/BST_FRS_wood_cribs/BST_FRS_3_Line6} &
\includegraphics[height=2.15in]{SCRIPT_FIGURES/BST_FRS_wood_cribs/BST_FRS_4_Line6} \\
\includegraphics[height=2.15in]{SCRIPT_FIGURES/BST_FRS_wood_cribs/BST_FRS_3_Line10} &
\includegraphics[height=2.15in]{SCRIPT_FIGURES/BST_FRS_wood_cribs/BST_FRS_4_Line10}
\end{tabular*}
\caption[BST/FRS Wood Cribs temperatures, Tests 3 and 4]{BST/FRS Wood Cribs temperatures, Tests 3 (left) and 4 (right).}
\label{BST_FRS_wood_cribs_Tests_3to4}
\end{figure}

\begin{figure}[p]
\begin{tabular*}{\textwidth}{l@{\extracolsep{\fill}}r}
\includegraphics[height=2.15in]{SCRIPT_FIGURES/BST_FRS_wood_cribs/BST_FRS_5_Line2} &
\includegraphics[height=2.15in]{SCRIPT_FIGURES/BST_FRS_wood_cribs/BST_FRS_6_Line2} \\
\includegraphics[height=2.15in]{SCRIPT_FIGURES/BST_FRS_wood_cribs/BST_FRS_5_Line6} &
\includegraphics[height=2.15in]{SCRIPT_FIGURES/BST_FRS_wood_cribs/BST_FRS_6_Line6} \\
\includegraphics[height=2.15in]{SCRIPT_FIGURES/BST_FRS_wood_cribs/BST_FRS_5_Line10} &
\includegraphics[height=2.15in]{SCRIPT_FIGURES/BST_FRS_wood_cribs/BST_FRS_6_Line10}
\end{tabular*}
\caption[BST/FRS Wood Cribs temperatures, Tests 5 and 6]{BST/FRS Wood Cribs temperatures, Tests 5 (left) and 6 (right).}
\label{BST_FRS_wood_cribs_Tests_5to6}
\end{figure}

\clearpage


\subsection{NIST/NRC Transient Combustibles}

Figure~\ref{NIST_NRC_Transient_Combustibles_cribs} compares measured and predicted heat release rates for a single wood crib and arrays of multiple cribs. The simulations are all performed using Lagrangian particles as surrogates for the 56~cm long, 3.8~cm square pine sticks. The grid resolution in all cases is 8~cm, with additional simulations performed with 4~cm and 2~cm resolution for the single crib.

\begin{figure}[!h]
\begin{tabular*}{\textwidth}{l@{\extracolsep{\fill}}r}
\includegraphics[height=2.15in]{SCRIPT_FIGURES/NIST_NRC_Transient_Combustibles/crib_1x1x1_HRR} &
\includegraphics[height=2.15in]{SCRIPT_FIGURES/NIST_NRC_Transient_Combustibles/crib_2x1x1_HRR} \\
\includegraphics[height=2.15in]{SCRIPT_FIGURES/NIST_NRC_Transient_Combustibles/crib_2x2x1_HRR} &
\includegraphics[height=2.15in]{SCRIPT_FIGURES/NIST_NRC_Transient_Combustibles/crib_2x2x2_HRR}
\end{tabular*}
\caption[NIST/NRC Transient Combustibles: wood cribs]{NIST/NRC Transient Combustibles: wood cribs.}
\label{NIST_NRC_Transient_Combustibles_cribs}
\end{figure}

\FloatBarrier

Figure~\ref{NIST_NRC_Transient_Combustibles_boxes} compares measured and predicted heat release rates for a single cardbard box filled with shredded craft paper and arrays of multiple boxes. The simulations are all performed using Lagrangian particles as surrogates for the small paper strips. The grid resolution in all cases is 5~cm.

\begin{figure}[!h]
\begin{tabular*}{\textwidth}{l@{\extracolsep{\fill}}r}
\includegraphics[height=2.15in]{SCRIPT_FIGURES/NIST_NRC_Transient_Combustibles/box_1x1x1_HRR} &
\includegraphics[height=2.15in]{SCRIPT_FIGURES/NIST_NRC_Transient_Combustibles/box_2x1x1_HRR} \\
\includegraphics[height=2.15in]{SCRIPT_FIGURES/NIST_NRC_Transient_Combustibles/box_2x2x1_HRR} &
\includegraphics[height=2.15in]{SCRIPT_FIGURES/NIST_NRC_Transient_Combustibles/box_2x2x2_HRR}
\end{tabular*}
\caption[NIST/NRC Transient Combustibles: boxes]{NIST/NRC Transient Combustibles: boxes.}
\label{NIST_NRC_Transient_Combustibles_boxes}
\end{figure}

\FloatBarrier

Figure~\ref{NIST_NRC_Transient_Combustibles_pallets} compares measured and predicted heat release rates for a stack of two, four and eight wood pallets. The simulations are all performed using Lagrangian particles as surrogates for the wooden planks. The grid resolution in all cases is 6~cm.

\begin{figure}[!h]
\begin{tabular*}{\textwidth}{l@{\extracolsep{\fill}}r}
\includegraphics[height=2.15in]{SCRIPT_FIGURES/NIST_NRC_Transient_Combustibles/pallet_1x1x2_HRR} &
\includegraphics[height=2.15in]{SCRIPT_FIGURES/NIST_NRC_Transient_Combustibles/pallet_1x1x4_HRR} \\
\multicolumn{2}{c}{\includegraphics[height=2.15in]{SCRIPT_FIGURES/NIST_NRC_Transient_Combustibles/pallet_1x1x8_HRR}}
\end{tabular*}
\caption[NIST/NRC Transient Combustibles: pallets]{NIST/NRC Transient Combustibles: pallets.}
\label{NIST_NRC_Transient_Combustibles_pallets}
\end{figure}


\clearpage


\subsection{SP Wood Crib Experiments}

These experiments and the modeling strategy are described in Sec.~\ref{SP_Wood_Cribs_Description}. Briefly, four piles of 1:4 scale wood pallets are positioned in a wind tunnel with various separation distances. The upwind pile is ignited and the fire spreads from pile to pile. The wind velocity for Test~1 is 0.3~m/s; and for Test~12 it is 0.9~m/s. For all others, it is 0.6~m/s. Tests~1, 4, and 12 have only a single pile; the rest have four.

Figures~\ref{SP_Wood_Cribs_HRR_1} and \ref{SP_Wood_Cribs_HRR_2} display the measured and predicted HRR for 11 of the 12 SP Wood Crib experiments. Test~2 was set up differently than the other experiments and was not used in the analysis.

\begin{figure}[!ht]
\begin{tabular*}{\textwidth}{l@{\extracolsep{\fill}}r}
\includegraphics[height=2.15in]{SCRIPT_FIGURES/SP_Wood_Cribs/Test_1_HRR} &
 \\
\includegraphics[height=2.15in]{SCRIPT_FIGURES/SP_Wood_Cribs/Test_3_HRR} &
\includegraphics[height=2.15in]{SCRIPT_FIGURES/SP_Wood_Cribs/Test_4_HRR} \\
\includegraphics[height=2.15in]{SCRIPT_FIGURES/SP_Wood_Cribs/Test_5_HRR} &
\includegraphics[height=2.15in]{SCRIPT_FIGURES/SP_Wood_Cribs/Test_6_HRR}
\end{tabular*}
\caption[SP Wood Cribs heat release rates, Tests 1-6]{SP Wood Cribs heat release rates, Tests 1-6.}
\label{SP_Wood_Cribs_HRR_1}
\end{figure}

\begin{figure}[p]
\begin{tabular*}{\textwidth}{l@{\extracolsep{\fill}}r}
\includegraphics[height=2.15in]{SCRIPT_FIGURES/SP_Wood_Cribs/Test_7_HRR} &
\includegraphics[height=2.15in]{SCRIPT_FIGURES/SP_Wood_Cribs/Test_8_HRR} \\
\includegraphics[height=2.15in]{SCRIPT_FIGURES/SP_Wood_Cribs/Test_9_HRR} &
\includegraphics[height=2.15in]{SCRIPT_FIGURES/SP_Wood_Cribs/Test_10_HRR} \\
\includegraphics[height=2.15in]{SCRIPT_FIGURES/SP_Wood_Cribs/Test_11_HRR} &
\includegraphics[height=2.15in]{SCRIPT_FIGURES/SP_Wood_Cribs/Test_12_HRR}
\end{tabular*}
\caption[SP Wood Cribs heat release rates, Tests 7-12]{SP Wood Cribs heat release rates, Tests 7-12.}
\label{SP_Wood_Cribs_HRR_2}
\end{figure}


\clearpage

\section{Liquid Pool Fires}
\label{sec:Liquid_Pool_Fires_MLR}

\subsection{DoJ/HAI Pool Fires}

Table~\ref{DoJ_HAI_Matrix} lists the relevant parameters for a series of gasolene and kerosene pool fire experiments.

Figures~\ref{DoJ_HAI_Pool_Fires_1} through \ref{DoJ_HAI_Pool_Fires_6} display measured and predicted burning rate of a variety of gasoline and kerosene pool fires in 0.3~m, 0.6~m, and 1.2~m square pans of various depths and two substrates---concrete and vinyl.

\newpage

\begin{table}[p]
\caption[Summary of DoJ/HAI Diked Fire Tests]{Summary of DoJ/HAI Diked Fire Tests.}
\begin{center}
\begin{tabular}{|l|c|c|c|c|}
\hline
Test ID   & Pan Side Length (m)  & Fuel &  Depth (mm)  & Substrate  \\ \hline
DSF1-C-CC-G-1-0.093    & 0.31  & Gasoline  & 1  & Coated Concrete   \\ \hline
DSF2-C-CC-G-5-0.093    & 0.31  & Gasoline  & 5  & Coated Concrete   \\ \hline
DSF3-C-CC-G-10-0.093   & 0.31  & Gasoline  & 10 & Coated Concrete   \\ \hline
DSF4-C-CC-G-20-0.093   & 0.31  & Gasoline  & 20 & Coated Concrete   \\ \hline
DSF5-C-CC-K-1-0.093    & 0.31  & Kerosene  & 1  & Coated Concrete   \\ \hline
DSF6-C-CC-K-5-0.093    & 0.31  & Kerosene  & 5  & Coated Concrete   \\ \hline
DSF7-C-CC-K-10-0.093   & 0.31  & Kerosene  & 10 & Coated Concrete   \\ \hline
DSF8-C-CC-K-20-0.093   & 0.31  & Kerosene  & 20 & Coated Concrete   \\ \hline
DSF9-C-CC-G-1-0.372    & 0.61  & Gasoline  & 1  & Coated Concrete   \\ \hline
DSF10-C-CC-G-5-0.372   & 0.61  & Gasoline  & 5  & Coated Concrete   \\ \hline
DSF11-C-CC-G-10-0.372  & 0.61  & Gasoline  & 10 & Coated Concrete   \\ \hline
DSF12-C-CC-G-20-0.372  & 0.61  & Gasoline  & 20 & Coated Concrete   \\ \hline
DSF13-C-CC-K-1-0.372   & 0.61  & Kerosene  & 1  & Coated Concrete   \\ \hline
DSF14-C-CC-K-5-0.372   & 0.61  & Kerosene  & 5  & Coated Concrete   \\ \hline
DSF15-C-CC-K-10-0.372  & 0.61  & Kerosene  & 10 & Coated Concrete   \\ \hline
DSF16-C-CC-K-20-0.372  & 0.61  & Kerosene  & 20 & Coated Concrete   \\ \hline
DSF17-C-CC-G-1-1.488   & 1.22  & Gasoline  & 1  & Coated Concrete   \\ \hline
DSF18-C-CC-G-2-1.488   & 1.22  & Gasoline  & 2  & Coated Concrete   \\ \hline
DSF19-C-CC-G-3-1.488   & 1.22  & Gasoline  & 3  & Coated Concrete   \\ \hline
DSF20-C-CC-G-4-1.488   & 1.22  & Gasoline  & 4  & Coated Concrete   \\ \hline
DSF21-C-CC-G-5-1.488   & 1.22  & Gasoline  & 5  & Coated Concrete   \\ \hline
DSF22-C-CC-G-20-1.488  & 1.22  & Gasoline  & 20 & Coated Concrete   \\ \hline
DSF25-C-V-G-1-0.093    & 0.31  & Gasoline  & 1  & Vinyl             \\ \hline
DSF26-C-V-G-5-0.093    & 0.31  & Gasoline  & 5  & Vinyl             \\ \hline
DSF27-C-V-G-10-0.093   & 0.31  & Gasoline  & 10 & Vinyl             \\ \hline
DSF28-C-V-G-20-0.093   & 0.31  & Gasoline  & 20 & Vinyl             \\ \hline
DSF29-C-V-K-1-0.093    & 0.31  & Kerosene  & 1  & Vinyl             \\ \hline
DSF30-C-V-K-5-0.093    & 0.31  & Kerosene  & 5  & Vinyl             \\ \hline
DSF31-C-V-K-10-0.093   & 0.31  & Kerosene  & 10 & Vinyl             \\ \hline
DSF32-C-V-K-20-0.093   & 0.31  & Kerosene  & 20 & Vinyl             \\ \hline
DSF33-C-V-G-1-0.372    & 0.61  & Gasoline  & 1  & Vinyl             \\ \hline
DSF34-C-V-G-5-0.372    & 0.61  & Gasoline  & 5  & Vinyl             \\ \hline
DSF35-C-V-G-10-0.372   & 0.61  & Gasoline  & 10 & Vinyl             \\ \hline
DSF36-C-V-G-20-0.372   & 0.61  & Gasoline  & 20 & Vinyl             \\ \hline
DSF37-C-V-K-1-0.372    & 0.61  & Kerosene  & 1  & Vinyl             \\ \hline
DSF38-C-V-K-5-0.372    & 0.61  & Kerosene  & 5  & Vinyl             \\ \hline
DSF39-C-V-K-10-0.372   & 0.61  & Kerosene  & 10 & Vinyl             \\ \hline
DSF40-C-V-K-20-0.372   & 0.61  & Kerosene  & 20 & Vinyl             \\ \hline
DSF41-C-V-G-1-1.488    & 1.22  & Gasoline  & 1  & Vinyl             \\ \hline
DSF42-C-V-G-2-1.488    & 1.22  & Gasoline  & 2  & Vinyl             \\ \hline
DSF43-C-V-G-3-1.488    & 1.22  & Gasoline  & 3  & Vinyl             \\ \hline
DSF44-C-V-G-4-1.488    & 1.22  & Gasoline  & 4  & Vinyl             \\ \hline
DSF45-C-V-G-5-1.488    & 1.22  & Gasoline  & 5  & Vinyl             \\ \hline
DSF46-C-V-G-15-1.488   & 1.22  & Gasoline  & 15 & Vinyl             \\ \hline
\end{tabular}
\end{center}
\label{DoJ_HAI_Matrix}
\end{table}

\begin{figure}[p]
\begin{tabular*}{\textwidth}{l@{\extracolsep{\fill}}r}
\includegraphics[height=2.15in]{SCRIPT_FIGURES/DoJ_HAI_Pool_Fires/DSF1} &
\includegraphics[height=2.15in]{SCRIPT_FIGURES/DoJ_HAI_Pool_Fires/DSF2} \\
\includegraphics[height=2.15in]{SCRIPT_FIGURES/DoJ_HAI_Pool_Fires/DSF3} &
\includegraphics[height=2.15in]{SCRIPT_FIGURES/DoJ_HAI_Pool_Fires/DSF4} \\
\includegraphics[height=2.15in]{SCRIPT_FIGURES/DoJ_HAI_Pool_Fires/DSF5} &
\includegraphics[height=2.15in]{SCRIPT_FIGURES/DoJ_HAI_Pool_Fires/DSF6} \\
\includegraphics[height=2.15in]{SCRIPT_FIGURES/DoJ_HAI_Pool_Fires/DSF7} &
\includegraphics[height=2.15in]{SCRIPT_FIGURES/DoJ_HAI_Pool_Fires/DSF8}
\end{tabular*}
\caption[DoJ/HAI Pool Fires, Tests 1-8]{DoJ/HAI Pool Fires, Tests 1-8.}
\label{DoJ_HAI_Pool_Fires_1}
\end{figure}

\begin{figure}[p]
\begin{tabular*}{\textwidth}{l@{\extracolsep{\fill}}r}
\includegraphics[height=2.15in]{SCRIPT_FIGURES/DoJ_HAI_Pool_Fires/DSF9} &
\includegraphics[height=2.15in]{SCRIPT_FIGURES/DoJ_HAI_Pool_Fires/DSF10} \\
\includegraphics[height=2.15in]{SCRIPT_FIGURES/DoJ_HAI_Pool_Fires/DSF11} &
\includegraphics[height=2.15in]{SCRIPT_FIGURES/DoJ_HAI_Pool_Fires/DSF12} \\
\includegraphics[height=2.15in]{SCRIPT_FIGURES/DoJ_HAI_Pool_Fires/DSF13} &
\includegraphics[height=2.15in]{SCRIPT_FIGURES/DoJ_HAI_Pool_Fires/DSF14} \\
\includegraphics[height=2.15in]{SCRIPT_FIGURES/DoJ_HAI_Pool_Fires/DSF15} &
\includegraphics[height=2.15in]{SCRIPT_FIGURES/DoJ_HAI_Pool_Fires/DSF16}
\end{tabular*}
\caption[DoJ/HAI Pool Fires, Tests 9-16]{DoJ/HAI Pool Fires, Tests 9-16.}
\label{DoJ_HAI_Pool_Fires_2}
\end{figure}

\begin{figure}[p]
\begin{tabular*}{\textwidth}{l@{\extracolsep{\fill}}r}
\includegraphics[height=2.15in]{SCRIPT_FIGURES/DoJ_HAI_Pool_Fires/DSF17} &
\includegraphics[height=2.15in]{SCRIPT_FIGURES/DoJ_HAI_Pool_Fires/DSF18} \\
\includegraphics[height=2.15in]{SCRIPT_FIGURES/DoJ_HAI_Pool_Fires/DSF19} &
\includegraphics[height=2.15in]{SCRIPT_FIGURES/DoJ_HAI_Pool_Fires/DSF20} \\
\includegraphics[height=2.15in]{SCRIPT_FIGURES/DoJ_HAI_Pool_Fires/DSF21} &
\includegraphics[height=2.15in]{SCRIPT_FIGURES/DoJ_HAI_Pool_Fires/DSF22}
\end{tabular*}
\caption[DoJ/HAI Pool Fires, Tests 17-22]{DoJ/HAI Pool Fires, Tests 17-22.}
\label{DoJ_HAI_Pool_Fires_3}
\end{figure}

\begin{figure}[p]
\begin{tabular*}{\textwidth}{l@{\extracolsep{\fill}}r}
\includegraphics[height=2.15in]{SCRIPT_FIGURES/DoJ_HAI_Pool_Fires/DSF25} &
\includegraphics[height=2.15in]{SCRIPT_FIGURES/DoJ_HAI_Pool_Fires/DSF26} \\
\includegraphics[height=2.15in]{SCRIPT_FIGURES/DoJ_HAI_Pool_Fires/DSF27} &
\includegraphics[height=2.15in]{SCRIPT_FIGURES/DoJ_HAI_Pool_Fires/DSF28} \\
\includegraphics[height=2.15in]{SCRIPT_FIGURES/DoJ_HAI_Pool_Fires/DSF29} &
\includegraphics[height=2.15in]{SCRIPT_FIGURES/DoJ_HAI_Pool_Fires/DSF30} \\
\includegraphics[height=2.15in]{SCRIPT_FIGURES/DoJ_HAI_Pool_Fires/DSF31} &
\includegraphics[height=2.15in]{SCRIPT_FIGURES/DoJ_HAI_Pool_Fires/DSF32}
\end{tabular*}
\caption[DoJ/HAI Pool Fires, Tests 25-32]{DoJ/HAI Pool Fires, Tests 25-32.}
\label{DoJ_HAI_Pool_Fires_4}
\end{figure}

\begin{figure}[p]
\begin{tabular*}{\textwidth}{l@{\extracolsep{\fill}}r}
\includegraphics[height=2.15in]{SCRIPT_FIGURES/DoJ_HAI_Pool_Fires/DSF33} &
\includegraphics[height=2.15in]{SCRIPT_FIGURES/DoJ_HAI_Pool_Fires/DSF34} \\
\includegraphics[height=2.15in]{SCRIPT_FIGURES/DoJ_HAI_Pool_Fires/DSF35} &
\includegraphics[height=2.15in]{SCRIPT_FIGURES/DoJ_HAI_Pool_Fires/DSF36} \\
\includegraphics[height=2.15in]{SCRIPT_FIGURES/DoJ_HAI_Pool_Fires/DSF37} &
\includegraphics[height=2.15in]{SCRIPT_FIGURES/DoJ_HAI_Pool_Fires/DSF38} \\
\includegraphics[height=2.15in]{SCRIPT_FIGURES/DoJ_HAI_Pool_Fires/DSF39} &
\includegraphics[height=2.15in]{SCRIPT_FIGURES/DoJ_HAI_Pool_Fires/DSF40}
\end{tabular*}
\caption[DoJ/HAI Pool Fires, Tests 33-40]{DoJ/HAI Pool Fires, Tests 33-40.}
\label{DoJ_HAI_Pool_Fires_5}
\end{figure}

\begin{figure}[p]
\begin{tabular*}{\textwidth}{l@{\extracolsep{\fill}}r}
\includegraphics[height=2.15in]{SCRIPT_FIGURES/DoJ_HAI_Pool_Fires/DSF41} &
\includegraphics[height=2.15in]{SCRIPT_FIGURES/DoJ_HAI_Pool_Fires/DSF42} \\
\includegraphics[height=2.15in]{SCRIPT_FIGURES/DoJ_HAI_Pool_Fires/DSF43} &
\includegraphics[height=2.15in]{SCRIPT_FIGURES/DoJ_HAI_Pool_Fires/DSF44} \\
\includegraphics[height=2.15in]{SCRIPT_FIGURES/DoJ_HAI_Pool_Fires/DSF45} &
\includegraphics[height=2.15in]{SCRIPT_FIGURES/DoJ_HAI_Pool_Fires/DSF46}
\end{tabular*}
\caption[DoJ/HAI Pool Fires, Tests 41-46]{DoJ/HAI Pool Fires, Tests 41-46.}
\label{DoJ_HAI_Pool_Fires_6}
\end{figure}


\clearpage

\subsection{LEMTA/UGent Pool Fires}

These experiments are described in Section~\ref{LEMTA_UGent_Pool_Fires_Description}.

Shown on the following pages are comparisons of measured and predicted burning rates of heptane and methanol pools, along with temperatures measured at various heights (5~mm, 20~mm, 35~mm) above the floor of the pan for which the starting depth of the liquid was approximately 46~mm.

\newpage

\begin{figure}[p]
\begin{tabular*}{\textwidth}{l@{\extracolsep{\fill}}r}
\includegraphics[height=2.15in]{SCRIPT_FIGURES/LEMTA_UGent_Pool_Fires/H5_MLRPUA} &
\includegraphics[height=2.15in]{SCRIPT_FIGURES/LEMTA_UGent_Pool_Fires/H5_temp} \\
\includegraphics[height=2.15in]{SCRIPT_FIGURES/LEMTA_UGent_Pool_Fires/H8_MLRPUA} &
\includegraphics[height=2.15in]{SCRIPT_FIGURES/LEMTA_UGent_Pool_Fires/H8_temp} \\
\includegraphics[height=2.15in]{SCRIPT_FIGURES/LEMTA_UGent_Pool_Fires/H10_MLRPUA} &
\includegraphics[height=2.15in]{SCRIPT_FIGURES/LEMTA_UGent_Pool_Fires/H10_temp} \\
\includegraphics[height=2.15in]{SCRIPT_FIGURES/LEMTA_UGent_Pool_Fires/H15_MLRPUA} &
\includegraphics[height=2.15in]{SCRIPT_FIGURES/LEMTA_UGent_Pool_Fires/H15_temp}
\end{tabular*}
\caption[LEMTA/UGent Pool Fires burning rate and sub-surface temperature, heptane]{LEMTA/UGent Pool Fires burning rate and sub-surface temperature, heptane.}
\label{LEMTA_UGent_heptane}
\end{figure}

\begin{figure}[p]
\begin{tabular*}{\textwidth}{l@{\extracolsep{\fill}}r}
\includegraphics[height=2.15in]{SCRIPT_FIGURES/LEMTA_UGent_Pool_Fires/M5_MLRPUA} &
\includegraphics[height=2.15in]{SCRIPT_FIGURES/LEMTA_UGent_Pool_Fires/M5_temp} \\
\includegraphics[height=2.15in]{SCRIPT_FIGURES/LEMTA_UGent_Pool_Fires/M8_MLRPUA} &
\includegraphics[height=2.15in]{SCRIPT_FIGURES/LEMTA_UGent_Pool_Fires/M8_temp} \\
\includegraphics[height=2.15in]{SCRIPT_FIGURES/LEMTA_UGent_Pool_Fires/M10_MLRPUA} &
\includegraphics[height=2.15in]{SCRIPT_FIGURES/LEMTA_UGent_Pool_Fires/M10_temp} \\
\includegraphics[height=2.15in]{SCRIPT_FIGURES/LEMTA_UGent_Pool_Fires/M15_MLRPUA} &
\includegraphics[height=2.15in]{SCRIPT_FIGURES/LEMTA_UGent_Pool_Fires/M15_temp}
\end{tabular*}
\caption[LEMTA/UGent Pool Fires burning rate and sub-surface temperature, methanol]{LEMTA/UGent Pool Fires burning rate and sub-surface temperature, methanol.}
\label{LEMTA_UGent_methanol}
\end{figure}



\clearpage

\subsection{Pool Fire Measurements}

Shown below are comparisons of measured and predicted evaporation/burning rates of various liquid pools. Surface temperature comparisons of these same experiments are shown in Sec.~\ref{sec:Liquid_Pool_Surf_Temp}.

Figure~\ref{ASTM_E2058_Water_Evap_MLR} compares the measured and predicted evaporation rate of water subjected to a 50~kW/m$^2$ heat flux.

\begin{figure}[!h]
\centering
\includegraphics[height=2.15in]{SCRIPT_FIGURES/Pool_Fires/ASTM_E2058_Water_Evap}
\caption[ASTM E2058 fire propagation apparatus water evaporation at 50~kW/m$^2$ heat flux]{ASTM E2058 fire propagation apparatus water evaporation at 50~kW/m$^2$ heat flux.}
\label{ASTM_E2058_Water_Evap_MLR}
\end{figure}

\noindent
Figure~\ref{VTT_MLRPUA} shows the measured and predicted mass loss rates of 1.17 m (1.07 m$^2$) and 1.6 m (2.0 m$^2$) diameter heptane pool fires. The measured mass loss rates are averages of two or three individual experiments.

\begin{figure}[!ht]
\begin{tabular*}{\textwidth}{l@{\extracolsep{\fill}}r}
\includegraphics[height=2.15in]{SCRIPT_FIGURES/Pool_Fires/VTT_heptane_1_m2} &
\includegraphics[height=2.15in]{SCRIPT_FIGURES/Pool_Fires/VTT_heptane_2_m2} \\
\end{tabular*}
\caption[VTT Large Hall Test burning rate]{VTT Large Hall Test burning rate.}
\label{VTT_MLRPUA}
\end{figure}

\noindent
Figure~\ref{POOL_MLR} compares predicted and measured burning rates of a variety of liquid fuels.
\begin{figure}[p]
\begin{tabular*}{\textwidth}{l@{\extracolsep{\fill}}r}
\includegraphics[height=2.15in]{SCRIPT_FIGURES/Pool_Fires/acetone_1_m} &
\includegraphics[height=2.15in]{SCRIPT_FIGURES/Pool_Fires/benzene_1_m} \\
\includegraphics[height=2.15in]{SCRIPT_FIGURES/Pool_Fires/butane_1_m} &
\includegraphics[height=2.15in]{SCRIPT_FIGURES/Pool_Fires/ethanol_1_m} \\
\includegraphics[height=2.15in]{SCRIPT_FIGURES/Pool_Fires/heptane_1_m} &
\includegraphics[height=2.15in]{SCRIPT_FIGURES/Pool_Fires/methanol_1_m} \\
\end{tabular*}
\caption[Comparison of burning rates for various liquid pool fires]{Comparison of burning rates for various liquid pool fires.}
\label{POOL_MLR}
\end{figure}

\FloatBarrier

\subsection{NIST Pool Fires}

Figure~\ref{NIST_1_m_methanol_MLRPUA} displays the measured and predicted burning rate of a 100~cm diameter methanol pool fire experiment conducted by Sung~et~al. at NIST~\cite{Sung:TN2019}.

\begin{figure}[!ht]
\centering
\includegraphics[height=2.15in]{SCRIPT_FIGURES/NIST_Pool_Fires/NIST_Methanol_1m_pan_MLRPUA}
\caption[NIST 1 m methanol burning rate]{Burning rate of a 1 m methanol pool fire.}
\label{NIST_1_m_methanol_MLRPUA}
\end{figure}

\FloatBarrier

\subsection{Waterloo Methanol Pool Fire}

Figure~\ref{Waterloo_HRR} displays the measured and predicted burning rate for a 30~cm diameter methanol pool fire experiment conducted by Weckman at the University of Waterloo~\cite{Weckman:CF1996}. The experimental result came after at least 10~min of burning, whereas the model is only run for 2~min.

\begin{figure}[!ht]
\centering
\includegraphics[height=2.15in]{SCRIPT_FIGURES/Waterloo_Methanol/Waterloo_Methanol_MLRPUA}
\caption[Waterloo Methanol mass loss rate]{Waterloo Methanol mass loss rate.}
\label{Waterloo_HRR}
\end{figure}



\clearpage

\section{Vertical Flame Spread}

\subsection{NIST/NRC Parallel Panel Experiments}

The figures below contain predictions of the burning rate of various plastics in a parallel panel apparatus described in Sec.~\ref{NIST_NRC_Parallel_Panels_Description}.

\begin{figure}[!ht]
\centering
\includegraphics[height=2.15in]{SCRIPT_FIGURES/NIST_NRC_Parallel_Panels/PMMA_60_kW} \\
\includegraphics[height=2.15in]{SCRIPT_FIGURES/NIST_NRC_Parallel_Panels/PBT_60_kW} \\
\includegraphics[height=2.15in]{SCRIPT_FIGURES/NIST_NRC_Parallel_Panels/PVC_60_kW}
\caption[NIST/NRC Parallel Panels experiments]{NIST/NRC Parallel Panels experiments.}
\label{NIST_NRC_PP_HRR}
\end{figure}


\clearpage

\subsection{UMD SBI Experiment}

A description of this experiment is given in Sec.~\ref{UMD_SBI_Description}.

Figure~\ref{UMD_SBI_HRR} displays the measured and predicted HRR for upward spread over PMMA panels in the SBI (Single Burning Item) apparatus.

\begin{figure}[!ht]
\centering
\includegraphics[height=2.15in]{SCRIPT_FIGURES/UMD_SBI/HRR}
\caption[UMD SBI experiment, heat release rate]{UMD SBI experiment, heat release rate.}
\label{UMD_SBI_HRR}
\end{figure}

\clearpage

\section{Scaling Pyrolysis}\label{sec_Scaling Pyrolysis}

FDS has a scaling-based pyrolysis model which dynamically scales a reference heat release rate per unit area curve based on the energy absorbed on the surface.
This section evaluates the performance of FDS in scaling the buring rate using this model.
Each heat flux shown in the figures in this section is a validation heat transfer level (i.e., the reference heat flux is not shown).
Validation statistics are calculated based on the peak HRRPUA.

\subsection{Aalto Woods Experiments}\label{sec_Aalto_Woods_Materials}

Table ~\ref{Properties_Aalto_Woods_All} lists the relevant parameters for modeling the cone calorimeter experiments for this study.
Note that the published properties from the test report are simplified in this analysis. Each multi-phase properties is represented as single bulk value.
In addition, the temperature-dependent thermal conductivity and specific capacity are fixed at an average value.
The ignition temperatures were calculated based on times to ignition at the lower heat fluxes using the other properties.
The results of the simulations are shown below.

\begin{table}[!h]
\caption[Properties of Aalto Woods]{Properties of Aalto Woods~\cite{Rinta-Paavola:2023}.}
\centering
\begin{tabular}{|l|c|c|c|c|c|c|c|c|}
\hline
            & \centering$\Delta$& \centering$\rho$& \centering$k$& \centering$c_{p}$& \centering$\varepsilon$& \centering$T_{\mathrm{ign}}^{a}$&\centering$\Delta H_{c}$& $Y_{s}^{b}$ \\
Material    & \centering$\mathrm{\left(mm\right)}$ & \centering$\mathrm{\left(\frac{kg}{m^{3}}\right)}$ & \centering$\mathrm{\left(\frac{W}{m\cdot K}\right)}$ & \centering$\mathrm{\left(\frac{kJ}{(kg\cdot K}\right)}$ & \centering$\mathrm{( - )}$ &  \centering($\mathrm{^{\circ}C}$)   & \centering$\left(\mathrm{\frac{MJ}{kg}}\right)$ & $\mathrm{\left(\frac{g}{g}\right)}$ \\ \hline
\hline
Pine Flaming                                      & 20.0 & 493 & 0.10 & 0.91 & 0.90 & 485 & 13.8 & 0.015 \\\hline
Spruce Flaming                                    & 20.0 & 408 & 0.06 & 0.91 & 0.90 & 487 & 13.8 & 0.015 \\\hline
\end{tabular}
\label{Properties_Aalto_Woods_All}
\end{table}
\vspace{-0.4cm}
\noindent\footnotesize{$^a$ $T_{\mathrm{ign}}$ is calculated based on measured times to ignition.}\\
\noindent\footnotesize{$^b$ $Y_{s}$ based on data from the SFPE handbook~\cite{SFPE:Tewarson}.}\\

\begin{figure}[pb]
\begin{tabular*}{\textwidth}{l@{\extracolsep{\fill}}r}
\includegraphics[height=2.10in]{SCRIPT_FIGURES/Scaling_Pyrolysis/Aalto_Pine_Flaming_cone_20p0.pdf} &
\includegraphics[height=2.10in]{SCRIPT_FIGURES/Scaling_Pyrolysis/Aalto_Spruce_Flaming_cone_20p0.pdf} \\
\end{tabular*}
\caption[HRRPUA of Aalto Woods using scaling model]
{Aalto Woods - Comparison of predicted and measured heat release rate per unit area using scaling-based approach for cone calorimeter experiments.}
\label{Aalto_Woods_HRR_Wood-Based}
\end{figure}

\clearpage

\subsection{FAA Polymers Experiments}\label{sec_FAA_Polymers_Materials}

Table ~\ref{Properties_FAA_Polymers_All} lists the relevant parameters for modeling the cone calorimeter experiments for this study.
Note that the published properties from the test report are simplified in this analysis. Each multi-phase properties is represented as single bulk value.
In addition, the temperature-dependent thermal conductivity and specific capacity are fixed at an average value.
The ignition temperatures were calculated based on times to ignition at the lower heat fluxes using the other properties.
The results of the simulations are shown on the following pages.

\begin{table}[!h]
\caption[Properties of FAA Polymers]{Properties of FAA Polymers~\cite{Stoliarov:CF2009,Stoliarov:FM2012}.}
\centering
\begin{tabular}{|p{2.5cm}|c|c|c|c|c|c|c|c|}
\hline
            & \centering$\Delta^{a}$& \centering$\rho$& \centering$k$& \centering$c_{p}$& \centering$\varepsilon$& \centering$T_{\mathrm{ign}}^{b}$&\centering$\Delta H_{c}$& $Y_{s}$ \\
Material    & \centering$\mathrm{\left(mm\right)}$ & \centering$\mathrm{\left(\frac{kg}{m^{3}}\right)}$ & \centering$\mathrm{\left(\frac{W}{m\cdot K}\right)}$ & \centering$\mathrm{\left(\frac{kJ}{(kg\cdot K}\right)}$ & \centering$\mathrm{( - )}$ &  \centering($\mathrm{^{\circ}C}$)   & \centering$\left(\mathrm{\frac{MJ}{kg}}\right)$ & $\mathrm{\left(\frac{g}{g}\right)}$ \\ \hline
\hline
HDPE                                              & 3.2, 8.1, 27.0 & 860  & 0.29 & 3.50 & 0.92 & 410 & 47.5 & 0.060 \\\hline
HIPS                                              & 3.2, 8.1, 27.0 & 950  & 0.22 & 2.00 & 0.86 & 425 & 39.2 & 0.164 \\\hline
PBT                                               & 4.0            & 1300 & 0.29 & 2.23 & 0.88 & 291 & 19.5 & 0.020 \\\hline
PBTGF                                             & 4.0            & 1520 & 0.36 & 1.68 & 0.87 & 235 & 19.5 & 0.020 \\\hline
PC                                                & 3.0, 5.5, 9.0  & 1180 & 0.22 & 1.90 & 0.90 & 530 & 25.6 & 0.112 \\\hline
PEEK                                              & 3.9            & 1300 & 0.28 & 2.05 & 0.90 & 463 & 22.8 & 0.020 \\\hline
PMMA                                              & 3.2, 8.1, 27.0 & 1100 & 0.20 & 2.20 & 0.85 & 373 & 33.5 & 0.022 \\\hline
PVC                                               & 3.0, 6.0, 9.0  & 1430 & 0.17 & 1.55 & 0.90 & 435 & 36.5 & 0.172 \\\hline
\end{tabular}
\label{Properties_FAA_Polymers_All}
\end{table}
\vspace{-0.4cm}
\noindent\footnotesize{$^a$ Comma separation indicates tests at multiple thicknesses.}\\
\noindent\footnotesize{$^b$ $T_{\mathrm{ign}}$ is calculated based on measured times to ignition. \\

\begin{figure}[p]
\begin{tabular*}{\textwidth}{l@{\extracolsep{\fill}}r}
\includegraphics[height=2.10in]{SCRIPT_FIGURES/Scaling_Pyrolysis/FAA_HDPE_cone_3p2.pdf} &
\includegraphics[height=2.10in]{SCRIPT_FIGURES/Scaling_Pyrolysis/FAA_HDPE_cone_8p1.pdf} \\
\includegraphics[height=2.10in]{SCRIPT_FIGURES/Scaling_Pyrolysis/FAA_HDPE_cone_27p0.pdf} &
\includegraphics[height=2.10in]{SCRIPT_FIGURES/Scaling_Pyrolysis/FAA_HIPS_cone_3p2.pdf} \\
\includegraphics[height=2.10in]{SCRIPT_FIGURES/Scaling_Pyrolysis/FAA_HIPS_cone_8p1.pdf} &
\includegraphics[height=2.10in]{SCRIPT_FIGURES/Scaling_Pyrolysis/FAA_HIPS_cone_27p0.pdf} \\
\includegraphics[height=2.10in]{SCRIPT_FIGURES/Scaling_Pyrolysis/FAA_PBT_cone_4p0.pdf} &
\includegraphics[height=2.10in]{SCRIPT_FIGURES/Scaling_Pyrolysis/FAA_PBTGF_cone_4p0.pdf} \\
\end{tabular*}
\caption[HRRPUA of FAA Polymers using scaling model]
{FAA Polymers - Comparison of predicted and measured heat release rate per unit area using scaling-based approach for cone calorimeter experiments.}
\label{FAA_Polymers_HRR_Polymers1}
\end{figure}

\begin{figure}[p]
\begin{tabular*}{\textwidth}{l@{\extracolsep{\fill}}r}
\includegraphics[height=2.10in]{SCRIPT_FIGURES/Scaling_Pyrolysis/FAA_PC_cone_3p0.pdf} &
\includegraphics[height=2.10in]{SCRIPT_FIGURES/Scaling_Pyrolysis/FAA_PC_cone_5p5.pdf} \\
\includegraphics[height=2.10in]{SCRIPT_FIGURES/Scaling_Pyrolysis/FAA_PC_cone_9p0.pdf} &
\includegraphics[height=2.10in]{SCRIPT_FIGURES/Scaling_Pyrolysis/FAA_PEEK_cone_3p9.pdf} \\
\includegraphics[height=2.10in]{SCRIPT_FIGURES/Scaling_Pyrolysis/FAA_PMMA_cone_3p2.pdf} &
\includegraphics[height=2.10in]{SCRIPT_FIGURES/Scaling_Pyrolysis/FAA_PMMA_cone_8p1.pdf} \\
\includegraphics[height=2.10in]{SCRIPT_FIGURES/Scaling_Pyrolysis/FAA_PMMA_cone_27p0.pdf} &
\includegraphics[height=2.10in]{SCRIPT_FIGURES/Scaling_Pyrolysis/FAA_PVC_cone_3p0.pdf} \\
\end{tabular*}
\caption[HRRPUA of FAA Polymers using scaling model]
{FAA Polymers - Comparison of predicted and measured heat release rate per unit area using scaling-based approach for cone calorimeter experiments.}
\label{FAA_Polymers_HRR_Polymers2}
\end{figure}

\begin{figure}[p]
\begin{tabular*}{\textwidth}{l@{\extracolsep{\fill}}r}
\includegraphics[height=2.10in]{SCRIPT_FIGURES/Scaling_Pyrolysis/FAA_PVC_cone_6p0.pdf} &
\includegraphics[height=2.10in]{SCRIPT_FIGURES/Scaling_Pyrolysis/FAA_PVC_cone_9p0.pdf} \\
\end{tabular*}
\caption[HRRPUA of FAA Polymers using scaling model]
{FAA Polymers - Comparison of predicted and measured heat release rate per unit area using scaling-based approach for cone calorimeter experiments.}
\label{FAA_Polymers_HRR_Polymers3}
\end{figure}

\clearpage

\subsection{FPL Wood Experiments}\label{sec_FPL_Woods_Materials}

Table ~\ref{Properties_FPL_Materials_All} lists the relevant parameters for modeling the cone calorimeter experiments for this study.
Note that the test reports in this study did not provide thermal properties.
Fixed values of specific heat capacity, 1.0 $\mathrm{\left(kJ/(kg\cdot K\right)}$, thermal conductivity, 0.4 $\mathrm{\left(W/m\cdot K\right)}$, and emissivity, 1.0, are used to calculate an effective ignition temperature.
The results of the simulations are shown on the following pages.

\begin{table}[!h]
\caption[Properties of FPL Materials]{Properties of FPL Materials~\cite{FPL:Fire_Database}.}
\centering
\begin{tabular}{|p{4.5cm}|c|c|c|c|c|l|}
\hline
            & \centering$\Delta$& \centering$\rho$& \centering$T_{\mathrm{ign}}^{a}$&\centering$\Delta H_{c}$&\centering$Y_{s}$ & Test \# \\
Material    & \centering$\mathrm{\left(mm\right)}$ & \centering$\mathrm{\left(\frac{kg}{m^{3}}\right)}$ &  \centering($\mathrm{^{\circ}C}$)   & \centering$\left(\mathrm{\frac{MJ}{kg}}\right)$ & \centering$\mathrm{\left(\frac{g}{g}\right)}$ & $\mathrm{( - )}$  \\ \hline
\hline
Hardboard                                         & 7.2  & 972 & 410 & 14.5 & 0.007 & 670a-b, 682a-b, 697a-b, 710a-b \\\hline
Lumber Redoak                                     & 19.8 & 775 & 434 & 10.9 & 0.004 & 675a-b, 687a-b, 702a-b, 753a-b \\\hline
OSB                                               & 11.5 & 754 & 406 & 13.1 & 0.010 & 669a-b, 681a-b, 696a-b, 709a-b \\\hline
Plywood Douglas Fir                               & 11.8 & 482 & 398 & 12.6 & 0.007 & 454a-b, 456a-c, 457a-c \\\hline
Plywood Douglas Fir FRT                           & 12.5 & 604 & 361 & 7.3  & 0.002 & 543a, 544a, 545a, 546a \\\hline
Plywood Oak                                       & 12.7 & 502 & 446 & 12.4 & 0.001 & 1a-c, 2a-c, 3a-c, 4a-c \\\hline
Plywood Southern Pine FRT                         & 11.2 & 741 & 505 & 7.4  & 0.001 & 665a-b, 677a-b, 690a-b, 705a-b \\\hline
Waferboard                                        & 13.0 & 712 & 425 & 13.3 & 0.010 & 674a-b, 686a-b, 701a-b, 752a-b \\\hline
\end{tabular}
\label{Properties_FPL_Materials_All}
\end{table}
\vspace{-0.4cm}
\noindent\footnotesize{$^a$ $T_{\mathrm{ign}}$ is calculated based on measured times to ignition. Assumed values used for thermal properties when not measured.}\\

\begin{figure}[p]
\begin{tabular*}{\textwidth}{l@{\extracolsep{\fill}}r}
\includegraphics[height=2.10in]{SCRIPT_FIGURES/Scaling_Pyrolysis/FPL_hardboard_6mm_cone_7p2.pdf} &
\includegraphics[height=2.10in]{SCRIPT_FIGURES/Scaling_Pyrolysis/FPL_lumber_redoak_20mm_cone_19p8.pdf} \\
\includegraphics[height=2.10in]{SCRIPT_FIGURES/Scaling_Pyrolysis/FPL_osb_12mm_cone_11p5.pdf} &
\includegraphics[height=2.10in]{SCRIPT_FIGURES/Scaling_Pyrolysis/FPL_plywood_douglas_fir_12mm_cone_11p8.pdf} \\
\includegraphics[height=2.10in]{SCRIPT_FIGURES/Scaling_Pyrolysis/FPL_plywood_douglas_fir_frt_12mm_cone_12p5.pdf} &
\includegraphics[height=2.10in]{SCRIPT_FIGURES/Scaling_Pyrolysis/FPL_plywood_oak_13mm_cone_12p7.pdf} \\
\includegraphics[height=2.10in]{SCRIPT_FIGURES/Scaling_Pyrolysis/FPL_plywood_southern_pine_frt_11mm_cone_11p2.pdf} &
\includegraphics[height=2.10in]{SCRIPT_FIGURES/Scaling_Pyrolysis/FPL_waferboard_13mm_cone_13p0.pdf} \\
\end{tabular*}
\caption[HRRPUA of FPL Materials using scaling model]
{FPL materials - Comparison of predicted and measured heat release rate per unit area using scaling-based approach for cone calorimeter experiments.}
\label{FPL_Materials_HRR_Wood-Based}
\end{figure}

\clearpage

\subsection{FSRI/NIJ Experiments}\label{sec_FSRI_NIJ_Materials}

Table ~\ref{Properties_FSRI_Materials_Others}, Table ~\ref{Properties_FSRI_Materials_Polymers}, and Table ~\ref{Properties_FSRI_Materials_Wood-Based} lists the relevant parameters for modeling the cone calorimeter experiments for this study.
Note that the published properties from the test report are simplified in this analysis. Each multi-phase properties is represented as single bulk value.
In addition, the temperature-dependent thermal conductivity and specific capacity are fixed at an average value.
The ignition temperatures were calculated based on times to ignition at the lower heat fluxes using the other properties.

\begin{table}[!h]
\caption[Properties of FSRI Materials, other materials]{Properties of FSRI Materials, other materials ~\cite{McKinnon:FSRI2023_Data}.}
\centering
\begin{tabular}{|p{6.5cm}|c|c|c|c|c|c|c|}
\hline
            & \centering$\Delta$& \centering$\rho$& \centering$k$& \centering$c_{p}$ &\centering$T_{\mathrm{ign}}^{a}$&\centering$\Delta H_{c}$& $Y_{s}$ \\
Material    & \centering$\mathrm{\left(mm\right)}$ & \centering$\mathrm{\left(\frac{kg}{m^{3}}\right)}$ & \centering$\mathrm{\left(\frac{W}{m\cdot K}\right)}$ & \centering$\mathrm{\left(\frac{kJ}{(kg\cdot K}\right)}$ & \centering($\mathrm{^{\circ}C}$)   & \centering$\left(\mathrm{\frac{MJ}{kg}}\right)$ & $\mathrm{\left(\frac{g}{g}\right)}$ \\ \hline
\hline
Asphalt Shingle                                   & 3.0  & 1219 & 0.08 & 1.01 & 520 & 36.6 & 0.117 \\\hline
Cellulose Insulation                              & 37.7 & 68   & 0.05 & 1.74 & 592 & 11.9 & 0.001 \\\hline
Cotton Rug                                        & 6.1  & 263  & 0.07 & 1.56 & 522 & 19.4 & 0.018 \\\hline
Cotton Sheet                                      & 0.2  & 550  & 0.03 & 1.45 & 450 & 14.7 & 0.001 \\\hline
EPDM Membrane                                     & 7.8  & 1200 & 0.14 & 1.46 & 459 & 35.2 & 0.161 \\\hline
Excelsior                                         & 2.0  & 360  & 0.03 & 1.38 & 470 & 13.7 & 0.013 \\\hline
Expanded Polystyrene Board                        & 27.1 & 28   & 0.03 & 1.38 & 671 & 33.8 & 0.104 \\\hline
Face Shield$^{b}$                                 & 19.3 & 664  & 0.07 & 1.35 & 578 & 14.1 & 0.002 \\\hline
Feather Pillow Feathers$^{b}$                     & 0.8  & 671  & 0.03 & 1.49 & 456 & 8.2  & 0.007 \\\hline
Fiberglass Reinforced Panel                       & 6.4  & 524  & 0.12 & 1.66 & 514 & 27.3 & 0.069 \\\hline
Gypsum Wallboard                                  & 13.0 & 591  & 0.15 & 1.10 & 448 & 6.3  & 0.001 \\\hline
Hemp Sheet                                        & 0.4  & 523  & 0.04 & 1.79 & 410 & 17.0 & 0.001 \\\hline
House Wrap                                        & 0.1  & 714  & 0.15 & 1.57 & 484 & 1.5  & 0.052 \\\hline
Latex Pillow Foam$^{b}$                           & 3.3  & 677  & 0.13 & 1.58 & 440 & 32.8 & 0.094 \\\hline
Lightweight Gypsum Wallboard                      & 12.9 & 494  & 0.13 & 1.15 & 547 & 8.1  & 0.001 \\\hline
Low Density Fiberboard                            & 12.6 & 214  & 0.06 & 1.35 & 503 & 16.6 & 0.003 \\\hline
Overstuffed Chair Assembly                        & 4.2  & 379  & 0.10 & 1.47 & 478 & 24.9 & 0.026 \\\hline
Paper-faced R30 Fiberglass Insulation             & 2.0  & 360  & 0.03 & 1.38 & 470 & 13.7 & 0.013 \\\hline
PE Foam Pipe Insulation                           & 14.3 & 35   & 0.06 & 2.58 & 579 & 43.9 & 0.058 \\\hline
Polyisocyanurate Foam Board                       & 13.7 & 52   & 0.04 & 1.47 & 589 & 23.5 & 0.001 \\\hline
Pressure Treated Deck$^{b}$                       & 19.3 & 664  & 0.07 & 1.35 & 578 & 14.1 & 0.002 \\\hline
Roof Felt                                         & 1.1  & 726  & 0.06 & 1.74 & 394 & 28.3 & 0.067 \\\hline
Rubber Band$^{b}$                                 & 4.7  & 404  & 0.10 & 1.55 & 549 & 33.5 & 0.001 \\\hline
Rubber Foam Pipe Insulation                       & 12.8 & 52   & 0.04 & 1.47 & 587 & 19.5 & 0.061 \\\hline
Rug Pad                                           & 4.1  & 176  & 0.05 & 1.10 & 540 & 27.9 & 0.055 \\\hline
Wool Rug                                          & 16.6 & 226  & 0.06 & 1.33 & 502 & 21.0 & 0.010 \\\hline
\end{tabular}
\label{Properties_FSRI_Materials_Others}
\end{table}
\vspace{-0.4cm}
\noindent\footnotesize{$^a$ $T_{\mathrm{ign}}$ is calculated based on measured times to ignition. Assumed value for emissivity of 1 is used.}\\
\noindent\footnotesize{$^b$ Material not available in web interface. See FSRI database github (\url{https://github.com/ulfsri/fsri_materials_database/tree/main/01_Data}})\\

\begin{figure}[p]
\begin{tabular*}{\textwidth}{l@{\extracolsep{\fill}}r}
\includegraphics[height=2.10in]{SCRIPT_FIGURES/Scaling_Pyrolysis/FSRI_Asphalt_Shingle_cone_3p0.pdf} &
\includegraphics[height=2.10in]{SCRIPT_FIGURES/Scaling_Pyrolysis/FSRI_Cellulose_Insulation_cone_37p7.pdf} \\
\includegraphics[height=2.10in]{SCRIPT_FIGURES/Scaling_Pyrolysis/FSRI_Cotton_Rug_cone_6p1.pdf} &
\includegraphics[height=2.10in]{SCRIPT_FIGURES/Scaling_Pyrolysis/FSRI_Cotton_Sheet_cone_0p2.pdf} \\
\includegraphics[height=2.10in]{SCRIPT_FIGURES/Scaling_Pyrolysis/FSRI_EPDM_Membrane_cone_7p8.pdf} &
\includegraphics[height=2.10in]{SCRIPT_FIGURES/Scaling_Pyrolysis/FSRI_Excelsior_cone_2p0.pdf} \\
\includegraphics[height=2.10in]{SCRIPT_FIGURES/Scaling_Pyrolysis/FSRI_XPS_Foam_Board_cone_27p1.pdf} &
\includegraphics[height=2.10in]{SCRIPT_FIGURES/Scaling_Pyrolysis/FSRI_Face_Shield_cone_19p3.pdf} \\
\end{tabular*}
\caption[HRRPUA of FSRI Materials using scaling model, other materials]
{FSRI, other materials - Comparison of predicted and measured heat release rate per unit area using scaling-based approach for cone calorimeter experiments.}
\label{FSRI_Materials_HRR_Others1}
\end{figure}

\begin{figure}[p]
\begin{tabular*}{\textwidth}{l@{\extracolsep{\fill}}r}
\includegraphics[height=2.10in]{SCRIPT_FIGURES/Scaling_Pyrolysis/FSRI_Feather_Pillow_Feathers_cone_0p8.pdf} &
\includegraphics[height=2.10in]{SCRIPT_FIGURES/Scaling_Pyrolysis/FSRI_FRP_Panel_cone_6p4.pdf} \\
\includegraphics[height=2.10in]{SCRIPT_FIGURES/Scaling_Pyrolysis/FSRI_Gypsum_Wallboard_cone_13p0.pdf} &
\includegraphics[height=2.10in]{SCRIPT_FIGURES/Scaling_Pyrolysis/FSRI_Hemp_Sheet_cone_0p4.pdf} \\
\includegraphics[height=2.10in]{SCRIPT_FIGURES/Scaling_Pyrolysis/FSRI_House_Wrap_cone_0p1.pdf} &
\includegraphics[height=2.10in]{SCRIPT_FIGURES/Scaling_Pyrolysis/FSRI_Latex_Pillow_Foam_cone_3p3.pdf} \\
\includegraphics[height=2.10in]{SCRIPT_FIGURES/Scaling_Pyrolysis/FSRI_Lightweight_Gypsum_Wallboard_cone_12p9.pdf} &
\includegraphics[height=2.10in]{SCRIPT_FIGURES/Scaling_Pyrolysis/FSRI_FDNY_LDF_cone_12p6.pdf} \\
\end{tabular*}
\caption[HRRPUA of FSRI Materials using scaling model, others materials]
{FSRI, other materials - Comparison of predicted and measured heat release rate per unit area using scaling-based approach for cone calorimeter experiments.}
\label{FSRI_Materials_HRR_Others2}
\end{figure}

\begin{figure}[p]
\begin{tabular*}{\textwidth}{l@{\extracolsep{\fill}}r}
\includegraphics[height=2.10in]{SCRIPT_FIGURES/Scaling_Pyrolysis/FSRI_Overstuffed_Chair_Assembly_cone_4p2.pdf} &
\includegraphics[height=2.10in]{SCRIPT_FIGURES/Scaling_Pyrolysis/FSRI_Fiberglass_Insulation_R30_Paper_Faced_cone_2p0.pdf} \\
\includegraphics[height=2.10in]{SCRIPT_FIGURES/Scaling_Pyrolysis/FSRI_PE_Foam_Pipe_Insulation_cone_14p3.pdf} &
\includegraphics[height=2.10in]{SCRIPT_FIGURES/Scaling_Pyrolysis/FSRI_Polyisocyanurate_Foam_Board_cone_13p7.pdf} \\
\includegraphics[height=2.10in]{SCRIPT_FIGURES/Scaling_Pyrolysis/FSRI_Pressure_Treated_Deck_cone_19p3.pdf} &
\includegraphics[height=2.10in]{SCRIPT_FIGURES/Scaling_Pyrolysis/FSRI_Roof_Felt_cone_1p1.pdf} \\
\includegraphics[height=2.10in]{SCRIPT_FIGURES/Scaling_Pyrolysis/FSRI_Rubber_Band_cone_4p7.pdf} &
\includegraphics[height=2.10in]{SCRIPT_FIGURES/Scaling_Pyrolysis/FSRI_Rubber_Foam_Pipe_Insulation_cone_12p8.pdf} \\
\end{tabular*}
\caption[HRRPUA of FSRI Materials using scaling model, others materials]
{FSRI, other materials - Comparison of predicted and measured heat release rate per unit area using scaling-based approach for cone calorimeter experiments.}
\label{FSRI_Materials_HRR_Others3}
\end{figure}

\begin{figure}[p]
\begin{tabular*}{\textwidth}{l@{\extracolsep{\fill}}r}
\includegraphics[height=2.10in]{SCRIPT_FIGURES/Scaling_Pyrolysis/FSRI_Rug_Pad_cone_4p1.pdf} &
\includegraphics[height=2.10in]{SCRIPT_FIGURES/Scaling_Pyrolysis/FSRI_Wool_Rug_cone_16p6.pdf} \\

\end{tabular*}
\caption[HRRPUA of FSRI Materials using scaling model, others materials]
{FSRI, other materials - Comparison of predicted and measured heat release rate per unit area using scaling-based approach for cone calorimeter experiments.}
\label{FSRI_Materials_HRR_Others4}
\end{figure}

\clearpage

\begin{table}[!h]
\caption[Properties of FSRI Materials, polymer materials]{Properties of FSRI Materials, polymer materials ~\cite{McKinnon:FSRI2023_Data}.}
\centering
\begin{tabular}{|p{6.5cm}|c|c|c|c|c|c|c|}
\hline
            & \centering$\Delta$& \centering$\rho$& \centering$k$& \centering$c_{p}$ &\centering$T_{\mathrm{ign}}^{a}$&\centering$\Delta H_{c}$& $Y_{s}$ \\
Material    & \centering$\mathrm{\left(mm\right)}$ & \centering$\mathrm{\left(\frac{kg}{m^{3}}\right)}$ & \centering$\mathrm{\left(\frac{W}{m\cdot K}\right)}$ & \centering$\mathrm{\left(\frac{kJ}{(kg\cdot K}\right)}$ & \centering($\mathrm{^{\circ}C}$)   & \centering$\left(\mathrm{\frac{MJ}{kg}}\right)$ & $\mathrm{\left(\frac{g}{g}\right)}$ \\ \hline
\hline
ABS                                               & 3.0  & 1100 & 0.39 & 1.35 & 388 & 35.4 & 0.102 \\\hline
Black PMMA                                        & 8.7  & 1182 & 0.15 & 1.48 & 436 & 29.5 & 0.006 \\\hline
Cotton Raw$^{b}$                                  & 2.4  & 664  & 0.07 & 1.35 & 388 & 17.3 & 0.088 \\\hline
HDPE                                              & 3.2  & 971  & 0.30 & 1.92 & 481 & 49.1 & 0.026 \\\hline
HIPS                                              & 3.0  & 1067 & 0.13 & 1.35 & 499 & 35.9 & 0.111 \\\hline
High Temperature Scba Facepiece$^{b}$             & 37.2 & 737  & 0.12 & 1.28 & 527 & 14.2 & 0.001 \\\hline
LDPE                                              & 3.2  & 927  & 0.22 & 2.60 & 420 & 48.9 & 0.026 \\\hline
Memory Foam Carpet Pad                            & 12.4 & 134  & 0.04 & 1.87 & 486 & 30.6 & 0.024 \\\hline
Nylon                                             & 3.4  & 1070 & 0.15 & 1.84 & 533 & 36.3 & 0.020 \\\hline
Nylon Carpet High Pile$^{b}$                      & 13.7 & 188  & 0.06 & 1.53 & 615 & 36.0 & 0.049 \\\hline
Overstuffed Chair Polyester - Batting             & 1.5  & 404  & 0.10 & 1.55 & 596 & 17.6 & 0.051 \\\hline
Overstuffed Chair - Polyester Fabric              & 0.5  & 360  & 0.03 & 1.38 & 571 & 19.6 & 0.054 \\\hline
Overstuffed Chair - Polyurethane Foam             & 0.8  & 927  & 0.22 & 2.60 & 173 & 29.2 & 0.014 \\\hline
PC                                                & 5.3  & 1220 & 0.18 & 1.26 & 451 & 26.2 & 0.090 \\\hline
PET                                               & 6.5  & 1401 & 0.19 & 1.17 & 527 & 19.9 & 0.049 \\\hline
PETG                                              & 2.6  & 1314 & 0.16 & 1.25 & 529 & 23.8 & 0.040 \\\hline
PMMA                                              & 2.8  & 1182 & 0.15 & 1.48 & 440 & 26.8 & 0.014 \\\hline
PP                                                & 3.2  & 887  & 0.15 & 1.92 & 460 & 48.6 & 0.040 \\\hline
PVC                                               & 3.2  & 1388 & 0.14 & 1.08 & 506 & 11.8 & 0.068 \\\hline
PlasticC$^{b}$                                    & 3.0  & 1058 & 0.12 & 1.37 & 499 & 35.3 & 0.121 \\\hline
Plastic Laminate Countertop                       & 37.2 & 737  & 0.12 & 1.28 & 527 & 14.2 & 0.001 \\\hline
Polyester Bed Skirt                               & 1.2  & 432  & 0.02 & 1.61 & 614 & 19.6 & 0.040 \\\hline
Polyester Microfiber Sheet                        & 1.0  & 457  & 0.02 & 2.08 & 609 & 20.5 & 0.036 \\\hline
Low Pile Polyolefin Carpet                        & 7.2  & 214  & 0.06 & 1.35 & 501 & 40.3 & 0.074 \\\hline
Rebond Foam Carpet Pad                            & 9.0  & 86   & 0.04 & 1.82 & 520 & 33.3 & 0.025 \\\hline
Vinyl Plank Flooring                              & 2.4  & 664  & 0.07 & 1.35 & 388 & 17.3 & 0.088 \\\hline
Vinyl Siding                                      & 1.2  & 1377 & 0.10 & 1.10 & 534 & 10.6 & 0.099 \\\hline
Vinyl Tile                                        & 7.9  & 726  & 0.06 & 1.74 & 551 & 16.1 & 0.061 \\\hline
\end{tabular}
\label{Properties_FSRI_Materials_Polymers}
\end{table}
\vspace{-0.4cm}
\noindent\footnotesize{$^a$ $T_{\mathrm{ign}}$ is calculated based on measured times to ignition. Assumed value for emissivity of 1 is used.}\\
\noindent\footnotesize{$^b$ Material not available in web interface. See FSRI database github (\url{https://github.com/ulfsri/fsri_materials_database/tree/main/01_Data}})\\

\begin{figure}[p]
\begin{tabular*}{\textwidth}{l@{\extracolsep{\fill}}r}
\includegraphics[height=2.10in]{SCRIPT_FIGURES/Scaling_Pyrolysis/FSRI_ABS_cone_3p0.pdf} &
\includegraphics[height=2.10in]{SCRIPT_FIGURES/Scaling_Pyrolysis/FSRI_Black_PMMA_cone_8p7.pdf} \\
\includegraphics[height=2.10in]{SCRIPT_FIGURES/Scaling_Pyrolysis/FSRI_Cotton_Raw_cone_2p4.pdf} &
\includegraphics[height=2.10in]{SCRIPT_FIGURES/Scaling_Pyrolysis/FSRI_HDPE_cone_3p2.pdf} \\
\includegraphics[height=2.10in]{SCRIPT_FIGURES/Scaling_Pyrolysis/FSRI_HIPS_cone_3p0.pdf} &
\includegraphics[height=2.10in]{SCRIPT_FIGURES/Scaling_Pyrolysis/FSRI_High_Temperature_SCBA_Facepiece_cone_37p2.pdf} \\
\includegraphics[height=2.10in]{SCRIPT_FIGURES/Scaling_Pyrolysis/FSRI_LDPE_cone_3p2.pdf} &
\includegraphics[height=2.10in]{SCRIPT_FIGURES/Scaling_Pyrolysis/FSRI_Memory_Foam_Carpet_Pad_cone_12p4.pdf} \\
\end{tabular*}
\caption[HRRPUA of FSRI Materials using scaling model, polymer materials]
{FSRI, polymer materials - Comparison of predicted and measured heat release rate per unit area using scaling-based approach for cone calorimeter experiments.}
\label{FSRI_Materials_HRR_Polymers1}
\end{figure}

\begin{figure}[p]
\begin{tabular*}{\textwidth}{l@{\extracolsep{\fill}}r}
\includegraphics[height=2.10in]{SCRIPT_FIGURES/Scaling_Pyrolysis/FSRI_Nylon_cone_3p4.pdf} &
\includegraphics[height=2.10in]{SCRIPT_FIGURES/Scaling_Pyrolysis/FSRI_Nylon_Carpet_High_Pile_cone_13p7.pdf} \\
\includegraphics[height=2.10in]{SCRIPT_FIGURES/Scaling_Pyrolysis/FSRI_Overstuffed_Chair_Polyester_Batting_cone_1p5.pdf} &
\includegraphics[height=2.10in]{SCRIPT_FIGURES/Scaling_Pyrolysis/FSRI_Overstuffed_Chair_Polyester_Fabric_cone_0p5.pdf} \\
\includegraphics[height=2.10in]{SCRIPT_FIGURES/Scaling_Pyrolysis/FSRI_Overstuffed_Chair_Polyurethane_Foam_cone_0p8.pdf} &
\includegraphics[height=2.10in]{SCRIPT_FIGURES/Scaling_Pyrolysis/FSRI_PC_cone_5p3.pdf} \\
\includegraphics[height=2.10in]{SCRIPT_FIGURES/Scaling_Pyrolysis/FSRI_PET_cone_6p5.pdf} &
\includegraphics[height=2.10in]{SCRIPT_FIGURES/Scaling_Pyrolysis/FSRI_PETG_cone_2p6.pdf} \\
\end{tabular*}
\caption[HRRPUA of FSRI Materials using scaling model, polymer materials]
{FSRI, polymer materials - Comparison of predicted and measured heat release rate per unit area using scaling-based approach for cone calorimeter experiments.}
\label{FSRI_Materials_HRR_Polymers2}
\end{figure}

\begin{figure}[p]
\begin{tabular*}{\textwidth}{l@{\extracolsep{\fill}}r}
\includegraphics[height=2.10in]{SCRIPT_FIGURES/Scaling_Pyrolysis/FSRI_PMMA_cone_2p8.pdf} &
\includegraphics[height=2.10in]{SCRIPT_FIGURES/Scaling_Pyrolysis/FSRI_PP_cone_3p2.pdf} \\
\includegraphics[height=2.10in]{SCRIPT_FIGURES/Scaling_Pyrolysis/FSRI_PVC_cone_3p2.pdf} &
\includegraphics[height=2.10in]{SCRIPT_FIGURES/Scaling_Pyrolysis/FSRI_PlasticC_cone_3p0.pdf} \\
\includegraphics[height=2.10in]{SCRIPT_FIGURES/Scaling_Pyrolysis/FSRI_Plastic_Laminate_Countertop_cone_37p2.pdf} &
\includegraphics[height=2.10in]{SCRIPT_FIGURES/Scaling_Pyrolysis/FSRI_Polyester_Bed_Skirt_cone_1p2.pdf} \\
\includegraphics[height=2.10in]{SCRIPT_FIGURES/Scaling_Pyrolysis/FSRI_Polyester_Microfiber_Sheet_cone_1p0.pdf} &
\includegraphics[height=2.10in]{SCRIPT_FIGURES/Scaling_Pyrolysis/FSRI_Polyolefin_Carpet_Low_Pile_cone_7p2.pdf} \\
\end{tabular*}
\caption[HRRPUA of FSRI Materials using scaling model, polymer materials]
{FSRI, polymer materials - Comparison of predicted and measured heat release rate per unit area using scaling-based approach for cone calorimeter experiments.}
\label{FSRI_Materials_HRR_Polymers3}
\end{figure}

\begin{figure}[p]
\begin{tabular*}{\textwidth}{l@{\extracolsep{\fill}}r}
\includegraphics[height=2.10in]{SCRIPT_FIGURES/Scaling_Pyrolysis/FSRI_Rebond_Foam_Carpet_Pad_cone_9p0.pdf} &
\includegraphics[height=2.10in]{SCRIPT_FIGURES/Scaling_Pyrolysis/FSRI_Vinyl_Plank_Flooring_cone_2p4.pdf} \\
\includegraphics[height=2.10in]{SCRIPT_FIGURES/Scaling_Pyrolysis/FSRI_Vinyl_Siding_cone_1p2.pdf} &
\includegraphics[height=2.10in]{SCRIPT_FIGURES/Scaling_Pyrolysis/FSRI_Vinyl_Tile_cone_7p9.pdf} \\
\end{tabular*}
\caption[HRRPUA of FSRI Materials using scaling model, polymer materials]
{FSRI, polymer materials - Comparison of predicted and measured heat release rate per unit area using scaling-based approach for cone calorimeter experiments.}
\label{FSRI_Materials_HRR_Polymers4}
\end{figure}

\clearpage

\begin{table}[!h]
\caption[Properties of FSRI Materials, Wood-Based materials]{Properties of FSRI Materials, Wood-Based materials ~\cite{McKinnon:FSRI2023_Data}.}
\centering
\begin{tabular}{|p{4.5cm}|c|c|c|c|c|c|c|}
\hline
            & \centering$\Delta$& \centering$\rho$& \centering$k$& \centering$c_{p}$ &\centering$T_{\mathrm{ign}}^{a}$&\centering$\Delta H_{c}$& $Y_{s}$ \\
Material    & \centering$\mathrm{\left(mm\right)}$ & \centering$\mathrm{\left(\frac{kg}{m^{3}}\right)}$ & \centering$\mathrm{\left(\frac{W}{m\cdot K}\right)}$ & \centering$\mathrm{\left(\frac{kJ}{(kg\cdot K}\right)}$ & \centering($\mathrm{^{\circ}C}$)   & \centering$\left(\mathrm{\frac{MJ}{kg}}\right)$ & $\mathrm{\left(\frac{g}{g}\right)}$ \\ \hline
\hline
Basswood Panel                                    & 19.8 & 404  & 0.10 & 1.55 & 551 & 15.7 & 0.005 \\\hline
Composite Deck Board                              & 13.3 & 1103 & 0.23 & 1.57 & 403 & 28.8 & 0.001 \\\hline
Engineered Flooring                               & 9.0  & 815  & 0.13 & 1.60 & 482 & 16.8 & 0.001 \\\hline
Engineered Wood Furniture                         & 12.1 & 927  & 0.22 & 2.60 & 364 & 17.1 & 0.003 \\\hline
Engineered Wood Table$^{b}$                       & 89.8 & 379  & 0.10 & 1.47 & 567 & 12.6 & 0.002 \\\hline
Eucalyptus Flooring                               & 15.6 & 1025 & 0.18 & 1.67 & 444 & 14.7 & 0.001 \\\hline
Homasote                                          & 13.3 & 448  & 0.08 & 1.55 & 544 & 15.6 & 0.003 \\\hline
Luan Panel                                        & 5.9  & 345  & 0.07 & 1.58 & 504 & 15.0 & 0.007 \\\hline
MDF                                               & 19.4 & 737  & 0.12 & 1.28 & 518 & 14.4 & 0.003 \\\hline
Masonite Board$^{b}$                              & 3.1  & 1127 & 0.12 & 1.52 & 515 & 16.1 & 0.007 \\\hline
OSB                                               & 16.9 & 598  & 0.12 & 1.58 & 522 & 15.5 & 0.006 \\\hline
Oak Flooring                                      & 19.9 & 714  & 0.15 & 1.57 & 493 & 14.2 & 0.003 \\\hline
Pallet Wood                                       & 3.0  & 1067 & 0.13 & 1.35 & 499 & 35.9 & 0.111 \\\hline
Particleboard                                     & 20.3 & 677  & 0.13 & 1.58 & 536 & 13.1 & 0.001 \\\hline
Pine Siding                                       & 18.9 & 328  & 0.11 & 1.84 & 466 & 16.9 & 0.001 \\\hline
Plywood                                           & 8.7  & 1182 & 0.15 & 1.48 & 436 & 29.5 & 0.006 \\\hline
Wood Stud                                         & 46.2 & 379  & 0.10 & 1.47 & 488 & 15.0 & 0.001 \\\hline
\end{tabular}
\label{Properties_FSRI_Materials_Wood-Based}
\end{table}
\vspace{-0.4cm}
\noindent\footnotesize{$^a$ $T_{\mathrm{ign}}$ is calculated based on measured times to ignition. Assumed value for emissivity of 1 is used.}\\
\noindent\footnotesize{$^b$ Material not available in web interface. See FSRI database github (\url{https://github.com/ulfsri/fsri_materials_database/tree/main/01_Data}})\\


\begin{figure}[p]
\begin{tabular*}{\textwidth}{l@{\extracolsep{\fill}}r}
\includegraphics[height=2.10in]{SCRIPT_FIGURES/Scaling_Pyrolysis/FSRI_Basswood_Panel_cone_19p8.pdf} &
\includegraphics[height=2.10in]{SCRIPT_FIGURES/Scaling_Pyrolysis/FSRI_Composite_Deck_Board_cone_13p3.pdf} \\
\includegraphics[height=2.10in]{SCRIPT_FIGURES/Scaling_Pyrolysis/FSRI_Engineered_Flooring_cone_9p0.pdf} &
\includegraphics[height=2.10in]{SCRIPT_FIGURES/Scaling_Pyrolysis/FSRI_Engineered_Wood_Furniture_cone_12p1.pdf} \\
\includegraphics[height=2.10in]{SCRIPT_FIGURES/Scaling_Pyrolysis/FSRI_Engineered_Wood_Table_cone_89p8.pdf} &
\includegraphics[height=2.10in]{SCRIPT_FIGURES/Scaling_Pyrolysis/FSRI_Eucalyptus_Flooring_cone_15p6.pdf} \\
\includegraphics[height=2.10in]{SCRIPT_FIGURES/Scaling_Pyrolysis/FSRI_Homasote_cone_13p3.pdf} &
\includegraphics[height=2.10in]{SCRIPT_FIGURES/Scaling_Pyrolysis/FSRI_Luan_Panel_cone_5p9.pdf} \\
\end{tabular*}
\caption[HRRPUA of FSRI Materials using scaling model, wood-based materials]
{FSRI, wood-based materials - Comparison of predicted and measured heat release rate per unit area using scaling-based approach for cone calorimeter experiments.}
\label{FSRI_Materials_HRR_Wood-Based1}
\end{figure}

\begin{figure}[p]
\begin{tabular*}{\textwidth}{l@{\extracolsep{\fill}}r}
\includegraphics[height=2.10in]{SCRIPT_FIGURES/Scaling_Pyrolysis/FSRI_MDF_cone_19p4.pdf} &
\includegraphics[height=2.10in]{SCRIPT_FIGURES/Scaling_Pyrolysis/FSRI_Masonite_Board_cone_3p1.pdf} \\
\includegraphics[height=2.10in]{SCRIPT_FIGURES/Scaling_Pyrolysis/FSRI_OSB_cone_16p9.pdf} &
\includegraphics[height=2.10in]{SCRIPT_FIGURES/Scaling_Pyrolysis/FSRI_Oak_Flooring_cone_19p9.pdf} \\
\includegraphics[height=2.10in]{SCRIPT_FIGURES/Scaling_Pyrolysis/FSRI_Pallet_Wood_cone_3p0.pdf} &
\includegraphics[height=2.10in]{SCRIPT_FIGURES/Scaling_Pyrolysis/FSRI_Particleboard_cone_20p3.pdf} \\
\includegraphics[height=2.10in]{SCRIPT_FIGURES/Scaling_Pyrolysis/FSRI_Pine_Siding_cone_18p9.pdf} &
\includegraphics[height=2.10in]{SCRIPT_FIGURES/Scaling_Pyrolysis/FSRI_Plywood_cone_8p7.pdf} \\
\end{tabular*}
\caption[HRRPUA of FSRI Materials using scaling model, wood-based materials]
{FSRI, wood-based materials - Comparison of predicted and measured heat release rate per unit area using scaling-based approach for cone calorimeter experiments.}
\label{FSRI_Materials_HRR_Wood-Based2}
\end{figure}

\begin{figure}[p]
\begin{tabular*}{\textwidth}{l@{\extracolsep{\fill}}r}
\includegraphics[height=2.10in]{SCRIPT_FIGURES/Scaling_Pyrolysis/FSRI_Wood_Stud_cone_46p2.pdf} & \\
\end{tabular*}
\caption[HRRPUA of FSRI Materials using scaling model, wood-Based materials]
{FSRI, wood-based materials - Comparison of predicted and measured heat release rate per unit area using scaling-based approach for cone calorimeter experiments.}
\label{FSRI_Materials_HRR_Wood-Based3}
\end{figure}


\clearpage

\subsection{JH Experiments}\label{sec_JH_Materials}

Table ~\ref{Properties_JH_Materials_All} lists the relevant parameters for modeling the cone calorimeter experiments for this study.
Note that the published properties from the test report are simplified in this analysis. Each multi-phase properties is represented as single bulk value.
In addition, the temperature-dependent thermal conductivity and specific capacity are fixed at an average value.
The ignition temperatures were calculated based on times to ignition at the lower heat fluxes using the other properties.
The results of the simulations are shown on the following pages.

\begin{table}[!h]
\caption[Properties of JH Materials]{Properties of JH Materials~\cite{Luo:FRA2019,Lattimer:NIJ19}.}
\centering
\begin{tabular}{|l|c|c|c|c|c|c|c|c|}
\hline
            & \centering$\Delta$& \centering$\rho$& \centering$k$& \centering$c_{p}$& \centering$\varepsilon$& \centering$T_{\mathrm{ign}}^{a}$&\centering$\Delta H_{c}$& $Y_{s}$ \\
Material    & \centering$\mathrm{\left(mm\right)}$ & \centering$\mathrm{\left(\frac{kg}{m^{3}}\right)}$ & \centering$\mathrm{\left(\frac{W}{m\cdot K}\right)}$ & \centering$\mathrm{\left(\frac{kJ}{(kg\cdot K}\right)}$ & \centering$\mathrm{( - )}$ &  \centering($\mathrm{^{\circ}C}$)   & \centering$\left(\mathrm{\frac{MJ}{kg}}\right)$ & $\mathrm{\left(\frac{g}{g}\right)}$ \\ \hline
\hline
Acrylic                                           & 4.5  & 1178 & 0.24 & 0.62 & 0.88 & 387 & 24.9 & 0.033 \\\hline
Black PMMA                                        & 9.2  & 1154 & 0.12 & 1.52 & 0.95 & 374 & 23.3 & 0.006 \\\hline
CPS Balsa Facesheet                               & 15.9 & 630  & 0.39 & 0.62 & 0.70 & 437 & 11.4 & 0.041 \\\hline
CPS Plywood Facesheet                             & 12.7 & 630  & 0.39 & 0.51 & 0.70 & 544 & 10.2 & 0.020 \\\hline
Cardboard                                         & 4.1  & 128  & 0.23 & 0.69 & 0.93 & 462 & 12.0 & 0.001 \\\hline
FRP                                               & 12.7 & 1897 & 0.39 & 1.41 & 0.95 & 414 & 20.0 & 0.050 \\\hline
MDF                                               & 19.2 & 688  & 0.31 & 0.40 & 0.99 & 524 & 12.8 & 0.003 \\\hline
OSB                                               & 16.1 & 573  & 0.38 & 0.63 & 0.93 & 385 & 12.8 & 0.011 \\\hline
PC Blend                                          & 4.5  & 1320 & 0.89 & 0.67 & 0.93 & 448 & 24.5 & 0.233 \\\hline
PVC Blend                                         & 3.3  & 1314 & 0.29 & 0.42 & 0.88 & 534 & 21.4 & 0.151 \\\hline
Phenolic Resin Fiberglass                         & 3.3  & 1846 & 0.35 & 1.29 & 0.89 & 511 & 23.9 & 0.238 \\\hline
Plywood                                           & 6.3  & 625  & 0.29 & 3.13 & 1.00 & 266 & 21.6 & 0.006 \\\hline
Vinyl Ester Resin FRP                             & 4.5  & 1600 & 0.54 & 4.00 & 0.80 & 440 & 19.4 & 0.180 \\\hline
White Pine                                        & 19.1 & 499  & 0.31 & 0.60 & 0.90 & 421 & 12.0 & 0.015 \\\hline
White Spruce                                      & 37.2 & 430  & 0.24 & 1.10 & 0.90 & 399 & 12.0 & 0.015 \\\hline
\end{tabular}
\label{Properties_JH_Materials_All}
\end{table}
\vspace{-0.4cm}
\noindent\footnotesize{$^a$ $T_{\mathrm{ign}}$ is calculated based on measured times to ignition.}\\


\begin{figure}[p]
\begin{tabular*}{\textwidth}{l@{\extracolsep{\fill}}r}
\includegraphics[height=2.10in]{SCRIPT_FIGURES/Scaling_Pyrolysis/JH_Acrylic_cone_4p5.pdf} &
\includegraphics[height=2.10in]{SCRIPT_FIGURES/Scaling_Pyrolysis/JH_Black_PMMA_cone_9p2.pdf} \\
\includegraphics[height=2.10in]{SCRIPT_FIGURES/Scaling_Pyrolysis/JH_Cardboard_cone_4p1.pdf} &
\includegraphics[height=2.10in]{SCRIPT_FIGURES/Scaling_Pyrolysis/JH_CPS_Balsa_Facesheet_cone_15p9.pdf} \\
\includegraphics[height=2.10in]{SCRIPT_FIGURES/Scaling_Pyrolysis/JH_CPS_Plywood_Facesheet_cone_12p7.pdf} &
\includegraphics[height=2.10in]{SCRIPT_FIGURES/Scaling_Pyrolysis/JH_FRP_cone_12p7.pdf} \\
\includegraphics[height=2.10in]{SCRIPT_FIGURES/Scaling_Pyrolysis/JH_MDF_cone_19p2.pdf} &
\includegraphics[height=2.10in]{SCRIPT_FIGURES/Scaling_Pyrolysis/JH_OSB_cone_16p1.pdf} \\
\end{tabular*}
\caption[HRRPUA of JH Materials using scaling model]
{JH materials - Comparison of predicted and measured heat release rate per unit area using scaling-based approach for cone calorimeter experiments.}
\label{JH_Materials_HRR}
\end{figure}



\begin{figure}[p]
\begin{tabular*}{\textwidth}{l@{\extracolsep{\fill}}r}
\includegraphics[height=2.10in]{SCRIPT_FIGURES/Scaling_Pyrolysis/JH_PC_Blend_cone_4p5.pdf} &
\includegraphics[height=2.10in]{SCRIPT_FIGURES/Scaling_Pyrolysis/JH_Phenolic_Resin_Fiberglass_Composite_cone_3p3.pdf} \\
\includegraphics[height=2.10in]{SCRIPT_FIGURES/Scaling_Pyrolysis/JH_Plywood_cone_6p3.pdf} &
\includegraphics[height=2.10in]{SCRIPT_FIGURES/Scaling_Pyrolysis/JH_PVC_Blend_cone_3p3.pdf} \\
\includegraphics[height=2.10in]{SCRIPT_FIGURES/Scaling_Pyrolysis/JH_Vinyl_Ester_Resin_FRP_cone_4p5.pdf} &
\includegraphics[height=2.10in]{SCRIPT_FIGURES/Scaling_Pyrolysis/JH_White_Pine_cone_19p1.pdf} \\
\includegraphics[height=2.10in]{SCRIPT_FIGURES/Scaling_Pyrolysis/JH_White_Spruce_cone_37p2.pdf} &
\end{tabular*}
\caption[HRRPUA of JH Materials using scaling model]
{JH materials - Comparison of predicted and measured heat release rate per unit area using scaling-based approach for cone calorimeter experiments.}
\label{JH_Materials_HRR2}
\end{figure}

\clearpage







\subsection{RISE Experiments}\label{sec_RISE_Materials}

Table ~\ref{Properties_RISE_Materials_Mixtures1}, Table ~\ref{Properties_RISE_Materials_Mixtures2}, Table ~\ref{Properties_RISE_Materials_Others}, Table ~\ref{Properties_RISE_Materials_Polymers}, and Table ~\ref{Properties_RISE_Materials_Wood-Based} lists the relevant parameters for modeling the cone calorimeter experiments for this study.
Note that the test reports in this study did not provide thermal properties.
Fixed values of specific heat capacity, 1.0 $\mathrm{\left(kJ/(kg\cdot K\right)}$, thermal conductivity, 0.4 $\mathrm{\left(W/m\cdot K\right)}$, and emissivity, 1.0, are used to calculate an effective ignition temperature.
The results of the simulations are shown on the following pages.

\begin{table}[!h]
\caption[Properties of RISE Materials, mixture materials]{Properties of RISE Materials, mixture materials ~\cite{RISE:Fire_Database}.}
\centering
\begin{tabular}{|l|c|c|c|c|c|p{3.7cm}|}
\hline
            & \centering$\Delta$& \centering$\rho$& \centering$T_{\mathrm{ign}}^{a}$&\centering$\Delta H_{c}$&\centering$Y_{s}^{b}$ & Test \# \\
Material$^{c}$    & \centering$\mathrm{\left(mm\right)}$ & \centering$\mathrm{\left(\frac{kg}{m^{3}}\right)}$ &  \centering($\mathrm{^{\circ}C}$)   & \centering$\left(\mathrm{\frac{MJ}{kg}}\right)$ & \centering$\mathrm{\left(\frac{g}{g}\right)}$ & $\mathrm{( - )}$  \\ \hline
\hline
80 Wool 20 Nylon, Glue, Plywood                   & 22.0 & 576  & 275 & 14.3 & 0.050 & 1821-1829, 2206-2208 \\\hline
Carpet, Glue, Aluminum Plate                      & 5.0  & 1293 & 300 & 17.7 & 0.050 & 1673-1678, 2178-2180 \\\hline
Carpet, Glue, Recor Sealing                       & 10.0 & 222  & 436 & 22.2 & 0.050 & 1679, 1681, 1682, 1687-1692, 2181-2183, 2353-2355 \\\hline
Fabric Vandalize Protected, Foam                  & 42.0 & 88   & 366 & 13.0 & 0.050 & 2385-2390 \\\hline
Fabric, Foam                                      & 28.0 & 164  & 514 & 19.2 & 0.050 & 1658, 1659, 1661, 1662, 1664-1666, 1668, 1669, 2222-2224, 2409-2411 \\\hline
Fabric, Protection Layer, Foam                    & 32.0 & 124  & 348 & 14.8 & 0.050 & 1744-1749, 1751, 1753, 1754, 2216-2218, 2392-2395, 2401, 2402 \\\hline
FR EPS, Calcium Silicate Board                    & 25.0 & 37   & 547 & 25.4 & 0.190 & 137-145 \\\hline
HPL Melamine, Polyester Film                      & 13.4 & 1638 & 195 & 3.2  & 0.050 & 1716-1718, 1720-1725, 2343-2345 \\\hline
Melamine, Calcium Silicate                        & 12.5 & 1055 & 572 & 8.5  & 0.003 & 133-136 \\\hline
PE, XLPE                                          & 40.0 & 372  & 590 & 25.8 & 0.050 & 493-495 \\\hline
Polyolefin, EPR                                   & 18.1 & 278  & 547 & 19.9 & 0.050 & 414-416 \\\hline
Polyolefin, EPR                                   & 32.2 & 278  & 547 & 19.9 & 0.050 & 407, 409, 410 \\\hline
Polyolefin, PA                                    & 2.5  & 632  & 549 & 22.3 & 0.050 & 516-518 \\\hline
Polyolefin, PA                                    & 6.0  & 632  & 476 & 22.3 & 0.050 & 513-515 \\\hline
Polyolefin, PP                                    & 8.7  & 513  & 493 & 25.7 & 0.050 & 446-448 \\\hline
Polyolefin, XLPE                                  & 18.1 & 392 & 556 & 28.4 & 0.050 & 473-475 \\\hline
Polyolefin, XLPE                                  & 25.0 & 392 & 553 & 28.4 & 0.050 & 484-486 \\\hline
Polyolefin, XLPE                                  & 38.1 & 392 & 552 & 28.4 & 0.050 & 455-457 \\\hline
Polyolefin, XLPE                                  & 45.0 & 392 & 552 & 28.4 & 0.050 & 464-466 \\\hline
Pur Rigid, Plastic Faced Steel Sheet              & 79.0 & 170 & 497 & 10.2 & 0.072 & 80-85 \\\hline
\end{tabular}
\label{Properties_RISE_Materials_Mixtures1}
\end{table}
\vspace{-0.4cm}
\noindent\footnotesize{$^a$ $T_{\mathrm{ign}}$ is calculated based on measured times to ignition. Assumed values used for thermal properties when not measured.}\\
\noindent\footnotesize{$^b$ $Y_{s}$ fixed at 0.05 when measurements were not available.}\\
\noindent\footnotesize{$^c$ Comma separation indicates layers of material.}\\

\clearpage

\begin{table}[!ht]
\caption[Properties of RISE Materials, mixture materials]{Properties of RISE Materials, mixture materials ~\cite{RISE:Fire_Database}.}
\centering
\begin{tabular}{|l|c|c|c|c|c|p{3.7cm}|}
\hline
            & \centering$\Delta$& \centering$\rho$& \centering$T_{\mathrm{ign}}^{a}$&\centering$\Delta H_{c}$&\centering$Y_{s}^{b}$ & Test \# \\
Material$^{c}$    & \centering$\mathrm{\left(mm\right)}$ & \centering$\mathrm{\left(\frac{kg}{m^{3}}\right)}$ &  \centering($\mathrm{^{\circ}C}$)   & \centering$\left(\mathrm{\frac{MJ}{kg}}\right)$ & \centering$\mathrm{\left(\frac{g}{g}\right)}$ & $\mathrm{( - )}$  \\ \hline
\hline
PVC, EPR                                          & 32.5 & 170 & 508 & 13.9 & 0.050 & 411-413 \\\hline
PVC, XLPE                                         & 17.7 & 405 & 341 & 22.4 & 0.050 & 467-469 \\\hline
PVC, XLPE                                         & 22.0 & 405 & 341 & 22.4 & 0.050 & 477-479 \\\hline
PVC, XLPE                                         & 35.0 & 405 & 341 & 22.4 & 0.050 & 490-492 \\\hline
PVC, XLPE                                         & 38.6 & 405 & 342 & 22.4 & 0.050 & 449-451 \\\hline
PVC, XLPE                                         & 46.0 & 405 & 341 & 22.4 & 0.050 & 458-460 \\\hline
RPPVC, EPR                                        & 16.5 & 618 & 389 & 11.5 & 0.050 & 417-419 \\\hline
RPPVC, PEF                                        & 4.5 & 531 & 404 & 5.1 & 0.050 & 487-489 \\\hline
RPPVC, PVC                                        & 14.0 & 478 & 413 & 13.8 & 0.050 & 528-530 \\\hline
RPPVC, XLPE                                       & 17.7 & 429 & 463 & 19.0 & 0.050 & 470-472 \\\hline
RPPVC, XLPE                                       & 22.5 & 429 & 461 & 19.0 & 0.050 & 480-482 \\\hline
RPPVC, XLPE                                       & 39.3 & 429 & 462 & 19.0 & 0.050 & 452-454 \\\hline
RPPVC, XLPE                                       & 45.0 & 429 & 462 & 19.0 & 0.050 & 461-463 \\\hline
Synthetic Rubber, Glue, Plywood                   & 15.0 & 1096 & 435 & 15.0 & 0.050 & 1806, 1808-1811, 1813-1820, 1830-1838, 2202-2204, 2209-2211, 2380-2382 \\\hline
Zhpolyolefin, PP                                  & 8.4 & 509 & 561 & 29.0 & 0.050 & 608-610 \\\hline
Zhpolyolefin, XLPE                                & 13.0 & 412 & 528 & 27.3 & 0.050 & 499-501 \\\hline
Zhpolyolefin, XLPE                                & 27.0 & 412 & 527 & 27.3 & 0.050 & 496-498 \\\hline
\end{tabular}
\label{Properties_RISE_Materials_Mixtures2}
\end{table}
\vspace{-0.4cm}
\noindent\footnotesize{$^a$ $T_{\mathrm{ign}}$ is calculated based on measured times to ignition. Assumed values used for thermal properties when not measured.}\\
\noindent\footnotesize{$^b$ $Y_{s}$ fixed at 0.05 when measurements were not available.}\\
\noindent\footnotesize{$^c$ Comma separation indicates layers of material.}\\



\begin{figure}[p]
\begin{tabular*}{\textwidth}{l@{\extracolsep{\fill}}r}
\includegraphics[height=2.10in]{SCRIPT_FIGURES/Scaling_Pyrolysis/RISE_80_wool_20_nylon-glue-plywood-22_cone_22p0.pdf} &
\includegraphics[height=2.10in]{SCRIPT_FIGURES/Scaling_Pyrolysis/RISE_carpet-glue-aluminum_plate_2_mm-5_cone_5p0.pdf} \\
\includegraphics[height=2.10in]{SCRIPT_FIGURES/Scaling_Pyrolysis/RISE_needle_punched_carpet-glue-recor_sealing-10_cone_10p0.pdf} &
\includegraphics[height=2.10in]{SCRIPT_FIGURES/Scaling_Pyrolysis/RISE_fabric_vandalize_protected-foam-42_cone_42p0.pdf} \\
\includegraphics[height=2.10in]{SCRIPT_FIGURES/Scaling_Pyrolysis/RISE_fabric-foam-28_cone_28p0.pdf} &
\includegraphics[height=2.10in]{SCRIPT_FIGURES/Scaling_Pyrolysis/RISE_fabric-protection_layer-foam-32_cone_32p0.pdf} \\
\includegraphics[height=2.10in]{SCRIPT_FIGURES/Scaling_Pyrolysis/RISE_fr_eps-calcium_silicate_board-25_cone_25p0.pdf} &
\includegraphics[height=2.10in]{SCRIPT_FIGURES/Scaling_Pyrolysis/RISE_hpl_melamine-polyester_film_-13_cone_13p4.pdf} \\
\end{tabular*}
\caption[HRRPUA of RISE Materials using scaling model, mixtures materials]
{RISE, mixture materials - Comparison of predicted and measured heat release rate per unit area using scaling-based approach for cone calorimeter experiments.}
\label{RISE_Materials_HRR_Mixtures1}
\end{figure}


\begin{figure}[p]
\begin{tabular*}{\textwidth}{l@{\extracolsep{\fill}}r}
\includegraphics[height=2.10in]{SCRIPT_FIGURES/Scaling_Pyrolysis/RISE_melamine_face-calcium_silicate_board-12_cone_12p5.pdf} &
\includegraphics[height=2.10in]{SCRIPT_FIGURES/Scaling_Pyrolysis/RISE_pe-xlpe-40_cone_40p0.pdf} \\
\includegraphics[height=2.10in]{SCRIPT_FIGURES/Scaling_Pyrolysis/RISE_polyolefin-epr-18_cone_18p1.pdf} &
\includegraphics[height=2.10in]{SCRIPT_FIGURES/Scaling_Pyrolysis/RISE_polyolefin-epr-32_cone_32p2.pdf} \\
\includegraphics[height=2.10in]{SCRIPT_FIGURES/Scaling_Pyrolysis/RISE_polyolefin-pa-2_cone_2p5.pdf} &
\includegraphics[height=2.10in]{SCRIPT_FIGURES/Scaling_Pyrolysis/RISE_polyolefin-pa-6_cone_6p0.pdf} \\
\includegraphics[height=2.10in]{SCRIPT_FIGURES/Scaling_Pyrolysis/RISE_polyolefin-pp-9_cone_8p7.pdf} &
\includegraphics[height=2.10in]{SCRIPT_FIGURES/Scaling_Pyrolysis/RISE_polyolefin-xlpe-18_cone_18p1.pdf} \\
\end{tabular*}
\caption[HRRPUA of RISE Materials using scaling model, mixtures materials]
{RISE, mixture materials - Comparison of predicted and measured heat release rate per unit area using scaling-based approach for cone calorimeter experiments.}
\label{RISE_Materials_HRR_Mixtures2}
\end{figure}

\begin{figure}[p]
\begin{tabular*}{\textwidth}{l@{\extracolsep{\fill}}r}
\includegraphics[height=2.10in]{SCRIPT_FIGURES/Scaling_Pyrolysis/RISE_polyolefin-xlpe-25_cone_25p0.pdf} &
\includegraphics[height=2.10in]{SCRIPT_FIGURES/Scaling_Pyrolysis/RISE_polyolefin-xlpe-38_cone_38p1.pdf} \\
\includegraphics[height=2.10in]{SCRIPT_FIGURES/Scaling_Pyrolysis/RISE_polyolefin-xlpe-45_cone_45p0.pdf} &
\includegraphics[height=2.10in]{SCRIPT_FIGURES/Scaling_Pyrolysis/RISE_pur_rigid-plastic_faced_steel_sheet-79_cone_79p0.pdf} \\
\includegraphics[height=2.10in]{SCRIPT_FIGURES/Scaling_Pyrolysis/RISE_pvc-epr-32_cone_32p5.pdf} &
\includegraphics[height=2.10in]{SCRIPT_FIGURES/Scaling_Pyrolysis/RISE_pvc-xlpe-18_cone_17p7.pdf} \\
\includegraphics[height=2.10in]{SCRIPT_FIGURES/Scaling_Pyrolysis/RISE_pvc-xlpe-22_cone_22p0.pdf} &
\includegraphics[height=2.10in]{SCRIPT_FIGURES/Scaling_Pyrolysis/RISE_pvc-xlpe-35_cone_35p0.pdf} \\
\end{tabular*}
\caption[HRRPUA of RISE Materials using scaling model, mixtures materials]
{RISE, mixture materials - Comparison of predicted and measured heat release rate per unit area using scaling-based approach for cone calorimeter experiments.}
\label{RISE_Materials_HRR_Mixtures3}
\end{figure}

\begin{figure}[p]
\begin{tabular*}{\textwidth}{l@{\extracolsep{\fill}}r}
\includegraphics[height=2.10in]{SCRIPT_FIGURES/Scaling_Pyrolysis/RISE_pvc-xlpe-39_cone_38p6.pdf} &
\includegraphics[height=2.10in]{SCRIPT_FIGURES/Scaling_Pyrolysis/RISE_pvc-xlpe-46_cone_46p0.pdf} \\
\includegraphics[height=2.10in]{SCRIPT_FIGURES/Scaling_Pyrolysis/RISE_rppvc-epr-16_cone_16p5.pdf} &
\includegraphics[height=2.10in]{SCRIPT_FIGURES/Scaling_Pyrolysis/RISE_rppvc-pef-4_cone_4p5.pdf} \\
\includegraphics[height=2.10in]{SCRIPT_FIGURES/Scaling_Pyrolysis/RISE_rppvc-pvc-14_cone_14p0.pdf} &
\includegraphics[height=2.10in]{SCRIPT_FIGURES/Scaling_Pyrolysis/RISE_rppvc-xlpe-18_cone_17p7.pdf} \\
\includegraphics[height=2.10in]{SCRIPT_FIGURES/Scaling_Pyrolysis/RISE_rppvc-xlpe-22_cone_22p5.pdf} &
\includegraphics[height=2.10in]{SCRIPT_FIGURES/Scaling_Pyrolysis/RISE_rppvc-xlpe-39_cone_39p3.pdf} \\
\end{tabular*}
\caption[HRRPUA of RISE Materials using scaling model, mixtures materials]
{RISE, mixture materials - Comparison of predicted and measured heat release rate per unit area using scaling-based approach for cone calorimeter experiments.}
\label{RISE_Materials_HRR_Mixtures4}
\end{figure}

\begin{figure}[p]
\begin{tabular*}{\textwidth}{l@{\extracolsep{\fill}}r}
\includegraphics[height=2.10in]{SCRIPT_FIGURES/Scaling_Pyrolysis/RISE_rppvc-xlpe-45_cone_45p0.pdf} &
\includegraphics[height=2.10in]{SCRIPT_FIGURES/Scaling_Pyrolysis/RISE_synthetic_rubber-glue-plywood-15_cone_15p0.pdf} \\
\includegraphics[height=2.10in]{SCRIPT_FIGURES/Scaling_Pyrolysis/RISE_zhpolyolefin-pp-8_cone_8p4.pdf} &
\includegraphics[height=2.10in]{SCRIPT_FIGURES/Scaling_Pyrolysis/RISE_zhpolyolefin-xlpe-13_cone_13p0.pdf} \\
\includegraphics[height=2.10in]{SCRIPT_FIGURES/Scaling_Pyrolysis/RISE_zhpolyolefin-xlpe-27_cone_27p0.pdf} &
\end{tabular*}
\caption[HRRPUA of RISE Materials using scaling model, mixtures materials]
{RISE, mixture materials - Comparison of predicted and measured heat release rate per unit area using scaling-based approach for cone calorimeter experiments.}
\label{RISE_Materials_HRR_Mixtures5}
\end{figure}

\clearpage


\begin{table}[!h]
\caption[Properties of RISE Materials, other materials]{Properties of RISE Materials, other materials ~\cite{RISE:Fire_Database}.}
\centering
\begin{tabular}{|p{4.5cm}|c|c|c|c|c|p{3.5cm}|}
\hline
            & \centering$\Delta$& \centering$\rho$& \centering$T_{\mathrm{ign}}^{a}$&\centering$\Delta H_{c}$&\centering$Y_{s}^{b}$ & Test \# \\
Material    & \centering$\mathrm{\left(mm\right)}$ & \centering$\mathrm{\left(\frac{kg}{m^{3}}\right)}$ &  \centering($\mathrm{^{\circ}C}$)   & \centering$\left(\mathrm{\frac{MJ}{kg}}\right)$ & \centering$\mathrm{\left(\frac{g}{g}\right)}$ & $\mathrm{( - )}$  \\ \hline
\hline
Aluminium Honey Comb Coated With HPL              & 22.7 & 514 & 537 & 17.3 & 0.050 & 1785-1793, 2362-2364 \\\hline
HPL Compact                                       & 4.0 & 1610 & 500 & 13.8 & 0.050 & 1695-1699, 1701-1705, 2184-2186 \\\hline
Wool Fabric Mixed Fabric                          & 0.5 & 910 & 409 & 19.3 & 0.050 & 1776-1784, 2193-2195, 2356-2361 \\\hline
\end{tabular}
\label{Properties_RISE_Materials_Others}
\end{table}
\vspace{-0.4cm}
\noindent\footnotesize{$^a$ $T_{\mathrm{ign}}$ is calculated based on measured times to ignition. Assumed values used for thermal properties when not measured.}\\
\noindent\footnotesize{$^b$ $Y_{s}$ fixed at 0.05 when measurements were not available.}\\


\begin{figure}[p]
\begin{tabular*}{\textwidth}{l@{\extracolsep{\fill}}r}
\includegraphics[height=2.10in]{SCRIPT_FIGURES/Scaling_Pyrolysis/RISE_aluminium_honey_comb_coated_with_hpl-23_cone_22p7.pdf} &
\includegraphics[height=2.10in]{SCRIPT_FIGURES/Scaling_Pyrolysis/RISE_hpl_compact_-4_cone_4p0.pdf} \\
\includegraphics[height=2.10in]{SCRIPT_FIGURES/Scaling_Pyrolysis/RISE_wool_fabric_mixed_fabric-0_cone_0p5.pdf} &
\end{tabular*}
\caption[HRRPUA of RISE Materials using scaling model, other materials]
{RISE, other materials - Comparison of predicted and measured heat release rate per unit area using scaling-based approach for cone calorimeter experiments.}
\label{RISE_Materials_HRR_Others}
\end{figure}

\clearpage

\begin{table}[!h]
\caption[Properties of RISE Materials, polymer materials]{Properties of RISE Materials, polymer materials ~\cite{RISE:Fire_Database}.}
\centering
\begin{tabular}{|p{5cm}|c|c|c|c|c|p{3.5cm}|}
\hline
            & \centering$\Delta$& \centering$\rho$& \centering$T_{\mathrm{ign}}^{a}$&\centering$\Delta H_{c}$&\centering$Y_{s}^{b}$ & Test \# \\
Material$^{c}$    & \centering$\mathrm{\left(mm\right)}$ & \centering$\mathrm{\left(\frac{kg}{m^{3}}\right)}$ &  \centering($\mathrm{^{\circ}C}$)   & \centering$\left(\mathrm{\frac{MJ}{kg}}\right)$ & \centering$\mathrm{\left(\frac{g}{g}\right)}$ & $\mathrm{( - )}$  \\ \hline
\hline
FR Polycarbonate                                  & 16.0 & 175 & 584 & 21.5 & 0.080 & 240, 244, 248, 251 \\\hline
Paint GFK Polyester With Gelcoat Laminated        & 4.8 & 2022 & 355 & 10.5 & 0.050 & 1726-1734, 2336-2338 \\\hline
Paint GRP Polyester With Gelcoat Laminated        & 4.8 & 2019 & 334 & 5.5 & 0.050 & 2173-2175, 2340-2342 \\\hline
Polyester                                         & 2.1 & 408 & 604 & 23.6 & 0.050 & 543-545 \\\hline
Polyolefin                                        & 2.9 & 472 & 649 & 23.8 & 0.050 & 510-512 \\\hline
PVC                                               & 2.9 & 560 & 388 & 13.1 & 0.050 & 537-539 \\\hline
PVC, PVC                                          & 10.0 & 500 & 290 & 13.3 & 0.050 & 503-505 \\\hline
PVC, PVC                                          & 14.5 & 500 & 291 & 13.3 & 0.050 & 525-527 \\\hline
PVC, PVC                                          & 18.0 & 500 & 292 & 13.3 & 0.050 & 506, 507, 509 \\\hline
PVC, PVC                                          & 21.4 & 500 & 291 & 13.3 & 0.050 & 420-422 \\\hline
PVC, PVC                                          & 8.2 & 500 & 292 & 13.3 & 0.050 & 436-438 \\\hline
PVC, PVC                                          & 9.4 & 500 & 290 & 13.3 & 0.050 & 423-425 \\\hline
PVDF                                              & 1.9 & 410 & 645 & 19.8 & 0.050 & 540-542 \\\hline
RPPVC                                             & 2.9 & 542 & 411 & 13.0 & 0.050 & 433-435 \\\hline
Solid Acrylic                                     & 12.3 & 1903 & 456 & 15.1 & 0.050 & 1794-1798, 1800, 1801, 1804, 1805, 2199-2201 \\\hline
Transparent Polycarbonate                         & 2.3 & 1579 & 528 & 13.2 & 0.050 & 1707, 1708, 1710-1715, 2187-2189 \\\hline
\end{tabular}
\label{Properties_RISE_Materials_Polymers}
\end{table}
\vspace{-0.4cm}
\noindent\footnotesize{$^a$ $T_{\mathrm{ign}}$ is calculated based on measured times to ignition. Assumed values used for thermal properties when not measured.}\\
\noindent\footnotesize{$^b$ $Y_{s}$ fixed at 0.05 when measurements were not available.}\\
\noindent\footnotesize{$^c$ Comma separation indicates layers of material.}\\

\begin{figure}[p]
\begin{tabular*}{\textwidth}{l@{\extracolsep{\fill}}r}
\includegraphics[height=2.10in]{SCRIPT_FIGURES/Scaling_Pyrolysis/RISE_fr_polycarbonate-16_cone_16p0.pdf} &
\includegraphics[height=2.10in]{SCRIPT_FIGURES/Scaling_Pyrolysis/RISE_paint_gfk_polyester_with_gelcoat_laminated-5_cone_4p8.pdf} \\
\includegraphics[height=2.10in]{SCRIPT_FIGURES/Scaling_Pyrolysis/RISE_paint_grp_polyester_with_gelcoat_laminated-5_cone_4p8.pdf} &
\includegraphics[height=2.10in]{SCRIPT_FIGURES/Scaling_Pyrolysis/RISE_polyester-2_cone_2p1.pdf} \\
\includegraphics[height=2.10in]{SCRIPT_FIGURES/Scaling_Pyrolysis/RISE_polyolefin-3_cone_2p9.pdf} &
\includegraphics[height=2.10in]{SCRIPT_FIGURES/Scaling_Pyrolysis/RISE_pvc-3_cone_2p9.pdf} \\
\includegraphics[height=2.10in]{SCRIPT_FIGURES/Scaling_Pyrolysis/RISE_pvc-pvc-10_cone_10p0.pdf} &
\includegraphics[height=2.10in]{SCRIPT_FIGURES/Scaling_Pyrolysis/RISE_pvc-pvc-14_cone_14p5.pdf} \\
\end{tabular*}
\caption[HRRPUA of RISE Materials using scaling model, polymer materials]
{RISE, polymer materials - Comparison of predicted and measured heat release rate per unit area using scaling-based approach for cone calorimeter experiments.}
\label{RISE_Materials_HRR_Polymers1}
\end{figure}

\begin{figure}[p]
\begin{tabular*}{\textwidth}{l@{\extracolsep{\fill}}r}
\includegraphics[height=2.10in]{SCRIPT_FIGURES/Scaling_Pyrolysis/RISE_pvc-pvc-18_cone_18p0.pdf} &
\includegraphics[height=2.10in]{SCRIPT_FIGURES/Scaling_Pyrolysis/RISE_pvc-pvc-21_cone_21p4.pdf} \\
\includegraphics[height=2.10in]{SCRIPT_FIGURES/Scaling_Pyrolysis/RISE_pvc-pvc-8_cone_8p2.pdf} &
\includegraphics[height=2.10in]{SCRIPT_FIGURES/Scaling_Pyrolysis/RISE_pvc-pvc-9_cone_9p4.pdf} \\
\includegraphics[height=2.10in]{SCRIPT_FIGURES/Scaling_Pyrolysis/RISE_pvdf-2_cone_1p9.pdf} &
\includegraphics[height=2.10in]{SCRIPT_FIGURES/Scaling_Pyrolysis/RISE_rppvc-3_cone_2p9.pdf} \\
\includegraphics[height=2.10in]{SCRIPT_FIGURES/Scaling_Pyrolysis/RISE_solid_acrylic-12_cone_12p3.pdf} &
\includegraphics[height=2.10in]{SCRIPT_FIGURES/Scaling_Pyrolysis/RISE_transparent_polycarbonate-2_cone_2p3.pdf} \\
\end{tabular*}
\caption[HRRPUA of RISE Materials using scaling model, polymer materials]
{RISE, polymer materials - Comparison of predicted and measured heat release rate per unit area using scaling-based approach for cone calorimeter experiments.}
\label{RISE_Materials_HRR_Polymers2}
\end{figure}

\clearpage

\begin{table}[!h]
\caption[Properties of RISE Materials, Wood-Based materials]{Properties of RISE Materials, Wood-Based materials ~\cite{RISE:Fire_Database}.}
\centering
\begin{tabular}{|l|c|c|c|c|c|p{5.5cm}|}
\hline
            & \centering$\Delta$& \centering$\rho$& \centering$T_{\mathrm{ign}}^{a}$&\centering$\Delta H_{c}$&\centering$Y_{s}^{b}$ & Test \# \\
Material    & \centering$\mathrm{\left(mm\right)}$ & \centering$\mathrm{\left(\frac{kg}{m^{3}}\right)}$ &  \centering($\mathrm{^{\circ}C}$)   & \centering$\left(\mathrm{\frac{MJ}{kg}}\right)$ & \centering$\mathrm{\left(\frac{g}{g}\right)}$ & $\mathrm{( - )}$  \\ \hline
\hline
FR Particle Board                                 & 12.0 & 517 & 364 & 7.4 & 0.049 & 158-161 \\\hline
FR Particle Board                                 & 16.0 & 517 & 368 & 7.4 & 0.049 & 146-151, 72, 73 \\\hline
FR Particle Board                                 & 79.0 & 517 & 381 & 7.4 & 0.049 & 162-167 \\\hline
MDF Board                                         & 12.0 & 700 & 424 & 12.2 & 0.007 & 241, 243, 246, 249 \\\hline
Spruce                                            & 10.0 & 450 & 297 & 12.5 & 0.007 & 245, 247, 250 \\\hline
\end{tabular}
\label{Properties_RISE_Materials_Wood-Based}
\end{table}
\vspace{-0.4cm}
\noindent\footnotesize{$^a$ $T_{\mathrm{ign}}$ is calculated based on measured times to ignition. Assumed values used for thermal properties when not measured.}\\
\noindent\footnotesize{$^b$ $Y_{s}$ fixed at 0.05 when measurements were not available.}\\

\begin{figure}[p]
\begin{tabular*}{\textwidth}{l@{\extracolsep{\fill}}r}
\includegraphics[height=2.10in]{SCRIPT_FIGURES/Scaling_Pyrolysis/RISE_fr_particle_board-12_cone_12p0.pdf} &
\includegraphics[height=2.10in]{SCRIPT_FIGURES/Scaling_Pyrolysis/RISE_fr_particle_board-16_cone_16p0.pdf} \\
\includegraphics[height=2.10in]{SCRIPT_FIGURES/Scaling_Pyrolysis/RISE_fr_particle_board-79_cone_79p0.pdf} &
\includegraphics[height=2.10in]{SCRIPT_FIGURES/Scaling_Pyrolysis/RISE_mdf_board-12_cone_12p0.pdf} \\
\includegraphics[height=2.10in]{SCRIPT_FIGURES/Scaling_Pyrolysis/RISE_spruce-10_cone_10p0.pdf} &
\end{tabular*}
\caption[HRRPUA of RISE Materials using scaling model, wood-based materials]
{RISE, wood-based materials - Comparison of predicted and measured heat release rate per unit area using scaling-based approach for cone calorimeter experiments.}
\label{RISE_Materials_HRR_Wood-Based}
\end{figure}

\clearpage

\section{Compartment Fires}\label{sec_Compartment_Fires}

This section evaluates the performance of FDS in predicting the growth of a fire in an enclosure for well characterized materials.
Measurements and predictions of HRR in each test configuration are compared.

In the U.S., the NFPA 286 room-corner test is the common standard large scale test used to measure the contribution of interior finish materials on fire growth and regulate the use of interior finish in certain building occupancies.
The room is 2.4 m wide by 3.6 m long by 2.4 m high, and has a doorway 2 m (79.5 in) in height and 0.78 m (37.5 in) in width.
The lining materials are installed on ceiling and walls. A propane gas burner measuring 0.3 m x 0.3 m is placed in one of the back corners, in contact with the walls.
The HRR for the burner is 40 kW for 5 minutes, and then 160 kW for 10 minutes.

Many of these cases are in the NFPA 286/ISO 9705 room-corner configuration (or a modification thereof).

\subsection{JH/FRA Reduced Scale Compartment Experiments}\label{sec_JH_FRA_Scaled_Compartments}

The Federal Railroad Administration (FRA) in the U.S. sponsored a series of small to intermediate-scale enclosure experiments to evaluate the impact of modern materials on HRRs in fully-developed railcar fires.
The following pages compare FDS predictions of room-corner testing of scaled compartments.

\begin{figure}[h!]
\centering
\includegraphics[height=2.15in]{SCRIPT_FIGURES/JH_FRA/JH_FRA_compartment_03_hrr}
\caption[JH/FRA experiments, HRR, 1:4 scale plywood half lining configuration, Test 3]{JH/FRA experiments, HRR, 1:4 scale plywood half lining configuration, Test 3.}
\label{JH_FRA_plywood_01}
\end{figure}

\begin{figure}[h!]
\centering
\includegraphics[height=2.15in]{SCRIPT_FIGURES/JH_FRA/JH_FRA_compartment_03A_hrr}
\caption[JH/FRA experiments, HRR, 1:4 scale plywood full lining configuration, Test 3A]{JH/FRA experiments, HRR, 1:4 scale plywood full lining configuration, Test 3A.}
\label{JH_FRA_plywood_01A}
\end{figure}

%\begin{figure}[h!]
%\centering
%\includegraphics[height=2.15in]{SCRIPT_FIGURES/JH_FRA/JH_FRA_compartment_04_hrr}
%\caption[JH/FRA experiments, HRR, 1:4 scale FRP lining configuration]{JH/FRA experiments, HRR, 1:4 scale FRP lining configuration.}
%\label{JH_FRA_FRP_01}
%\end{figure}

\begin{figure}[h!]
\centering
\includegraphics[height=2.15in]{SCRIPT_FIGURES/JH_FRA/JH_FRA_compartment_13_hrr}
\caption[JH/FRA experiments, HRR, 1:2 scale plywood half lining configuration, Test 8]{JH/FRA experiments, HRR, 1:2 scale plywood half lining configuration, Test 8.}
\label{JH_FRA_plywood_02}
\end{figure}

%\begin{figure}[h!]
%\centering
%\includegraphics[height=2.15in]{SCRIPT_FIGURES/JH_FRA/JH_FRA_compartment_14_hrr}
%\caption[JH/FRA experiments, HRR, 1:1 scale FRP lining configuration]{JH/FRA experiments, HRR, 1:2 scale FRP lining configuration.}
%\label{JH_FRA_FRP_02}
%\end{figure}

\begin{figure}[h!]
\centering
\includegraphics[height=2.15in]{SCRIPT_FIGURES/JH_FRA/JH_FRA_compartment_23_hrr}
\caption[JH/FRA experiments, HRR, 1:1 scale plywood half lining configuration, Test 13]{JH/FRA experiments, HRR, 1:2 scale plywood half lining configuration, Test 13.}
\label{JH_FRA_plywood_03}
\end{figure}

%\clearpage

%\subsection{JH/FRA NFPA 286 Compartment Experiments}\label{sec_JH_FRA_NFPA286_Compartments}
%
%The Federal Railroad Administration (FRA) in the U.S. sponsored a series NFPA 286 enclosure experiments to evaluate the impact of modern materials on HRRs in fully-developed railcar fires.
%The following pages compare FDS predictions of room-corner testing of full-scale compartments.
%
%\begin{figure}[h!]
%\centering
%\includegraphics[height=2.15in]{SCRIPT_FIGURES/JH_FRA/JH_FRA_nfpa286_acrylic_hrr}
%\caption[JH/FRA NFPA 286 experiments, HRR, Acrylic lining configuration]{JH/FRA NFPA 286 experiments, HRR, Acrylic lining configuration.}
%\label{JH_FRA_acrylic}
%\end{figure}
%
%\begin{figure}[h!]
%\centering
%\includegraphics[height=2.15in]{SCRIPT_FIGURES/JH_FRA/JH_FRA_nfpa286_cps_balsa_hrr}
%\caption[JH/FRA NFPA 286 experiments, HRR, Balsa Composite lining configuration]{JH/FRA NFPA 286 experiments, HRR, Balsa Composite lining configuration.}
%\label{JH_FRA_balsa}
%\end{figure}
%
%\begin{figure}[h!]
%\centering
%\includegraphics[height=2.15in]{SCRIPT_FIGURES/JH_FRA/JH_FRA_nfpa286_cps_plywood_hrr}
%\caption[JH/FRA NFPA 286 experiments, HRR, Plywood Composite lining configuration]{JH/FRA NFPA 286 experiments, HRR, Plywood Composite lining configuration.}
%\label{JH_FRA_plywood}
%\end{figure}
%
%\begin{figure}[h!]
%\centering
%\includegraphics[height=2.15in]{SCRIPT_FIGURES/JH_FRA/JH_FRA_nfpa286_phenolic_resin_fiberglass_composite_hrr}
%\caption[JH/FRA NFPA 286 experiments, HRR, Phenolic Resin FRP lining configuration]{JH/FRA NFPA 286 experiments, HRR, Phenolic Resin FRP lining configuration.}
%\label{JH_FRA_phenolic}
%\end{figure}
%
%\begin{figure}[h!]
%\centering
%\includegraphics[height=2.15in]{SCRIPT_FIGURES/JH_FRA/JH_FRA_nfpa286_vinyl_ester_resin_frp_hrr}
%\caption[JH/FRA NFPA 286 experiments, HRR, Vinyl Ester Resin FRP lining configuration]{JH/FRA NFPA 286 experiments, HRR, Vinyl Ester Resin FRP lining configuration.}
%\label{JH_FRA_vinyl}
%\end{figure}
%
%\clearpage

\section{Wildland Fire Burning and Spread Rates}
\label{WUI}

The following sections present examples of fire spread through vegetation, both small and full-scale. Summary plots of burning and spread rates are presented in Figs.~\ref{Burning_Rate} and \ref{RoS_Summary}.


\subsection{Crown Fires}

This section presents the rate of spread for simulations of crown fires. For a description of the experiments and simulations, see Sec.~\ref{Crown_Fires_Description}. The experimental data consists of 57 observed crown fires. The simulations are performed with comparable conditions, but not all input parameters can be gleaned from the experimental reports. Thus, the comparison is largely qualitative, and has not been quantified in any way other than the comparison plot in Fig.~\ref{Crown_Fire_Plot}.

\begin{figure}[ht]
\centering
\includegraphics[height=2.15in]{SCRIPT_FIGURES/Crown_Fires/Crown_Fires_ROS}
\caption[Comparison observed and predicted rates of spread for a variety of crown fires]{Comparison observed and predicted rates of spread for a variety of crown fires.}
\label{Crown_Fire_Plot}
\end{figure}


\clearpage


\subsection{CSIRO Grassland Fires}

This section presents the rate of spread for simulations of two of the CSIRO Grassland Fire experiments. For details of the experiments and simulations, see Sec.~\ref{CSIRO_Grassland_Fires_Description}. The first experiment, C064, was conducted on a 100~m by 100~m plot; the second, F19, was conducted on a 200~m by 200~m plot. The results of the simulations are shown in Fig.~\ref{CSIRO}. The fire front in the FDS simulations is defined as the location of the maximum gas temperature in a 1~m wide, 1~m tall strip along the centerline of the grass field. The experimental points were determined from aerial photography.

For each case, we perform simulations with three different fuel models---one using Lagrangian particles to represent the vegetation, one using the Boundary Fuel Model (BFM), and the other using Rothermel-Albini fuel models in a level set fire spread simulation.  For each fuel model, we also run at three different grid resolutions, as indicated in the plots in Fig.~\ref{CSIRO}.  The Lagrangian particle and Boundary Fuel Model require higher grid resolution $\delta x = [1, 0.5, 0.25]$ m.  The level set simulations are designed for relatively coarse grids, in this case we run at $\delta x = [20, 10, 5]$ m.

\begin{figure}[p]
\begin{tabular*}{\textwidth}{l@{\extracolsep{\fill}}r}
\includegraphics[height=2.15in]{SCRIPT_FIGURES/CSIRO_Grassland_Fires/Case_C064} & \includegraphics[height=2.15in]{SCRIPT_FIGURES/CSIRO_Grassland_Fires/Case_C064_BFM} \\
\multicolumn{2}{c}{\includegraphics[height=2.15in]{SCRIPT_FIGURES/CSIRO_Grassland_Fires/Case_C064_LS} } \\
\includegraphics[height=2.15in]{SCRIPT_FIGURES/CSIRO_Grassland_Fires/Case_F19}  & \includegraphics[height=2.15in]{SCRIPT_FIGURES/CSIRO_Grassland_Fires/Case_F19_BFM} \\
\multicolumn{2}{c}{\includegraphics[height=2.15in]{SCRIPT_FIGURES/CSIRO_Grassland_Fires/Case_F19_LS} }
\end{tabular*}
\caption[Measured and predicted fire front position for the CSIRO Grassland Fires]{Comparison of the measured and predicted fire front position for the CSIRO Grassland Fires using three different methods of fire spread.}
\label{CSIRO}
\end{figure}


\clearpage


\subsection{USFS/Catchpole Experiments}
\label{USFS_Catchpole_Plots}

Figures~\ref{USFS_Catchpole_008} through \ref{USFS_Catchpole_354} present the results of 354 simulations of the USFS/Catchpole experiments. A brief description is given in Sec.~\ref{USFS_Catchpole_Description}. The paper by Catchpole et al.~\cite{Catchpole:CST1998} reports a single rate of spread for each experiment, which is depicted in the figures as a straight black line. The rate of spread of the simulations was calculated by fitting the best line through the data points over a time interval between 10~\% and 90~\% of the observed transit time of the real fire over the 8~m fuel bed. The red dashed line is the best fit line from which the rate of spread is taken.

\newpage

\begin{figure}[p]
\begin{tabular*}{\textwidth}{l@{\extracolsep{\fill}}r}
\includegraphics[height=2.15in]{SCRIPT_FIGURES/USFS_Catchpole/MF54} &
\includegraphics[height=2.15in]{SCRIPT_FIGURES/USFS_Catchpole/MF43} \\
\includegraphics[height=2.15in]{SCRIPT_FIGURES/USFS_Catchpole/MF50} &
\includegraphics[height=2.15in]{SCRIPT_FIGURES/USFS_Catchpole/MF48} \\
\includegraphics[height=2.15in]{SCRIPT_FIGURES/USFS_Catchpole/MF32} &
\includegraphics[height=2.15in]{SCRIPT_FIGURES/USFS_Catchpole/MF49} \\
\includegraphics[height=2.15in]{SCRIPT_FIGURES/USFS_Catchpole/MF42} &
\includegraphics[height=2.15in]{SCRIPT_FIGURES/USFS_Catchpole/MF51} \\
\end{tabular*}
\caption[Flame front, USFS/Catchpole experiments]{Flame front, USFS/Catchpole experiments}
\label{USFS_Catchpole_008}
\end{figure}

\begin{figure}[p]
\begin{tabular*}{\textwidth}{l@{\extracolsep{\fill}}r}
\includegraphics[height=2.15in]{SCRIPT_FIGURES/USFS_Catchpole/MF47} &
\includegraphics[height=2.15in]{SCRIPT_FIGURES/USFS_Catchpole/MF37} \\
\includegraphics[height=2.15in]{SCRIPT_FIGURES/USFS_Catchpole/MF31} &
\includegraphics[height=2.15in]{SCRIPT_FIGURES/USFS_Catchpole/MF55} \\
\includegraphics[height=2.15in]{SCRIPT_FIGURES/USFS_Catchpole/MF56} &
\includegraphics[height=2.15in]{SCRIPT_FIGURES/USFS_Catchpole/MF24} \\
\includegraphics[height=2.15in]{SCRIPT_FIGURES/USFS_Catchpole/MF52} &
\includegraphics[height=2.15in]{SCRIPT_FIGURES/USFS_Catchpole/MF38} \\
\end{tabular*}
\caption[Flame front, USFS/Catchpole experiments]{Flame front, USFS/Catchpole experiments}
\label{USFS_Catchpole_016}
\end{figure}

\begin{figure}[p]
\begin{tabular*}{\textwidth}{l@{\extracolsep{\fill}}r}
\includegraphics[height=2.15in]{SCRIPT_FIGURES/USFS_Catchpole/MF53} &
\includegraphics[height=2.15in]{SCRIPT_FIGURES/USFS_Catchpole/MF29} \\
\includegraphics[height=2.15in]{SCRIPT_FIGURES/USFS_Catchpole/MF21} &
\includegraphics[height=2.15in]{SCRIPT_FIGURES/USFS_Catchpole/MF20} \\
\includegraphics[height=2.15in]{SCRIPT_FIGURES/USFS_Catchpole/EXSC1} &
\includegraphics[height=2.15in]{SCRIPT_FIGURES/USFS_Catchpole/EXSC60} \\
\includegraphics[height=2.15in]{SCRIPT_FIGURES/USFS_Catchpole/EXSC61} &
\includegraphics[height=2.15in]{SCRIPT_FIGURES/USFS_Catchpole/EXSC2B} \\
\end{tabular*}
\caption[Flame front, USFS/Catchpole experiments]{Flame front, USFS/Catchpole experiments}
\label{USFS_Catchpole_024}
\end{figure}

\begin{figure}[p]
\begin{tabular*}{\textwidth}{l@{\extracolsep{\fill}}r}
\includegraphics[height=2.15in]{SCRIPT_FIGURES/USFS_Catchpole/EXSC1A} &
\includegraphics[height=2.15in]{SCRIPT_FIGURES/USFS_Catchpole/EXSC67} \\
\includegraphics[height=2.15in]{SCRIPT_FIGURES/USFS_Catchpole/EXSC71} &
\includegraphics[height=2.15in]{SCRIPT_FIGURES/USFS_Catchpole/EXSC3F} \\
\includegraphics[height=2.15in]{SCRIPT_FIGURES/USFS_Catchpole/EXSC8A} &
\includegraphics[height=2.15in]{SCRIPT_FIGURES/USFS_Catchpole/EXSC68} \\
\includegraphics[height=2.15in]{SCRIPT_FIGURES/USFS_Catchpole/EXSC46} &
\includegraphics[height=2.15in]{SCRIPT_FIGURES/USFS_Catchpole/EXSC9E} \\
\end{tabular*}
\caption[Flame front, USFS/Catchpole experiments]{Flame front, USFS/Catchpole experiments}
\label{USFS_Catchpole_032}
\end{figure}

\begin{figure}[p]
\begin{tabular*}{\textwidth}{l@{\extracolsep{\fill}}r}
\includegraphics[height=2.15in]{SCRIPT_FIGURES/USFS_Catchpole/EXSC84} &
\includegraphics[height=2.15in]{SCRIPT_FIGURES/USFS_Catchpole/EXSC6A} \\
\includegraphics[height=2.15in]{SCRIPT_FIGURES/USFS_Catchpole/EXSC98} &
\includegraphics[height=2.15in]{SCRIPT_FIGURES/USFS_Catchpole/EXSC96} \\
\includegraphics[height=2.15in]{SCRIPT_FIGURES/USFS_Catchpole/EXSC99} &
\includegraphics[height=2.15in]{SCRIPT_FIGURES/USFS_Catchpole/EXSC47} \\
\includegraphics[height=2.15in]{SCRIPT_FIGURES/USFS_Catchpole/EXSC1C} &
\includegraphics[height=2.15in]{SCRIPT_FIGURES/USFS_Catchpole/EXSC2C} \\
\end{tabular*}
\caption[Flame front, USFS/Catchpole experiments]{Flame front, USFS/Catchpole experiments}
\label{USFS_Catchpole_040}
\end{figure}

\begin{figure}[p]
\begin{tabular*}{\textwidth}{l@{\extracolsep{\fill}}r}
\includegraphics[height=2.15in]{SCRIPT_FIGURES/USFS_Catchpole/EXSC7A} &
\includegraphics[height=2.15in]{SCRIPT_FIGURES/USFS_Catchpole/EXSC5E} \\
\includegraphics[height=2.15in]{SCRIPT_FIGURES/USFS_Catchpole/EXSC2E} &
\includegraphics[height=2.15in]{SCRIPT_FIGURES/USFS_Catchpole/EXSC64} \\
\includegraphics[height=2.15in]{SCRIPT_FIGURES/USFS_Catchpole/EXSC4B} &
\includegraphics[height=2.15in]{SCRIPT_FIGURES/USFS_Catchpole/EXSC48} \\
\includegraphics[height=2.15in]{SCRIPT_FIGURES/USFS_Catchpole/EXSC6D} &
\includegraphics[height=2.15in]{SCRIPT_FIGURES/USFS_Catchpole/EXSC65} \\
\end{tabular*}
\caption[Flame front, USFS/Catchpole experiments]{Flame front, USFS/Catchpole experiments}
\label{USFS_Catchpole_048}
\end{figure}

\begin{figure}[p]
\begin{tabular*}{\textwidth}{l@{\extracolsep{\fill}}r}
\includegraphics[height=2.15in]{SCRIPT_FIGURES/USFS_Catchpole/EXSC7B} &
\includegraphics[height=2.15in]{SCRIPT_FIGURES/USFS_Catchpole/EXSC49} \\
\includegraphics[height=2.15in]{SCRIPT_FIGURES/USFS_Catchpole/EXSC32} &
\includegraphics[height=2.15in]{SCRIPT_FIGURES/USFS_Catchpole/EXSC6B} \\
\includegraphics[height=2.15in]{SCRIPT_FIGURES/USFS_Catchpole/EXSC93} &
\includegraphics[height=2.15in]{SCRIPT_FIGURES/USFS_Catchpole/EXSC92} \\
\includegraphics[height=2.15in]{SCRIPT_FIGURES/USFS_Catchpole/EXSC75} &
\includegraphics[height=2.15in]{SCRIPT_FIGURES/USFS_Catchpole/EXSC73} \\
\end{tabular*}
\caption[Flame front, USFS/Catchpole experiments]{Flame front, USFS/Catchpole experiments}
\label{USFS_Catchpole_056}
\end{figure}

\begin{figure}[p]
\begin{tabular*}{\textwidth}{l@{\extracolsep{\fill}}r}
\includegraphics[height=2.15in]{SCRIPT_FIGURES/USFS_Catchpole/EXSC5B} &
\includegraphics[height=2.15in]{SCRIPT_FIGURES/USFS_Catchpole/EXSC77} \\
\includegraphics[height=2.15in]{SCRIPT_FIGURES/USFS_Catchpole/EXSC2} &
\includegraphics[height=2.15in]{SCRIPT_FIGURES/USFS_Catchpole/EXSC3} \\
\includegraphics[height=2.15in]{SCRIPT_FIGURES/USFS_Catchpole/EXSC85} &
\includegraphics[height=2.15in]{SCRIPT_FIGURES/USFS_Catchpole/EXSC3D} \\
\includegraphics[height=2.15in]{SCRIPT_FIGURES/USFS_Catchpole/EXSC4D} &
\includegraphics[height=2.15in]{SCRIPT_FIGURES/USFS_Catchpole/EXSC66} \\
\end{tabular*}
\caption[Flame front, USFS/Catchpole experiments]{Flame front, USFS/Catchpole experiments}
\label{USFS_Catchpole_064}
\end{figure}

\begin{figure}[p]
\begin{tabular*}{\textwidth}{l@{\extracolsep{\fill}}r}
\includegraphics[height=2.15in]{SCRIPT_FIGURES/USFS_Catchpole/EXSC8F} &
\includegraphics[height=2.15in]{SCRIPT_FIGURES/USFS_Catchpole/EXSC5C} \\
\includegraphics[height=2.15in]{SCRIPT_FIGURES/USFS_Catchpole/EXSC9B} &
\includegraphics[height=2.15in]{SCRIPT_FIGURES/USFS_Catchpole/EXSC8B} \\
\includegraphics[height=2.15in]{SCRIPT_FIGURES/USFS_Catchpole/EXSC4C} &
\includegraphics[height=2.15in]{SCRIPT_FIGURES/USFS_Catchpole/EXSC95} \\
\includegraphics[height=2.15in]{SCRIPT_FIGURES/USFS_Catchpole/EXSC3C} &
\includegraphics[height=2.15in]{SCRIPT_FIGURES/USFS_Catchpole/EXSC7C} \\
\end{tabular*}
\caption[Flame front, USFS/Catchpole experiments]{Flame front, USFS/Catchpole experiments}
\label{USFS_Catchpole_072}
\end{figure}

\begin{figure}[p]
\begin{tabular*}{\textwidth}{l@{\extracolsep{\fill}}r}
\includegraphics[height=2.15in]{SCRIPT_FIGURES/USFS_Catchpole/EXSC6C} &
\includegraphics[height=2.15in]{SCRIPT_FIGURES/USFS_Catchpole/EXSC8C} \\
\includegraphics[height=2.15in]{SCRIPT_FIGURES/USFS_Catchpole/EXSC58} &
\includegraphics[height=2.15in]{SCRIPT_FIGURES/USFS_Catchpole/EXSC55} \\
\includegraphics[height=2.15in]{SCRIPT_FIGURES/USFS_Catchpole/EXSC52} &
\includegraphics[height=2.15in]{SCRIPT_FIGURES/USFS_Catchpole/EXSC59} \\
\includegraphics[height=2.15in]{SCRIPT_FIGURES/USFS_Catchpole/EXSC2D} &
\includegraphics[height=2.15in]{SCRIPT_FIGURES/USFS_Catchpole/EXSC51} \\
\end{tabular*}
\caption[Flame front, USFS/Catchpole experiments]{Flame front, USFS/Catchpole experiments}
\label{USFS_Catchpole_080}
\end{figure}

\begin{figure}[p]
\begin{tabular*}{\textwidth}{l@{\extracolsep{\fill}}r}
\includegraphics[height=2.15in]{SCRIPT_FIGURES/USFS_Catchpole/EXSC56} &
\includegraphics[height=2.15in]{SCRIPT_FIGURES/USFS_Catchpole/EXSC1D} \\
\includegraphics[height=2.15in]{SCRIPT_FIGURES/USFS_Catchpole/EXSC9C} &
\includegraphics[height=2.15in]{SCRIPT_FIGURES/USFS_Catchpole/EXSC82} \\
\includegraphics[height=2.15in]{SCRIPT_FIGURES/USFS_Catchpole/EXSC7D} &
\includegraphics[height=2.15in]{SCRIPT_FIGURES/USFS_Catchpole/EXSC1E} \\
\includegraphics[height=2.15in]{SCRIPT_FIGURES/USFS_Catchpole/EXSC9D} &
\includegraphics[height=2.15in]{SCRIPT_FIGURES/USFS_Catchpole/EXSC8D} \\
\end{tabular*}
\caption[Flame front, USFS/Catchpole experiments]{Flame front, USFS/Catchpole experiments}
\label{USFS_Catchpole_088}
\end{figure}

\begin{figure}[p]
\begin{tabular*}{\textwidth}{l@{\extracolsep{\fill}}r}
\includegraphics[height=2.15in]{SCRIPT_FIGURES/USFS_Catchpole/EXSC62} &
\includegraphics[height=2.15in]{SCRIPT_FIGURES/USFS_Catchpole/EXSC83} \\
\includegraphics[height=2.15in]{SCRIPT_FIGURES/USFS_Catchpole/EXSC86} &
\includegraphics[height=2.15in]{SCRIPT_FIGURES/USFS_Catchpole/EXSC63} \\
\includegraphics[height=2.15in]{SCRIPT_FIGURES/USFS_Catchpole/EXSC3B} &
\includegraphics[height=2.15in]{SCRIPT_FIGURES/USFS_Catchpole/EXSC88} \\
\includegraphics[height=2.15in]{SCRIPT_FIGURES/USFS_Catchpole/EXSC44} &
\includegraphics[height=2.15in]{SCRIPT_FIGURES/USFS_Catchpole/EXSC45} \\
\end{tabular*}
\caption[Flame front, USFS/Catchpole experiments]{Flame front, USFS/Catchpole experiments}
\label{USFS_Catchpole_096}
\end{figure}

\begin{figure}[p]
\begin{tabular*}{\textwidth}{l@{\extracolsep{\fill}}r}
\includegraphics[height=2.15in]{SCRIPT_FIGURES/USFS_Catchpole/EXSC22} &
\includegraphics[height=2.15in]{SCRIPT_FIGURES/USFS_Catchpole/EXSC1B} \\
\includegraphics[height=2.15in]{SCRIPT_FIGURES/USFS_Catchpole/EXSC2A} &
\includegraphics[height=2.15in]{SCRIPT_FIGURES/USFS_Catchpole/EXSC21} \\
\includegraphics[height=2.15in]{SCRIPT_FIGURES/USFS_Catchpole/EXSC57} &
\includegraphics[height=2.15in]{SCRIPT_FIGURES/USFS_Catchpole/EXSC70} \\
\includegraphics[height=2.15in]{SCRIPT_FIGURES/USFS_Catchpole/EXSC9A} &
\includegraphics[height=2.15in]{SCRIPT_FIGURES/USFS_Catchpole/EXSC5} \\
\end{tabular*}
\caption[Flame front, USFS/Catchpole experiments]{Flame front, USFS/Catchpole experiments}
\label{USFS_Catchpole_104}
\end{figure}

\begin{figure}[p]
\begin{tabular*}{\textwidth}{l@{\extracolsep{\fill}}r}
\includegraphics[height=2.15in]{SCRIPT_FIGURES/USFS_Catchpole/EXSC53} &
\includegraphics[height=2.15in]{SCRIPT_FIGURES/USFS_Catchpole/EXSC97} \\
\includegraphics[height=2.15in]{SCRIPT_FIGURES/USFS_Catchpole/EXSC5A} &
\includegraphics[height=2.15in]{SCRIPT_FIGURES/USFS_Catchpole/EXSC8E} \\
\includegraphics[height=2.15in]{SCRIPT_FIGURES/USFS_Catchpole/EXSC4A} &
\includegraphics[height=2.15in]{SCRIPT_FIGURES/USFS_Catchpole/EXSC69} \\
\includegraphics[height=2.15in]{SCRIPT_FIGURES/USFS_Catchpole/EXSC3A} &
\includegraphics[height=2.15in]{SCRIPT_FIGURES/USFS_Catchpole/EXSC23} \\
\end{tabular*}
\caption[Flame front, USFS/Catchpole experiments]{Flame front, USFS/Catchpole experiments}
\label{USFS_Catchpole_112}
\end{figure}

\begin{figure}[p]
\begin{tabular*}{\textwidth}{l@{\extracolsep{\fill}}r}
\includegraphics[height=2.15in]{SCRIPT_FIGURES/USFS_Catchpole/EXSC4} &
\includegraphics[height=2.15in]{SCRIPT_FIGURES/USFS_Catchpole/EXSC16} \\
\includegraphics[height=2.15in]{SCRIPT_FIGURES/USFS_Catchpole/EXSC24} &
\includegraphics[height=2.15in]{SCRIPT_FIGURES/USFS_Catchpole/EXSC27} \\
\includegraphics[height=2.15in]{SCRIPT_FIGURES/USFS_Catchpole/EXSC25} &
\includegraphics[height=2.15in]{SCRIPT_FIGURES/USFS_Catchpole/EXSC80} \\
\includegraphics[height=2.15in]{SCRIPT_FIGURES/USFS_Catchpole/EXSC78} &
\includegraphics[height=2.15in]{SCRIPT_FIGURES/USFS_Catchpole/EXSC87} \\
\end{tabular*}
\caption[Flame front, USFS/Catchpole experiments]{Flame front, USFS/Catchpole experiments}
\label{USFS_Catchpole_120}
\end{figure}

\begin{figure}[p]
\begin{tabular*}{\textwidth}{l@{\extracolsep{\fill}}r}
\includegraphics[height=2.15in]{SCRIPT_FIGURES/USFS_Catchpole/EXSC54} &
\includegraphics[height=2.15in]{SCRIPT_FIGURES/USFS_Catchpole/EXSC6} \\
\includegraphics[height=2.15in]{SCRIPT_FIGURES/USFS_Catchpole/EXSC14} &
\includegraphics[height=2.15in]{SCRIPT_FIGURES/USFS_Catchpole/EXSC30} \\
\includegraphics[height=2.15in]{SCRIPT_FIGURES/USFS_Catchpole/EXSC15} &
\includegraphics[height=2.15in]{SCRIPT_FIGURES/USFS_Catchpole/EXSC34} \\
\includegraphics[height=2.15in]{SCRIPT_FIGURES/USFS_Catchpole/EXSC28} &
\includegraphics[height=2.15in]{SCRIPT_FIGURES/USFS_Catchpole/EXSC26} \\
\end{tabular*}
\caption[Flame front, USFS/Catchpole experiments]{Flame front, USFS/Catchpole experiments}
\label{USFS_Catchpole_128}
\end{figure}

\begin{figure}[p]
\begin{tabular*}{\textwidth}{l@{\extracolsep{\fill}}r}
\includegraphics[height=2.15in]{SCRIPT_FIGURES/USFS_Catchpole/EXSC79} &
\includegraphics[height=2.15in]{SCRIPT_FIGURES/USFS_Catchpole/EXSC81} \\
\includegraphics[height=2.15in]{SCRIPT_FIGURES/USFS_Catchpole/EXSC89} &
\includegraphics[height=2.15in]{SCRIPT_FIGURES/USFS_Catchpole/EXSC42} \\
\includegraphics[height=2.15in]{SCRIPT_FIGURES/USFS_Catchpole/EXSC7E} &
\includegraphics[height=2.15in]{SCRIPT_FIGURES/USFS_Catchpole/EXSC6E} \\
\includegraphics[height=2.15in]{SCRIPT_FIGURES/USFS_Catchpole/EXSC7} &
\includegraphics[height=2.15in]{SCRIPT_FIGURES/USFS_Catchpole/EXSC43} \\
\end{tabular*}
\caption[Flame front, USFS/Catchpole experiments]{Flame front, USFS/Catchpole experiments}
\label{USFS_Catchpole_136}
\end{figure}

\begin{figure}[p]
\begin{tabular*}{\textwidth}{l@{\extracolsep{\fill}}r}
\includegraphics[height=2.15in]{SCRIPT_FIGURES/USFS_Catchpole/EXSC4F} &
\includegraphics[height=2.15in]{SCRIPT_FIGURES/USFS_Catchpole/EXSC3E} \\
\includegraphics[height=2.15in]{SCRIPT_FIGURES/USFS_Catchpole/EXSC4E} &
\includegraphics[height=2.15in]{SCRIPT_FIGURES/USFS_Catchpole/EXSC20} \\
\includegraphics[height=2.15in]{SCRIPT_FIGURES/USFS_Catchpole/EXSC38} &
\includegraphics[height=2.15in]{SCRIPT_FIGURES/USFS_Catchpole/EXSC8} \\
\includegraphics[height=2.15in]{SCRIPT_FIGURES/USFS_Catchpole/EXSC17} &
\includegraphics[height=2.15in]{SCRIPT_FIGURES/USFS_Catchpole/EXSC29} \\
\end{tabular*}
\caption[Flame front, USFS/Catchpole experiments]{Flame front, USFS/Catchpole experiments}
\label{USFS_Catchpole_144}
\end{figure}

\begin{figure}[p]
\begin{tabular*}{\textwidth}{l@{\extracolsep{\fill}}r}
\includegraphics[height=2.15in]{SCRIPT_FIGURES/USFS_Catchpole/EXSC33} &
\includegraphics[height=2.15in]{SCRIPT_FIGURES/USFS_Catchpole/EXSC40} \\
\includegraphics[height=2.15in]{SCRIPT_FIGURES/USFS_Catchpole/EXSC36} &
\includegraphics[height=2.15in]{SCRIPT_FIGURES/USFS_Catchpole/EXSC50} \\
\includegraphics[height=2.15in]{SCRIPT_FIGURES/USFS_Catchpole/EXSC18} &
\includegraphics[height=2.15in]{SCRIPT_FIGURES/USFS_Catchpole/EXSC74} \\
\includegraphics[height=2.15in]{SCRIPT_FIGURES/USFS_Catchpole/EXSC9} &
\includegraphics[height=2.15in]{SCRIPT_FIGURES/USFS_Catchpole/EXSC19} \\
\end{tabular*}
\caption[Flame front, USFS/Catchpole experiments]{Flame front, USFS/Catchpole experiments}
\label{USFS_Catchpole_152}
\end{figure}

\begin{figure}[p]
\begin{tabular*}{\textwidth}{l@{\extracolsep{\fill}}r}
\includegraphics[height=2.15in]{SCRIPT_FIGURES/USFS_Catchpole/EXSC31} &
\includegraphics[height=2.15in]{SCRIPT_FIGURES/USFS_Catchpole/EXSC35} \\
\includegraphics[height=2.15in]{SCRIPT_FIGURES/USFS_Catchpole/EXSC37} &
\includegraphics[height=2.15in]{SCRIPT_FIGURES/USFS_Catchpole/EXSC76} \\
\includegraphics[height=2.15in]{SCRIPT_FIGURES/USFS_Catchpole/EXSC72} &
\includegraphics[height=2.15in]{SCRIPT_FIGURES/USFS_Catchpole/EXSC12} \\
\includegraphics[height=2.15in]{SCRIPT_FIGURES/USFS_Catchpole/EXSC10} &
\includegraphics[height=2.15in]{SCRIPT_FIGURES/USFS_Catchpole/EXSC13} \\
\end{tabular*}
\caption[Flame front, USFS/Catchpole experiments]{Flame front, USFS/Catchpole experiments}
\label{USFS_Catchpole_160}
\end{figure}

\begin{figure}[p]
\begin{tabular*}{\textwidth}{l@{\extracolsep{\fill}}r}
\includegraphics[height=2.15in]{SCRIPT_FIGURES/USFS_Catchpole/EXSC11} &
\includegraphics[height=2.15in]{SCRIPT_FIGURES/USFS_Catchpole/PPMC78} \\
\includegraphics[height=2.15in]{SCRIPT_FIGURES/USFS_Catchpole/PPMC87} &
\includegraphics[height=2.15in]{SCRIPT_FIGURES/USFS_Catchpole/PPMC9H} \\
\includegraphics[height=2.15in]{SCRIPT_FIGURES/USFS_Catchpole/PPMC3C} &
\includegraphics[height=2.15in]{SCRIPT_FIGURES/USFS_Catchpole/PPMC59} \\
\includegraphics[height=2.15in]{SCRIPT_FIGURES/USFS_Catchpole/PPMC1C} &
\includegraphics[height=2.15in]{SCRIPT_FIGURES/USFS_Catchpole/PPMC7B} \\
\end{tabular*}
\caption[Flame front, USFS/Catchpole experiments]{Flame front, USFS/Catchpole experiments}
\label{USFS_Catchpole_168}
\end{figure}

\begin{figure}[p]
\begin{tabular*}{\textwidth}{l@{\extracolsep{\fill}}r}
\includegraphics[height=2.15in]{SCRIPT_FIGURES/USFS_Catchpole/PPMC60} &
\includegraphics[height=2.15in]{SCRIPT_FIGURES/USFS_Catchpole/PPMC2C} \\
\includegraphics[height=2.15in]{SCRIPT_FIGURES/USFS_Catchpole/PPMC8B} &
\includegraphics[height=2.15in]{SCRIPT_FIGURES/USFS_Catchpole/PPMC7H} \\
\includegraphics[height=2.15in]{SCRIPT_FIGURES/USFS_Catchpole/PPMC3D} &
\includegraphics[height=2.15in]{SCRIPT_FIGURES/USFS_Catchpole/PPMC9C} \\
\includegraphics[height=2.15in]{SCRIPT_FIGURES/USFS_Catchpole/PPMC7F} &
\includegraphics[height=2.15in]{SCRIPT_FIGURES/USFS_Catchpole/PPMC8J} \\
\end{tabular*}
\caption[Flame front, USFS/Catchpole experiments]{Flame front, USFS/Catchpole experiments}
\label{USFS_Catchpole_176}
\end{figure}

\begin{figure}[p]
\begin{tabular*}{\textwidth}{l@{\extracolsep{\fill}}r}
\includegraphics[height=2.15in]{SCRIPT_FIGURES/USFS_Catchpole/PPMC5F} &
\includegraphics[height=2.15in]{SCRIPT_FIGURES/USFS_Catchpole/PPMC9J} \\
\includegraphics[height=2.15in]{SCRIPT_FIGURES/USFS_Catchpole/PPMC1H} &
\includegraphics[height=2.15in]{SCRIPT_FIGURES/USFS_Catchpole/PPMC6C} \\
\includegraphics[height=2.15in]{SCRIPT_FIGURES/USFS_Catchpole/PPMC6F} &
\includegraphics[height=2.15in]{SCRIPT_FIGURES/USFS_Catchpole/PPMC3H} \\
\includegraphics[height=2.15in]{SCRIPT_FIGURES/USFS_Catchpole/PPMC7C} &
\includegraphics[height=2.15in]{SCRIPT_FIGURES/USFS_Catchpole/PPMC74} \\
\end{tabular*}
\caption[Flame front, USFS/Catchpole experiments]{Flame front, USFS/Catchpole experiments}
\label{USFS_Catchpole_184}
\end{figure}

\begin{figure}[p]
\begin{tabular*}{\textwidth}{l@{\extracolsep{\fill}}r}
\includegraphics[height=2.15in]{SCRIPT_FIGURES/USFS_Catchpole/PPMC49} &
\includegraphics[height=2.15in]{SCRIPT_FIGURES/USFS_Catchpole/PPMC54} \\
\includegraphics[height=2.15in]{SCRIPT_FIGURES/USFS_Catchpole/PPMC2B} &
\includegraphics[height=2.15in]{SCRIPT_FIGURES/USFS_Catchpole/PPMC2} \\
\includegraphics[height=2.15in]{SCRIPT_FIGURES/USFS_Catchpole/PPMC45} &
\includegraphics[height=2.15in]{SCRIPT_FIGURES/USFS_Catchpole/PPMC88} \\
\includegraphics[height=2.15in]{SCRIPT_FIGURES/USFS_Catchpole/PPMC99} &
\includegraphics[height=2.15in]{SCRIPT_FIGURES/USFS_Catchpole/PPMC50} \\
\end{tabular*}
\caption[Flame front, USFS/Catchpole experiments]{Flame front, USFS/Catchpole experiments}
\label{USFS_Catchpole_192}
\end{figure}

\begin{figure}[p]
\begin{tabular*}{\textwidth}{l@{\extracolsep{\fill}}r}
\includegraphics[height=2.15in]{SCRIPT_FIGURES/USFS_Catchpole/PPMC11} &
\includegraphics[height=2.15in]{SCRIPT_FIGURES/USFS_Catchpole/PPMC56} \\
\includegraphics[height=2.15in]{SCRIPT_FIGURES/USFS_Catchpole/PPMC16} &
\includegraphics[height=2.15in]{SCRIPT_FIGURES/USFS_Catchpole/PPMC51} \\
\includegraphics[height=2.15in]{SCRIPT_FIGURES/USFS_Catchpole/PPMC52} &
\includegraphics[height=2.15in]{SCRIPT_FIGURES/USFS_Catchpole/PPMC72} \\
\includegraphics[height=2.15in]{SCRIPT_FIGURES/USFS_Catchpole/PPMC12} &
\includegraphics[height=2.15in]{SCRIPT_FIGURES/USFS_Catchpole/PPMC77} \\
\end{tabular*}
\caption[Flame front, USFS/Catchpole experiments]{Flame front, USFS/Catchpole experiments}
\label{USFS_Catchpole_200}
\end{figure}

\begin{figure}[p]
\begin{tabular*}{\textwidth}{l@{\extracolsep{\fill}}r}
\includegraphics[height=2.15in]{SCRIPT_FIGURES/USFS_Catchpole/PPMC75} &
\includegraphics[height=2.15in]{SCRIPT_FIGURES/USFS_Catchpole/PPMC73} \\
\includegraphics[height=2.15in]{SCRIPT_FIGURES/USFS_Catchpole/PPMC55} &
\includegraphics[height=2.15in]{SCRIPT_FIGURES/USFS_Catchpole/PPMC15} \\
\includegraphics[height=2.15in]{SCRIPT_FIGURES/USFS_Catchpole/PPMC61} &
\includegraphics[height=2.15in]{SCRIPT_FIGURES/USFS_Catchpole/PPMC53} \\
\includegraphics[height=2.15in]{SCRIPT_FIGURES/USFS_Catchpole/PPMC1E} &
\includegraphics[height=2.15in]{SCRIPT_FIGURES/USFS_Catchpole/PPMC9D} \\
\end{tabular*}
\caption[Flame front, USFS/Catchpole experiments]{Flame front, USFS/Catchpole experiments}
\label{USFS_Catchpole_208}
\end{figure}

\begin{figure}[p]
\begin{tabular*}{\textwidth}{l@{\extracolsep{\fill}}r}
\includegraphics[height=2.15in]{SCRIPT_FIGURES/USFS_Catchpole/PPMC7D} &
\includegraphics[height=2.15in]{SCRIPT_FIGURES/USFS_Catchpole/PPMC3F} \\
\includegraphics[height=2.15in]{SCRIPT_FIGURES/USFS_Catchpole/PPMC76} &
\includegraphics[height=2.15in]{SCRIPT_FIGURES/USFS_Catchpole/PPMC13} \\
\includegraphics[height=2.15in]{SCRIPT_FIGURES/USFS_Catchpole/PPMC8D} &
\includegraphics[height=2.15in]{SCRIPT_FIGURES/USFS_Catchpole/PPMC71} \\
\includegraphics[height=2.15in]{SCRIPT_FIGURES/USFS_Catchpole/PPMC2F} &
\includegraphics[height=2.15in]{SCRIPT_FIGURES/USFS_Catchpole/PPMC4D} \\
\end{tabular*}
\caption[Flame front, USFS/Catchpole experiments]{Flame front, USFS/Catchpole experiments}
\label{USFS_Catchpole_216}
\end{figure}

\FloatBarrier

\begin{figure}[p]
\begin{tabular*}{\textwidth}{l@{\extracolsep{\fill}}r}
\includegraphics[height=2.15in]{SCRIPT_FIGURES/USFS_Catchpole/PPMC17} &
\includegraphics[height=2.15in]{SCRIPT_FIGURES/USFS_Catchpole/PPMC8C} \\
\includegraphics[height=2.15in]{SCRIPT_FIGURES/USFS_Catchpole/PPMC4F} &
\includegraphics[height=2.15in]{SCRIPT_FIGURES/USFS_Catchpole/PPMC8E} \\
\includegraphics[height=2.15in]{SCRIPT_FIGURES/USFS_Catchpole/PPMC9E} &
\includegraphics[height=2.15in]{SCRIPT_FIGURES/USFS_Catchpole/PPMC98} \\
\includegraphics[height=2.15in]{SCRIPT_FIGURES/USFS_Catchpole/PPMC57} &
\includegraphics[height=2.15in]{SCRIPT_FIGURES/USFS_Catchpole/PPMC4E} \\
\end{tabular*}
\caption[Flame front, USFS/Catchpole experiments]{Flame front, USFS/Catchpole experiments}
\label{USFS_Catchpole_224}
\end{figure}

\begin{figure}[p]
\begin{tabular*}{\textwidth}{l@{\extracolsep{\fill}}r}
\includegraphics[height=2.15in]{SCRIPT_FIGURES/USFS_Catchpole/PPMC3E} &
\includegraphics[height=2.15in]{SCRIPT_FIGURES/USFS_Catchpole/PPMC2E} \\
\includegraphics[height=2.15in]{SCRIPT_FIGURES/USFS_Catchpole/PPMC5E} &
\includegraphics[height=2.15in]{SCRIPT_FIGURES/USFS_Catchpole/PPMC14} \\
\includegraphics[height=2.15in]{SCRIPT_FIGURES/USFS_Catchpole/PPMC6H} &
\includegraphics[height=2.15in]{SCRIPT_FIGURES/USFS_Catchpole/PPMC5H} \\
\includegraphics[height=2.15in]{SCRIPT_FIGURES/USFS_Catchpole/PPMC7E} &
\includegraphics[height=2.15in]{SCRIPT_FIGURES/USFS_Catchpole/PPMC79} \\
\end{tabular*}
\caption[Flame front, USFS/Catchpole experiments]{Flame front, USFS/Catchpole experiments}
\label{USFS_Catchpole_232}
\end{figure}

\begin{figure}[p]
\begin{tabular*}{\textwidth}{l@{\extracolsep{\fill}}r}
\includegraphics[height=2.15in]{SCRIPT_FIGURES/USFS_Catchpole/PPMC6E} &
\includegraphics[height=2.15in]{SCRIPT_FIGURES/USFS_Catchpole/PPMC58} \\
\includegraphics[height=2.15in]{SCRIPT_FIGURES/USFS_Catchpole/PPMC9G} &
\includegraphics[height=2.15in]{SCRIPT_FIGURES/USFS_Catchpole/PPMC1B} \\
\includegraphics[height=2.15in]{SCRIPT_FIGURES/USFS_Catchpole/PPMC8} &
\includegraphics[height=2.15in]{SCRIPT_FIGURES/USFS_Catchpole/PPMC8F} \\
\includegraphics[height=2.15in]{SCRIPT_FIGURES/USFS_Catchpole/PPMC2H} &
\includegraphics[height=2.15in]{SCRIPT_FIGURES/USFS_Catchpole/PPMC1} \\
\end{tabular*}
\caption[Flame front, USFS/Catchpole experiments]{Flame front, USFS/Catchpole experiments}
\label{USFS_Catchpole_240}
\end{figure}

\begin{figure}[p]
\begin{tabular*}{\textwidth}{l@{\extracolsep{\fill}}r}
\includegraphics[height=2.15in]{SCRIPT_FIGURES/USFS_Catchpole/PPMC9F} &
\includegraphics[height=2.15in]{SCRIPT_FIGURES/USFS_Catchpole/PPMC10} \\
\includegraphics[height=2.15in]{SCRIPT_FIGURES/USFS_Catchpole/PPMC4H} &
\includegraphics[height=2.15in]{SCRIPT_FIGURES/USFS_Catchpole/PPMC44} \\
\includegraphics[height=2.15in]{SCRIPT_FIGURES/USFS_Catchpole/PPMC5} &
\includegraphics[height=2.15in]{SCRIPT_FIGURES/USFS_Catchpole/PPMC6} \\
\includegraphics[height=2.15in]{SCRIPT_FIGURES/USFS_Catchpole/EXMC6J} &
\includegraphics[height=2.15in]{SCRIPT_FIGURES/USFS_Catchpole/EXMC20} \\
\end{tabular*}
\caption[Flame front, USFS/Catchpole experiments]{Flame front, USFS/Catchpole experiments}
\label{USFS_Catchpole_248}
\end{figure}

\begin{figure}[p]
\begin{tabular*}{\textwidth}{l@{\extracolsep{\fill}}r}
\includegraphics[height=2.15in]{SCRIPT_FIGURES/USFS_Catchpole/EXMC83} &
\includegraphics[height=2.15in]{SCRIPT_FIGURES/USFS_Catchpole/EXMC95} \\
\includegraphics[height=2.15in]{SCRIPT_FIGURES/USFS_Catchpole/EXMC86} &
\includegraphics[height=2.15in]{SCRIPT_FIGURES/USFS_Catchpole/EXMC41} \\
\includegraphics[height=2.15in]{SCRIPT_FIGURES/USFS_Catchpole/EXMC97} &
\includegraphics[height=2.15in]{SCRIPT_FIGURES/USFS_Catchpole/EXMC42} \\
\includegraphics[height=2.15in]{SCRIPT_FIGURES/USFS_Catchpole/EXMC3I} &
\includegraphics[height=2.15in]{SCRIPT_FIGURES/USFS_Catchpole/EXMC1I} \\
\end{tabular*}
\caption[Flame front, USFS/Catchpole experiments]{Flame front, USFS/Catchpole experiments}
\label{USFS_Catchpole_256}
\end{figure}

\begin{figure}[p]
\begin{tabular*}{\textwidth}{l@{\extracolsep{\fill}}r}
\includegraphics[height=2.15in]{SCRIPT_FIGURES/USFS_Catchpole/EXMC9I} &
\includegraphics[height=2.15in]{SCRIPT_FIGURES/USFS_Catchpole/EXMC8I} \\
\includegraphics[height=2.15in]{SCRIPT_FIGURES/USFS_Catchpole/EXMC2I} &
\includegraphics[height=2.15in]{SCRIPT_FIGURES/USFS_Catchpole/EXMC9B} \\
\includegraphics[height=2.15in]{SCRIPT_FIGURES/USFS_Catchpole/EXMC3A} &
\includegraphics[height=2.15in]{SCRIPT_FIGURES/USFS_Catchpole/EXMC91} \\
\includegraphics[height=2.15in]{SCRIPT_FIGURES/USFS_Catchpole/EXMC92} &
\includegraphics[height=2.15in]{SCRIPT_FIGURES/USFS_Catchpole/EXMC94} \\
\end{tabular*}
\caption[Flame front, USFS/Catchpole experiments]{Flame front, USFS/Catchpole experiments}
\label{USFS_Catchpole_264}
\end{figure}

\begin{figure}[p]
\begin{tabular*}{\textwidth}{l@{\extracolsep{\fill}}r}
\includegraphics[height=2.15in]{SCRIPT_FIGURES/USFS_Catchpole/EXMC33} &
\includegraphics[height=2.15in]{SCRIPT_FIGURES/USFS_Catchpole/EXMC9A} \\
\includegraphics[height=2.15in]{SCRIPT_FIGURES/USFS_Catchpole/EXMC61} &
\includegraphics[height=2.15in]{SCRIPT_FIGURES/USFS_Catchpole/EXMC4I} \\
\includegraphics[height=2.15in]{SCRIPT_FIGURES/USFS_Catchpole/EXMC51} &
\includegraphics[height=2.15in]{SCRIPT_FIGURES/USFS_Catchpole/EXMC18} \\
\includegraphics[height=2.15in]{SCRIPT_FIGURES/USFS_Catchpole/EXMC24} &
\includegraphics[height=2.15in]{SCRIPT_FIGURES/USFS_Catchpole/EXMC26} \\
\end{tabular*}
\caption[Flame front, USFS/Catchpole experiments]{Flame front, USFS/Catchpole experiments}
\label{USFS_Catchpole_272}
\end{figure}

\begin{figure}[p]
\begin{tabular*}{\textwidth}{l@{\extracolsep{\fill}}r}
\includegraphics[height=2.15in]{SCRIPT_FIGURES/USFS_Catchpole/EXMC34} &
\includegraphics[height=2.15in]{SCRIPT_FIGURES/USFS_Catchpole/EXMC63} \\
\includegraphics[height=2.15in]{SCRIPT_FIGURES/USFS_Catchpole/EXMC46} &
\includegraphics[height=2.15in]{SCRIPT_FIGURES/USFS_Catchpole/EXMC68} \\
\includegraphics[height=2.15in]{SCRIPT_FIGURES/USFS_Catchpole/EXMC28} &
\includegraphics[height=2.15in]{SCRIPT_FIGURES/USFS_Catchpole/EXMC22} \\
\includegraphics[height=2.15in]{SCRIPT_FIGURES/USFS_Catchpole/EXMC27} &
\includegraphics[height=2.15in]{SCRIPT_FIGURES/USFS_Catchpole/EXMC35} \\
\end{tabular*}
\caption[Flame front, USFS/Catchpole experiments]{Flame front, USFS/Catchpole experiments}
\label{USFS_Catchpole_280}
\end{figure}

\begin{figure}[p]
\begin{tabular*}{\textwidth}{l@{\extracolsep{\fill}}r}
\includegraphics[height=2.15in]{SCRIPT_FIGURES/USFS_Catchpole/EXMC64} &
\includegraphics[height=2.15in]{SCRIPT_FIGURES/USFS_Catchpole/EXMC47} \\
\includegraphics[height=2.15in]{SCRIPT_FIGURES/USFS_Catchpole/EXMC69} &
\includegraphics[height=2.15in]{SCRIPT_FIGURES/USFS_Catchpole/EXMC23} \\
\includegraphics[height=2.15in]{SCRIPT_FIGURES/USFS_Catchpole/EXMC30} &
\includegraphics[height=2.15in]{SCRIPT_FIGURES/USFS_Catchpole/EXMC36} \\
\includegraphics[height=2.15in]{SCRIPT_FIGURES/USFS_Catchpole/EXMC65} &
\includegraphics[height=2.15in]{SCRIPT_FIGURES/USFS_Catchpole/EXMC48} \\
\end{tabular*}
\caption[Flame front, USFS/Catchpole experiments]{Flame front, USFS/Catchpole experiments}
\label{USFS_Catchpole_288}
\end{figure}

\begin{figure}[p]
\begin{tabular*}{\textwidth}{l@{\extracolsep{\fill}}r}
\includegraphics[height=2.15in]{SCRIPT_FIGURES/USFS_Catchpole/EXMC84} &
\includegraphics[height=2.15in]{SCRIPT_FIGURES/USFS_Catchpole/EXMC96} \\
\includegraphics[height=2.15in]{SCRIPT_FIGURES/USFS_Catchpole/MF23} &
\includegraphics[height=2.15in]{SCRIPT_FIGURES/USFS_Catchpole/EX77} \\
\includegraphics[height=2.15in]{SCRIPT_FIGURES/USFS_Catchpole/EXMC25} &
\includegraphics[height=2.15in]{SCRIPT_FIGURES/USFS_Catchpole/MF30} \\
\includegraphics[height=2.15in]{SCRIPT_FIGURES/USFS_Catchpole/EXMC3} &
\includegraphics[height=2.15in]{SCRIPT_FIGURES/USFS_Catchpole/EXMC4} \\
\end{tabular*}
\caption[Flame front, USFS/Catchpole experiments]{Flame front, USFS/Catchpole experiments}
\label{USFS_Catchpole_296}
\end{figure}

\begin{figure}[p]
\begin{tabular*}{\textwidth}{l@{\extracolsep{\fill}}r}
\includegraphics[height=2.15in]{SCRIPT_FIGURES/USFS_Catchpole/EXMC7} &
\includegraphics[height=2.15in]{SCRIPT_FIGURES/USFS_Catchpole/EXMC5J} \\
\includegraphics[height=2.15in]{SCRIPT_FIGURES/USFS_Catchpole/EXMC2J} &
\includegraphics[height=2.15in]{SCRIPT_FIGURES/USFS_Catchpole/EXMC1J} \\
\includegraphics[height=2.15in]{SCRIPT_FIGURES/USFS_Catchpole/EXMC2A} &
\includegraphics[height=2.15in]{SCRIPT_FIGURES/USFS_Catchpole/EXMC1A} \\
\includegraphics[height=2.15in]{SCRIPT_FIGURES/USFS_Catchpole/EXMC4A} &
\includegraphics[height=2.15in]{SCRIPT_FIGURES/USFS_Catchpole/EXMC7A} \\
\end{tabular*}
\caption[Flame front, USFS/Catchpole experiments]{Flame front, USFS/Catchpole experiments}
\label{USFS_Catchpole_304}
\end{figure}

\begin{figure}[p]
\begin{tabular*}{\textwidth}{l@{\extracolsep{\fill}}r}
\includegraphics[height=2.15in]{SCRIPT_FIGURES/USFS_Catchpole/EXMC19} &
\includegraphics[height=2.15in]{SCRIPT_FIGURES/USFS_Catchpole/EXMC82} \\
\includegraphics[height=2.15in]{SCRIPT_FIGURES/USFS_Catchpole/EXMC39} &
\includegraphics[height=2.15in]{SCRIPT_FIGURES/USFS_Catchpole/EXMC40} \\
\includegraphics[height=2.15in]{SCRIPT_FIGURES/USFS_Catchpole/EXMC66} &
\includegraphics[height=2.15in]{SCRIPT_FIGURES/USFS_Catchpole/EXMC4G} \\
\includegraphics[height=2.15in]{SCRIPT_FIGURES/USFS_Catchpole/EXMC2G} &
\includegraphics[height=2.15in]{SCRIPT_FIGURES/USFS_Catchpole/EXMC6G} \\
\end{tabular*}
\caption[Flame front, USFS/Catchpole experiments]{Flame front, USFS/Catchpole experiments}
\label{USFS_Catchpole_312}
\end{figure}

\begin{figure}[p]
\begin{tabular*}{\textwidth}{l@{\extracolsep{\fill}}r}
\includegraphics[height=2.15in]{SCRIPT_FIGURES/USFS_Catchpole/EXMC8G} &
\includegraphics[height=2.15in]{SCRIPT_FIGURES/USFS_Catchpole/EXMC3G} \\
\includegraphics[height=2.15in]{SCRIPT_FIGURES/USFS_Catchpole/EXMC1G} &
\includegraphics[height=2.15in]{SCRIPT_FIGURES/USFS_Catchpole/EXMC7G} \\
\includegraphics[height=2.15in]{SCRIPT_FIGURES/USFS_Catchpole/EXMC5G} &
\includegraphics[height=2.15in]{SCRIPT_FIGURES/USFS_Catchpole/EXMC5D} \\
\includegraphics[height=2.15in]{SCRIPT_FIGURES/USFS_Catchpole/EXMC3B} &
\includegraphics[height=2.15in]{SCRIPT_FIGURES/USFS_Catchpole/EXMC5B} \\
\end{tabular*}
\caption[Flame front, USFS/Catchpole experiments]{Flame front, USFS/Catchpole experiments}
\label{USFS_Catchpole_320}
\end{figure}

\begin{figure}[p]
\begin{tabular*}{\textwidth}{l@{\extracolsep{\fill}}r}
\includegraphics[height=2.15in]{SCRIPT_FIGURES/USFS_Catchpole/EXMC71} &
\includegraphics[height=2.15in]{SCRIPT_FIGURES/USFS_Catchpole/EXMC1D} \\
\includegraphics[height=2.15in]{SCRIPT_FIGURES/USFS_Catchpole/EXMC1K} &
\includegraphics[height=2.15in]{SCRIPT_FIGURES/USFS_Catchpole/EXMC4C} \\
\includegraphics[height=2.15in]{SCRIPT_FIGURES/USFS_Catchpole/EXMC6D} &
\includegraphics[height=2.15in]{SCRIPT_FIGURES/USFS_Catchpole/EXMC8H} \\
\includegraphics[height=2.15in]{SCRIPT_FIGURES/USFS_Catchpole/EXMC4B} &
\includegraphics[height=2.15in]{SCRIPT_FIGURES/USFS_Catchpole/EXMC6B} \\
\end{tabular*}
\caption[Flame front, USFS/Catchpole experiments]{Flame front, USFS/Catchpole experiments}
\label{USFS_Catchpole_328}
\end{figure}

\begin{figure}[p]
\begin{tabular*}{\textwidth}{l@{\extracolsep{\fill}}r}
\includegraphics[height=2.15in]{SCRIPT_FIGURES/USFS_Catchpole/EXMC2D} &
\includegraphics[height=2.15in]{SCRIPT_FIGURES/USFS_Catchpole/EXMC5C} \\
\includegraphics[height=2.15in]{SCRIPT_FIGURES/USFS_Catchpole/EXMC89} &
\includegraphics[height=2.15in]{SCRIPT_FIGURES/USFS_Catchpole/EXMC90} \\
\includegraphics[height=2.15in]{SCRIPT_FIGURES/USFS_Catchpole/EXMC70} &
\includegraphics[height=2.15in]{SCRIPT_FIGURES/USFS_Catchpole/EXMC31} \\
\includegraphics[height=2.15in]{SCRIPT_FIGURES/USFS_Catchpole/EXMC32} &
\includegraphics[height=2.15in]{SCRIPT_FIGURES/USFS_Catchpole/EXMC37} \\
\end{tabular*}
\caption[Flame front, USFS/Catchpole experiments]{Flame front, USFS/Catchpole experiments}
\label{USFS_Catchpole_336}
\end{figure}

\begin{figure}[p]
\begin{tabular*}{\textwidth}{l@{\extracolsep{\fill}}r}
\includegraphics[height=2.15in]{SCRIPT_FIGURES/USFS_Catchpole/EXMC62} &
\includegraphics[height=2.15in]{SCRIPT_FIGURES/USFS_Catchpole/EXMC43} \\
\includegraphics[height=2.15in]{SCRIPT_FIGURES/USFS_Catchpole/EXMC93} &
\includegraphics[height=2.15in]{SCRIPT_FIGURES/USFS_Catchpole/EXMC67} \\
\includegraphics[height=2.15in]{SCRIPT_FIGURES/USFS_Catchpole/EXMC85} &
\includegraphics[height=2.15in]{SCRIPT_FIGURES/USFS_Catchpole/EXMC6A} \\
\includegraphics[height=2.15in]{SCRIPT_FIGURES/USFS_Catchpole/EXMC8A} &
\includegraphics[height=2.15in]{SCRIPT_FIGURES/USFS_Catchpole/EXMC80} \\
\end{tabular*}
\caption[Flame front, USFS/Catchpole experiments]{Flame front, USFS/Catchpole experiments}
\label{USFS_Catchpole_344}
\end{figure}

\begin{figure}[p]
\begin{tabular*}{\textwidth}{l@{\extracolsep{\fill}}r}
\includegraphics[height=2.15in]{SCRIPT_FIGURES/USFS_Catchpole/EXMC29} &
\includegraphics[height=2.15in]{SCRIPT_FIGURES/USFS_Catchpole/EXMC21} \\
\includegraphics[height=2.15in]{SCRIPT_FIGURES/USFS_Catchpole/EXMC81} &
\includegraphics[height=2.15in]{SCRIPT_FIGURES/USFS_Catchpole/EXMC38} \\
\includegraphics[height=2.15in]{SCRIPT_FIGURES/USFS_Catchpole/EX74} &
\includegraphics[height=2.15in]{SCRIPT_FIGURES/USFS_Catchpole/EX73} \\
\includegraphics[height=2.15in]{SCRIPT_FIGURES/USFS_Catchpole/EX72} &
\includegraphics[height=2.15in]{SCRIPT_FIGURES/USFS_Catchpole/EX75} \\
\end{tabular*}
\caption[Flame front, USFS/Catchpole experiments]{Flame front, USFS/Catchpole experiments}
\label{USFS_Catchpole_352}
\end{figure}

\begin{figure}[p]
\begin{tabular*}{\textwidth}{l@{\extracolsep{\fill}}r}
\includegraphics[height=2.15in]{SCRIPT_FIGURES/USFS_Catchpole/EX76} &
\includegraphics[height=2.15in]{SCRIPT_FIGURES/USFS_Catchpole/EXMC3J} \\
\end{tabular*}
\caption[Flame front, USFS/Catchpole experiments]{Flame front, USFS/Catchpole experiments}
\label{USFS_Catchpole_354}
\end{figure}





\clearpage

\subsection{USFS/Corsica Fire Spread Experiments}
\label{USFS_Corsica_Results}

A description of the experiments and modeling strategy can be found in Sec.~\ref{USFS_Corsica_Description}. Comparisons of the measured and predicted heat release rates for six bench-scale (1~m by 2~m) fires spreading over pine needles are shown in Fig.~\ref{USFS_Corsica_HRR} below. Front trajectory plots are shown in Fig.~\ref{USFS_Corsica_RoS}.

\begin{figure}[!ht]
\begin{tabular*}{\textwidth}{l@{\extracolsep{\fill}}r}
\includegraphics[height=2.15in]{SCRIPT_FIGURES/USFS_Corsica/Test_1_0_HRR} &
\includegraphics[height=2.15in]{SCRIPT_FIGURES/USFS_Corsica/Test_1_20_HRR} \\
\includegraphics[height=2.15in]{SCRIPT_FIGURES/USFS_Corsica/Test_2_0_HRR} &
\includegraphics[height=2.15in]{SCRIPT_FIGURES/USFS_Corsica/Test_2_20_HRR} \\
\includegraphics[height=2.15in]{SCRIPT_FIGURES/USFS_Corsica/Test_3_0_HRR} &
\includegraphics[height=2.15in]{SCRIPT_FIGURES/USFS_Corsica/Test_3_20_HRR}
\end{tabular*}
\caption[HRR, USFS/Corsica experiments]{HRR, USFS/Corsica experiments.}
\label{USFS_Corsica_HRR}
\end{figure}

\begin{figure}[p]
\begin{tabular*}{\textwidth}{l@{\extracolsep{\fill}}r}
\includegraphics[height=2.15in]{SCRIPT_FIGURES/USFS_Corsica/Test_1_0_RoS} &
\includegraphics[height=2.15in]{SCRIPT_FIGURES/USFS_Corsica/Test_1_20_RoS} \\
\includegraphics[height=2.15in]{SCRIPT_FIGURES/USFS_Corsica/Test_2_0_RoS} &
\includegraphics[height=2.15in]{SCRIPT_FIGURES/USFS_Corsica/Test_2_20_RoS} \\
\includegraphics[height=2.15in]{SCRIPT_FIGURES/USFS_Corsica/Test_3_0_RoS} &
\includegraphics[height=2.15in]{SCRIPT_FIGURES/USFS_Corsica/Test_3_20_RoS}
\end{tabular*}
\caption[Rate of Spread, USFS/Corsica experiments]{Rate of Spread, USFS/Corsica experiments.}
\label{USFS_Corsica_RoS}
\end{figure}




\clearpage

\subsection{Burning Trees (NIST Douglas Firs)}
\label{Douglas_Firs}

A description of the experiments and modeling assumptions are given in Sec.~\ref{NIST_Douglas_Firs_Description}.

Snapshots of the simulation of the 2~m tall, 14~\% moisture tree are shown in Fig.~\ref{tree_snaps}. The computational domain in this case is 2~m by 2~m by 4~m. The grid cells are 5~cm cubes. The pine needles are represented by 130,000 Lagrangian particles with a cylindrical geometry, or about 25 simulated needles per grid cell. The radius of the cylinder is derived from the measured surface area to volume ratio. Each simulated pine needle or segment of roundwood represents many more actual needles or segments. The weighting factor is determined from the estimated bulk mass per unit volume. The results of the simulations are shown in Fig.~\ref{NIST_Douglas_Fir_MLR}.

\begin{figure}[ht]
\includegraphics[width=2in]{FIGURES/NIST_Douglas_Firs/tree_2_m_14_pc_0290}
\includegraphics[width=2in]{FIGURES/NIST_Douglas_Firs/tree_2_m_14_pc_0544}
\includegraphics[width=2in]{FIGURES/NIST_Douglas_Firs/tree_2_m_14_pc_0879}
\caption[Snapshots of a 2~m Douglas fir fire simulation]{Snapshots of the simulation of the 2~m tall Douglas fir tree, 14~\% moisture.}
\label{tree_snaps}
\end{figure}



\newpage

\begin{figure}[h]
\begin{tabular*}{\textwidth}{l@{\extracolsep{\fill}}r}
\includegraphics[height=2.15in]{SCRIPT_FIGURES/NIST_Douglas_Firs/tree_2_m_14_pc} &
\includegraphics[height=2.15in]{SCRIPT_FIGURES/NIST_Douglas_Firs/tree_2_m_49_pc} \\
\multicolumn{2}{c}{\includegraphics[height=2.15in]{SCRIPT_FIGURES/NIST_Douglas_Firs/tree_5_m_26_pc} }
\end{tabular*}
\caption[Measured and predicted MLR for the Douglas fir tree experiments]{Comparison measured and predicted mass loss rate for the Douglas fir tree experiments.}
\label{NIST_Douglas_Fir_MLR}
\end{figure}


\newpage


\section{Summary of Burning and Spread Rates}
\label{Burning Rate}
\label{Heat Release Rate}
\label{Liquid Pool Burning Rate}
\label{Rate of Spread}
\label{Scaling Heat Release Rate Per Unit Area}

\begin{figure}[!ht]
\centering
\begin{tabular}{c}
\includegraphics[height=3.5in]{SCRIPT_FIGURES/ScatterPlots/FDS_Heat_Release_Rates}
\end{tabular}
\caption[Summary of heat release rate predictions]{Summary of heat release rate predictions.}
\label{Heat_Release_Rate}
\end{figure}


\begin{figure}[!ht]
\centering
\begin{tabular}{c}
\includegraphics[height=3.5in]{SCRIPT_FIGURES/ScatterPlots/FDS_Burning_Rates}
\end{tabular}
\caption[Summary of burning rate predictions]{Summary of burning rate predictions.}
\label{Burning_Rate}
\end{figure}


\begin{figure}[!ht]
\centering
\begin{tabular}{c}
\includegraphics[height=3.5in]{SCRIPT_FIGURES/ScatterPlots/FDS_Liquid_Pool_Burning_Rates}
\end{tabular}
\caption[Summary of liquid pool burning rate predictions]{Summary of liquid pool burning rate predictions.}
\label{Liquid_Pool_Burning_Rate}
\end{figure}

\begin{figure}[!ht]
\centering
\begin{tabular}{c}
\includegraphics[height=3.5in]{SCRIPT_FIGURES/ScatterPlots/FDS_Scaling_HRRPUA}
\end{tabular}
\caption[Summary of scaling heat release rate per unit area predictions]{Summary of scaling heat release rate per unit area predictions.}
\label{Scaling_Heat_Release_Rate_Per_Unit_Area}
\end{figure}

\begin{figure}[!ht]
\centering
\begin{tabular}{c}
\includegraphics[height=3.5in]{SCRIPT_FIGURES/ScatterPlots/FDS_Rate_of_Spread}
\end{tabular}
\caption[Summary, Wildfire Rate of Spread]
{Summary, Wildfire Rate of Spread.}
\label{RoS_Summary}
\end{figure}




