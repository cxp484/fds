% !TEX root = FDS_Technical_Reference_Guide.tex

\typeout{new file: Introduction_Chapter.tex}

\chapter{Introduction}
\subsubsection*{Howard Baum, NIST Fellow Emeritus}

The idea that the dynamics of a fire might be studied numerically dates back
to the beginning of the computer age. Indeed, the fundamental
conservation equations governing fluid dynamics, heat transfer, and
combustion were first written down over a century ago.
Despite this, practical mathematical models of fire
(as distinct from controlled combustion) are relatively recent due
to the inherent complexity of the problem.
Indeed, in his brief history of the early days of fire research,
Hoyt Hottel noted ``A case can be made for fire being,
next to the life processes, the most complex of phenomena to understand''~\cite{Hottel:1}.

The difficulties revolve about three issues:
First, there are an enormous number of possible fire scenarios
to consider due to their accidental nature. Second, the physical
insight and computing power required to perform all the necessary calculations
for most fire scenarios are limited. Any fundamentally based study of fires
must consider at least some aspects of bluff body aerodynamics, multiphase flow,
turbulent mixing and combustion, radiative transport, and conjugate heat transfer;
all of which are active research areas in their own right.
Finally, the ``fuel'' in most fires was never intended as such.
Thus, the mathematical models and the data needed to characterize the
degradation of the condensed phase materials that supply the fuel may not be available.
Indeed, the mathematical modeling of the physical and
chemical transformations of real materials as they burn is still in its infancy.

In order to make progress, the questions that are asked have to be greatly simplified.
To begin with, instead of seeking a methodology that can be applied to all fire problems,
we begin by looking at a few scenarios that seem to be most amenable to analysis.
Hopefully, the methods developed to study these ``simple'' problems can be generalized
over time so that more complex scenarios can be analyzed.
Second, we must learn to live with idealized descriptions of fires and approximate
solutions to our idealized equations. Finally, the methods should be capable of systematic improvement.
As our physical insight and computing power grow more powerful, the methods of analysis can
grow with them.

To date, three distinct approaches to the simulation of fires have emerged.
Each of these treats the fire as an inherently three dimensional process evolving in time.
The first to reach maturity, the ``zone'' models, describe compartment fires.
Each compartment is divided into two spatially homogeneous volumes, a hot upper layer and a cooler lower layer.
Mass and energy balances are enforced for each layer, with additional models describing other
physical processes appended as differential or algebraic equations as appropriate.
Examples of such phenomena include fire plumes; flows through doors, windows and other vents,
radiative and convective heat transfer; and solid fuel pyrolysis.
Descriptions of the physical and mathematical assumptions
behind the zone modeling concept are given in separate papers by Jones~\cite{Jones:1} and Quintiere~\cite{Quintiere:1},
who chronicle developments through 1983.
Model development since then has progressed to the point where
documented and supported software implementing these models are widely available~\cite{Forney:1}.

The relative physical and computational simplicity of the zone models has led to their
widespread use in the analysis of fire scenarios. So long as detailed spatial
distributions of physical properties are not required, and the two layer description
reasonably approximates reality, these models are quite reliable.
However, by their very nature, there is no way to systematically improve them.
The rapid growth of computing power and the corresponding maturing of computational
fluid dynamics (CFD), has led to the development of CFD based ``field'' models applied to fire research problems.
Virtually all this work is based on the conceptual framework provided by the Reynolds-averaged form of the
Navier-Stokes equations (RANS), in particular the $k -\epsilon$ turbulence model pioneered by
Patankar and Spalding~\cite{Patankar:1}. The use of CFD models has allowed
the description of fires in complex geometries, and the incorporation of a wide variety of
physical phenomena. However, these models have a fundamental limitation for fire applications --
the averaging procedure at the root of the model equations.

RANS models were developed as a time-averaged approximation to the conservation equations of fluid dynamics.
While the precise nature of the averaging time is not specified, it is clearly long enough to
require the introduction of large eddy transport coefficients to describe the unresolved fluxes of mass,
momentum and energy. This is the root cause of the smoothed appearance of the results of even the most
highly resolved fire simulations. The smallest resolvable length scales are determined by the product
of the local velocity and the averaging time rather than the spatial resolution of the underlying computational grid.
This property of RANS models is typically exploited in numerical computations by using implicit
numerical techniques to take large time steps.

Unfortunately, the evolution of large eddy structures characteristic of most fire plumes is lost
with such an approach, as is the prediction of local transient events. It is sometimes
argued that the averaging process used to define the equations is an ``ensemble average'' over many
replicates of the same experiment or postulated scenario. However, this is a moot point in
fire research since neither experiments nor real scenarios are replicated in the sense required
by that interpretation of the equations. The application of ``Large Eddy Simulation'' (LES)
techniques to fire is aimed at extracting greater temporal and spatial fidelity from simulations
of fire performed on the more finely meshed grids allowed by ever faster computers.

The phrase LES refers to the description of turbulent mixing of the gaseous fuel and combustion
products with the local atmosphere surrounding the fire. This process, which determines the burning
rate in most fires and controls the spread of smoke and hot gases, is extremely difficult
to predict accurately. This is true not only in fire research but in almost all phenomena
involving turbulent fluid motion. The basic idea behind the LES technique is that the eddies
that account for most of the mixing are large enough to be calculated with reasonable
accuracy from the equations of fluid dynamics. The hope (which must ultimately be justified
by comparison to experiments) is that small-scale eddy motion can either be crudely accounted for or ignored.

The equations describing the transport of mass, momentum, and energy by the fire-induced flows must
be simplified so that they can be efficiently solved for the fire scenarios of interest.
The general equations of fluid dynamics describe a rich variety of physical processes,
many of which have nothing to do with fires. Retaining this generality would lead to an
enormously complex computational task that would shed very little additional insight on fire dynamics.
The simplified equations, developed by Rehm and Baum~\cite{Rehm:1}, have been widely adopted
by the larger combustion research community, where they are referred to as the ``low Mach number''
combustion equations. They describe the low speed motion of a gas driven by chemical heat release and buoyancy forces.
The phrase low speed refers to gas velocities less than a Mach number of 0.3 (100~m/s); however, it should be noted that
current validation of FDS only extends to a Mach number of approximately 0.1.
Oran and Boris provide a useful discussion of the technique as applied to various reactive flow regimes in the chapter
entitled ``Coupled Continuity Equations for Fast and Slow Flows'' in Ref.~\cite{Oran:1}.
They comment that ``There is generally a heavy price for being able to use a single algorithm for both
fast and slow flows, a price that translates into many computer operations per time step often spent in
solving multiple and complicated matrix operations.''

The low Mach number equations are solved numerically by dividing the physical space where
the fire is to be simulated into a large number of rectangular cells. Within each cell the gas
velocity, temperature, etc., are assumed to be uniform; changing only with time.
The accuracy with which the fire dynamics can be simulated depends on the number of cells
that can be incorporated into the simulation. This number is ultimately limited
by the computing power available. Present day, single processor desktop computers limit the number of
such cells to at most a few million. This means that the ratio of largest to smallest eddy length
scales that can be resolved by the computation (the ``dynamic range'' of the simulation) is on the order of 100.
Parallel processing can be used to extend this range to some extent, but
the range of length scales that need to be accounted for if all relevant
fire processes are to be simulated is roughly $10^4$ to $10^5$ because combustion processes take place at
length scales of \SI{1}{\milli m} or less, while the length scales associated with building fires are of the order of
tens of meters. The form of the numerical equations discussed below depends on which end of the
spectrum one wants to capture directly, and which end is to be ignored or approximated.


